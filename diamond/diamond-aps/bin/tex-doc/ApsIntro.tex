	



\part{Diamond Advanced Planning System}
TODO document how overshipment works
\chapter{Introduction}
The Diamond Advanced Planning System (APS) is the most feature-rich inventory planning system in the market today. There is no other planning engine out there that can match the Diamond APS in terms of speed and versatility. Diamond distribution was designed to address the supply chain planning and execution requirements of distributors.  Diamond was custom designed to address the rigorous specifications of the Aerospace industry, which imposes significantly tighter controls and tracking functionality than a normal distribution environment.

Aeros

Features
Supports Synchronous and Asynchronous planning queues. Parts in the synchronous planning queue are planned the instant they are put into the queue. The planning engine plans parts in the asynchronous planning queue when it reaches the top of the queue. The asynchronous planning queue is activated every time a part is inserted into the queue.

\section{Demand Types}
Supports multiple types of Demand. Demand types supported are
\begin{itemize}
\item Firm Customer Order
\item Work Orders (Components of Work Orders)
\item Safety Stock
\item Forecast
\end{itemize}
\section{Supply Types}
Supports multiple types of Supply. Supply Types supported are
\begin{itemize}
\item On-Hand at Location
\item In-Transit from another Location
\item Purchase Orders
\item Work Orders 
\end{itemize}
\section{Supply Partitioning}
Physical Segregation of Inventory is done using Facilities

Logical Segregation of Inventory within a facility is done using Supply Pools

Sourcing rules attached to every demand specify which supply to consume and the order in which to consume. 

The most commonly used database entities are cached in memory. Cached entities include Facilities, Supply Pools, Sourcing Rules, Certifications, Business Calendar, Equivalent Part Numbers, Customer specific Substitutes, Global Substitutes, Organizations, Forecast Groups, Application Control variables and Purchase Order Equivalents. Caching saves on the expensive operations of database access for the most commonly used entities.

Triggers on cached tables signal the Planning engine to refresh those values

Part Number, substitutes and equivalents are all planned together

\section{Qualifying Inventory}
\begin{itemize}
\item Ability to specify the revision level required for Sales Order and Work Order, Safety Stock and Forecast. The revision level hierarchy setup in the Parts Master will automatically allocate a higher revision level if the requested revision level is not available.

\item Ability to specify certification requirements for Sales Order and Work Order and associate it with the demand.

\item Certification requirements for Safety Stock and Forecasts are picked up from the Customer Master

\item Each certification has a weight associated with it. APS uses qualifying lots with the lowest weight to satisfy any demand. 

\item Ability to specify the manufacturer required at the Demand level or can specify a list of approved manufacturers at the Customer Level. Customer List can be set to Include or Exclude approved manufacturers

\item Ability to specify Country of Origin on the Sales Orders

\item Ability to specify “manufactured after” or “must not expire before” date on Sales Orders for parts with a shelf life.

\item Automatically excludes expired part numbers

\item Kits can be setup in the system to not have mixed manufacturer lots allocated to them for any single kit. APS automatically determines the lot quantity and how many full kits it can satisfy and then switches to the next available lot to satisfy any remaining kits.
\end{itemize}
\section{Demand Prioritization}
Demand Prioritization module compares every demand for a given set of part numbers and sorts them in the following order
\begin{itemize}
\item Sales Orders
\item Work Orders
\item Safety Stock
\item Forecasts
\end{itemize}
\section{Preserve Allocations}
Once sorted, Demand Prioritization also preserves On-hand allocations to Customer Orders and Work Orders within lead-time. 
All allocations to Safety Stock are preserved. 
Preserving existing allocations prevents demands that have been put in later from taking stock away from already existing orders. 
Safety Stock demand is allocated to Customer Orders and Work Orders within the same forecast groups if there is a shortage. 
Forecast Demands are reduced by the quantity of open Customer Order demands for the same month for the same forecast group, which stops over stating of demand for a given month.

\section{Supply Prioritization}
Supply prioritization uses the sourcing rules to determine which supplies to use to satisfy a given demand. Once the qualifying supplies are identified, they are then sorted based on the type of the demand and type of the supply. An example for this would be to use the oldest lots to satisfy open sales orders while using the newest possible lots to satisfy safety stock demand. Since safety stock demand is never shipped, it blocks the newer inventory allowing the older lots to ship out before the newer ones. Supply prioritization changes the FIFO order for parts based on the settings in the Parts Master. Parts with a shelf life can be consumed based on the Manufacture Date or on the Expiration Date of the lots. Supply Prioritization also automatically relaxes all the constraints on the demand when allocating consignment or buyback supply to a demand. Buyback and consignment supply is stock received from the customer that is shipped back to them when they need it. This stock is always deemed to meet customer requirements.

\begin{itemize}
\item Allocations against on-hand supply are classified as Firm or Planned depending on if the on-hand supply is readily available in the primary facility or if the on-hand supply is a planned facility transfer or a processed facility transfer in transit to the primary facility. The primary facility for a demand is identified based on the sourcing rule used to determine the eligible supply for the demand.

\item APS automatically creates work orders for kits. Since APS supports multi-level Bills of Material, it creates work orders for sub-kits and re-plans all the items recursively till all the demands for kits have been allocated either to on-hand inventory or to a work order. 

\item Purchase Orders schedules that are late are automatically padded by a user-defined factor and pushed forward. This enables the system to provide realistic availability dates for the demands which are allocated to those PO schedules.

\item Automatically allocates demands to a Purchase Order if the demand is “X” days out in the future and there is a PO Schedule coming available “X” days before the demand is due. The value of “X” is read in from a Control table.

\item APS will suggest a optimum reschedule date for the PO Schedules that have allocations against demands that need to be expedited or rescheduled to come in at a date later than the current promise date provided by the vendor

\item APS will also suggest cancellation of PO schedules that are not needed to meet any demand that is present in the system.

\end{itemize}


Allocation logic fully traceable. An XML log file may be created created for each item group planned detailing each demand and all supplies, which ones were allocated and which ones were rejected and the reason for rejection.

Ability to bind a given supply to a given demand as long as the supply is qualified for the demand. Allocations once bound are held bound unless unbound by the user.

\section{APS Output}
The Inventory planning process is the most impacted by running Diamond APS. The APS output is fully web-based and provides the Inventory planners with all the information required to make sound buying decisions. Inventory planners have the ability to lookup shortfalls by specifying a whole range of filter conditions. Listed below are the details of the outputs provided by Diamond APS.

\section{Shortages}
Diamond APS classifies shortages into the following categories
\begin{itemize}
\item Unallocated Customer Orders
\item Unallocated Work Orders
\item Unallocated Safety Stock
\item Unallocated Forecasts
\item Customers Orders allocated beyond the requested date
\item Work Orders allocated beyond the requested date
\item Safety Stock allocated beyond the requested date
\item Forecasts allocated beyond the requested date
\end{itemize}
Users can choose a combination of any of these shortage conditions and then apply the following filters to narrow their search

Part Number Mask (A Wildcard search for a range of Part Numbers)

The Part Category. Normally buyers are responsible for purchasing a certain category or categories of parts. This help narrow the results to only the parts they are responsible for purchasing. Within Lead Time. This restricts the output only to shortages that occur within the lead time for a given Part
Outstanding Vendor Quotes less than “X” days. This further narrows the search and ignores the parts that have outstanding vendor quotes that are less than “X” days old. Vendors normally take some time to respond to quotes and this help buyers from seeing the same parts on the list even after they have worked on it.
Planning Horizon End Date. This restricts the list of parts being shown to have shortages only within the Date specified here
Buyer. This only shows the parts the specified buyer is responsible for buying.
Customer Code. This restricts the list of parts only to the shortages for the customer specified.
Maximum Part to display. The default is set to 100. The users can specify any number greater than 0.

Once the search criteria is specified, APS will go through its planning results and find all the part numbers that match the specified search criteria. It will then sort them into 4 groups. 
\begin{itemize}
\item Unallocated or Late Customer Orders
\item Unallocated or Late Work Orders
\item Unallocated or Late Safety Stock
\item Unallocated or Late Forecasts
\end{itemize}
A Part will only appear in one of these groups, the group in which the part has the earliest shortage. Each part then links off into a 12-month time-phased view of the Demand and Supply outlook. The time-phased output has columns for past due, current, 12 months starting with the current month and a column for demands and supplies coming in beyond 12 months. This page also provides links to see the following information
The Allocation Trace Log. This file contains a complete log of the allocation process for the part and all its substitutes and equivalents. Provides a listing of all the supplies available to allocation and also lists each demand followed by which supplies were allocated to it and which supplies were not and also provides a reason for ineligible supplies for the Demand.

Work Book. The Work Book create an excel spread sheet with the following information about the Part. The On-Hand inventory summary by Lot, Facility and Supply Pool, Customer and Vendor Quotes for the Part, Open Purchase Orders, Open Sales Orders, Shipments and a 12 month forecast by forecast group. The work book can then be saved of and helps buyers maintain a log of the demand/supply scenario at the time they made any purchase.
A listing of all available supply and which demands it is allocated to
A listing of all demands and what supplies are allocated to it
Approved Manufacturers. Provides a cross-tab view of the customers and their approved manufacturers. Helps buyers make a choice of buying from the manufacturer that will satisfy the most number of customers.

Customer Quotes Listing

Vendor Quotes Listing

Rescheduling information for any PO Schedules in the system
Shipment Details. Provides a listing of all the shipment of this part to any customer.
Forecast History. Provides a time-phased display of the forecast history by forecast group for the given part. Also lists the forecasts by forecast group.
Shipment Summary. Provides a year-month cross-tab of shipment of this part.
Shipment Summary by Customer. Provides a year-month cross-tab of shipment by customer of this part.
Ability to lookup shortfalls by the following
\begin{itemize}
\item Item Certification
\item Manufacturer and vendor certification 
\item Lists shortfalls based on explicit certifications requested on the Demand. A review of these might help the user offer 
\item Country of Origin
\item Approved Manufacturer
\item Explicit Manufacturer Requested on the Demand
\item Revision Level
\item Re-certification opportunities
\end{itemize}

Ability to lookup Rescheduling Requirements in the following groups
\subsubsection{Purchase Orders to be Expedited}
Provides a summary by Vendor of the Purchase Orders that need to be expedited to meet current and forecasted demand. Users can then drill down into each Vendor and look at each individual PO Schedules need to be expedited and the system also suggested expedite date taking into account the time required to process the receipt after it arrives.
\subsubsection{Purchase Orders to be Rescheduled}
Provides a summary by Vendor of the Carrying Cost and the Cancelable Cost for PO Schedules that can either be pushed out for cancelled. Users can then drill down into each Vendor and look at each individual PO Schedule that needs to be rescheduled. For PO schedules that need to be pushed out, the system suggests the new date by which they are required.
\subsubsection{Purchase Orders to be Cancelled}
Provides a summary by Vendor of the cancelable cost of outstanding PO Schedules. Users can then drill down into each Vendor and see the individual PO Schedules that need to be cancelled.





