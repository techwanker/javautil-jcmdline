\part{Technical}

\chapter {Principle of Operation}
\section{Queue Populator}
The queue populator inserts items to be planned into the request queue.

The only implementation at this time is a stored procedure \textit{APS\_RQST\_QUEUE\_POPULATOR}

\section{Queue Manager}
The queue manager is responsible for managing the queue of items for which planning has been requested.

The request queue is fully read and Planning Groups are established.  
The priority for dispatching is established.

A call to 
\begin{verbatim}
int[] getPlanningItems()
\end{verbatim}
Will return all of the items in a planning group.

\section{Planning Engine}

The planning engine obtains the identifiers for a set of items to be planned, either through a pull from 
the engine to the Queue Manager or through a push by an invocation of 
\begin{verbatim}
 public void planItems(int[] itemIds)
\end{verbatim}
\subsection{Populate Data}

All of the data necessary to plan an item are populated in PlanningData and inter object references are 
resolved.  Various dimension objects are populated from the Cache Manager \ref{CacheManager}
There are two implementation of Persistence interface, one which is based on generated and hand tweaked JDBC
and one which is backed by Hibernate, using a few tweaks to optimize performance.

\subsection{Planning}


\chapter{Architecture}
\section{Language}
Most of the logic in APS is expressed in Java.  

\section{Inversion of Control}
\section{Interface}
\begin{verbatim}
package com.dbexperts.diamond.planning;

public interface PlanningOperation {
        public void process(PlanningData planningData);
}
\end{verbatim}
Some may object to exposing all of the data available for arbitrary operations but and 

\section{Persistence}
APS employs the DAO paradigm with default implementation for persistence using Hibernate for 
Object Relational Mapping.  
\section{Planning Data and Related Interfaces}

\section{Planning Modes}
\subsection{Batch}
Batch planning is equivalent to fully regenerative MRP.

When no more items are in the request queue the program terminates.
\subsection{Interactive}
\section{Queue Manager}
The queue manager is responsible for monitoring all of the items in the request queue and creating a queue of
groups of planning items to be dispatched.

See
\section{CommandLineExecution}
java com.dbexperts.diamond.cli dataSourceName

TODO should create a big fat jar

in /common/home/jjs/workspace/Diamond11 ./run\_planning

note document EJB3PlanningDataPersistence