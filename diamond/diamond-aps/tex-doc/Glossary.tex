\section{Glossary}
\paragraph{Allocation}
\paragraph{Allocation Based Pricing} It is a common practice in the aerospace industry for a distributor to provide
Just-In-Time fulfillment of products under contractual arrangements.  Often the distributor will purchase 
an airframe supplier's inventory and sell it back for cost plus a service fee although other arrangements are
certainly not rare.  After the inventory that was purchased is exhausted additional demand is satisfied by
inventory of the distributor.  It is not known at the time the order is placed the source of the inventory 
that will be used to satisfy the demand.  Allocation based pricing updates the order line with the appropriate price
and creates sublines if necessary in the case of 
\paragraph{Approved Manufacturer}
\paragraph{Bound Allocation}
A
\paragraph{C of C}
See ??? Manufacturer Certificate of Compliance
\paragraph{Country of Origin}
\paragraph{Dispatch Group}
A set of items that are retrieved concurrently from the underlying datastore for the purpose of improving 
performance of the persistent data store.  This is a technical and not a functional issue.
\paragraph{Eligible Supply}
For each demand any supply that passes all filters is considered eligible supply.
\paragraph{Engineered Parts}
An engineered part is an item produced in compliance with an engineering
drawing.  Most everyone is familiar
with Stock Keeping Unit or SKU, usually identified by a UPC (Universal Product
Code) or an EAN (European ???).

An engineered part may be produced by a wide variety of manufacturers.
\paragraph{Equivalent Part}
Two parts are equivalent if they could be cross certified.   

For example AN960-10 and NAS1149 are very common flat washers and the former diagrams where drawn
by the Army Navy coalition and the latter by ???

These parts are identical, the only difference being whether the manufacturer 
has provided a \textit{Manufacturer Certificate of Compliance}.  

One of the byproducts of APS is the ability to query for lots which may be made suitable for the demand by
simply buying a \textit{Manufacturer Certificate of Compliance}, another query will show unsatisified demand
that could be satisfied by changing the requested part to be the equivalent part which frequently is 
readily accommodated by the party that gave rise to the demand. 
\paragraph{Forecast Consumption}
The process of decreasing Forecasted Demand by the amount of firm orders.  

This is necessary to eliminate ``double dipping'' requirements for supply.
\paragraph{Forecasting Granularity}
TODO bucketed by week in TMP\_APS\_DMD\_FC\_PRD
\paragraph{Forecast Group}
\paragraph{Hard Alloc Fc Within Lead Time}
APS\_DMD\_FC
\paragraph{In Pick}
\paragraph{Manufacturer Certificate of Compliance}
A physical or electronic document issued by the manufacturer of an engineered part specifying a specific 
lot or set of lots with assurances backed solely by the credibility of the manufacturer that the items in
the lot comply with the specifications set forth on the engineering diagrams for the part.
\paragraph{Manufacturer Lot Code}
A designator assigned by the manufacturer for items produced, generally on the same machines after setup and
demarcated by policies set forth by the manufacturer which may include quantity produced and machine setup,
internal quality tests and so forth.
\paragraph{Multicerted Item}


\paragraph{Pre-approved Lot}  
A lot which has been pre-approved by the organization ordering.
\paragraph{Planning Data}
Planning Data is the domain of all information necessary to generate a planning result for a \ref{PlanningGroup} planning group.
\label{PlanningGroup} 
\paragraph{Planning Group}
A planning group consists of all related items whether by equivalence, or substitution to arbitrary
depths until such relationships are exhausted.  This is performed procedurally in memory as all attempts
to effect this declaratively have resulted in quantitative run times three or four orders of magnitude greater
than object traversal.

\paragraph{Receiver Number}
A receiver number is created by Diamond Distribution for each unique manufacturer lot on a shipping manifest.

Therefor it is quite possible that 

\paragraph{Safety Stock Consumption}

\paragraph{Substitute Part}
A customer may agree for a substitute for a part when form, fit and function are satisfied irrespective 
of whether the part is an engineered part.

For example 'D' batteries from Everready and Ray-O-Vac with the same number of ampere hours.
\paragraph{UPC}
\paragraph{Quality Assurance Tests}
\section{Planning Mode}
Allocation has different behavior based on the allocation mode.  The aerospace
planning implementation supports 
four phases each with a different mode.

\subsection{Pick Restore}
Existing allocations to onhand inventory from work orders or customer orders
that are in pick status are preserved.

This Mode is also used in binding allocations.

An allocation is bound by updating the alloc\_type\_id to 'R' and requesting the
engine to bind the allocation.  
The alloc\_type\_id is then set to 'B' for Bound.
\subsection{First Pass}
The first pass of allocation attempts to restore prior allocations to those to whom inventory
has been committed.  The provided mechanism generally accommodates this on a first come, first served basis.

Delivery schedules from manufacturing or manufacturers are often revised due to cyclicality in production 
capacity.  In aerospace this arises from a scarcity of materials or heavy handed pricing requirements 
from airframe manufacturers that drive profit margins so low that manufacturers exit the business; when 
production is ramped up, fewer providers may quote lead times in excess of 18 months and slip on even
these extended schedules.

As with most everything in planning, this is a pluggable module and any program (``java class'') that conforms to 
the required call specifications (``interface'') may be substituted for the default mechanism.
\subsection{Customer Prioritized}
Customer Prioritization is a phase in which inventory allocated to a customer may be shifted 
from later demand dates to earlier demand dates for inventory allocated to the customer.
This is the default second pass of allocation.
\subsection{Overship}
The request for Overship is used when variations in the requested ship quantity
are supported.  For example a customer orders
1000 washers, after the picker picks the order there are only 10 washers left in
the box.  If the customer allows over shipping then 1010 
washers will be picked for his order as it is not worth the time to put 10
washers back in the box and put them away.

The overship phase ensures that the 10 washers are not needed by another demand
and changes the allocated quantity within.

This is the default third pass of allocation. 
\section{Restore Onhand Pick Allocations}
Picking is in an expensive operation.  Although Diamond Warehouse operations knows the exact location of 
every item other warehouse systems don't have this level of interactivity.

We assume that once an item has been submitted to pick that it is no longer available for any other demand.

In the default implementation this occurs as part of the first pass. ???