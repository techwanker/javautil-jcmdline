\part{Functional}
\section{Introduction}
Diamond Advanced Planning is a module that associates demand for product with
inventory that satisfies the demand according to pluggable (``extensible'') filters and 

Although it was designed explicitly to accomodate the requirements of aerospace
subsets of its 
features will readily support virtually any distribution, MRO or discrete
manufacturing environment.

Diamond Advanced Planning was first written in 1983 by the author of this Document and served well in 
planning for Fine China and giftware.  Same was employed for a clothing manufacturer and subsequently 
for Tri-Star Aerospace.   In 2001 a total rewrite was effected written in Java accommodating Aerospace 
requirements in version 10 of Diamond Distribution.   

\subsection{Features}
\begin{itemize}
 \item Simultaneously supports interactive and batch mode planning
 \item Generates work orders for kits, assemblies and ???
 \item Supports Engineered Parts, a single item may simultaneously have multiple part numbers
 \item Allocation Based Pricing
\end{itemize}

\chapter{Concepts}

\section{Demand Types}
\subsection{Customer Orders}
A customer may be an external organization not within the legal hierarchy of legal entities 
encompassed by the organization that provides the planning service or be subordinate to the penultimate
holding company of the organization providing planning. These relationships have no bearing on the requirement
of the planning organization to fulfill demand in accordance with the contractual rules.

Stated more simply, a customer in the sense of planning is an organization with a specified set of requirements
for the fulfillment of inventory.
\subsection{Forecasted Demand}
Diamond provides a statistical forecasting model that forecasts demand based on historical consumption.
The planning module is agnostic with respect to the source of the forecast and any forecast system may be used.
\subsection{Work Orders}
Work orders may take the form of kits, assemblies or rework.  ???


\section{Supply Types}
\subsection{On hand inventory}
Onhand inventory includes inventory in transit to another warehouse and considers the availability date
the time it will take to ship, receive and putaway the inventory.
\subsection{Replenishments} 
A replenishment is additional inventory coming from an external source. This may be the result of a purchase order
or anticipated receipt of buyback ??? or consignment inventory. 
\subsection{Work Orders}
A work order is used to designate a part which has components, such as kits, assemblies and parts that require
transformation to convert them to another part.  

A zinc plated bolt may be converted to a nickel plated bolt by stripping the zinc, copper plating and then nickel plating.
Each such operation may have a yield of less than 100%.
\subsection{Planning Group}
It is posited that no aerospace planning can be of much value in the absence of planning demand for parts that 
constitute a planning group, that is all parts which have the same form fit and function are interchangeable and 
those parts which have been previously been approved by the consumer of said parts as acceptable substitutes for
the specified part.  
Interchangeability is defined as the parts being equivalent. A single part may comply with multiple engineering
diagrams and when produced by a manufacturer of repute acceptable to the consumer and accompanied by a
manufacturer Certificate of Compliance with said diagram, these parts are considered equivalent even in the 
absence of aforesaid Certificate, for often such certification may be purchased from the manufacturer, or waived
by the knowing consumer.
\section{Sourcing Rules}

\section{Eligibility Tests}
Generally elibility tests may be considered as filters, that is, the order of the tests is inconsequential and 
each test may exclude the supply as being applicable for the demand.  On occassion two or more tests may exhibit
characteristics such that the qualification of either test is sufficient.  Consider the case of buyback inventory
in which inventory has been purchased from an airframe manufacturer but traceability to the ultimate source
is not available.  The airframe manufacturer may state that if the inventory was procured from the manufacturer
under certain conditions those parts may bypass requirements that would otherwise be in place.  It is the 
responsibility of the kitting or JIT provider to provide traceability back to the airframe manufacturer without
any bearing any responsibility for ultimate traceability.

\subsection{Approved Manufacturer}
A customer may define a \textit{white list} or \textit{black list} of manufacturers by part.

 
\subsection{Buyback}
\subsection{Certifications}
Certifications can assume arbitrary meanings.  

Examples of Certifications include for example only
\begin{itemize}
 \item Manufacturer Certificate of Compliance
 \item ??? list additional
\end{itemize}


\begin{itemize}
 \item Manufacturer Certificate of Compliance
 
\end{itemize}

\subsection{Consignment Inventory}
\subsection{Contract Aerospace}
\subsection{Country of Origin}
A demand may specify that the supply must be from a specified country.  A filter exists to enforce this 
requirement.
\subsection{Expiry Date}
A demand may explicate the latest date of expiry for the supply.  This filter ensures that the supply complies with
the demand minimum expiration date.
\subsection{Explicit Manufacturer}
Customers may be set up with rules for approved manufacturers inclusively or exclusively.
\subsection{Lot Date}
A demand may have an associated maximum date of manufacture.
\subsection{Revision Level}

\subsection{Sourcing Rule}

\section{Planning Groups}


\section{Aerospace Features}
\begin{itemize}
 \item Show which items could obtain certification rather than procure
 \item Generate Warehouse Transfers
 \item Generate Work Orders
 \item Overship Capability
\end{itemize}

\section{Planning Mode}
\subsection{Execution Planning Mode}
In this mode demand the supply prioritizer gives higher precedence to the supply associated with delivery
most  prior to ???

\subsection{Inventory Planning Mode}
Inventory Planning Mode attempts to allocate demand such that on hand inventory is exhausted, deferring 
allocation to purchase orders or work orders to the first available by date.  The result of this is
that a reschedule date can be derived from the first requirement date.

\section{Bound Allocations}
A demand may be bound to a supply.  These allocations are restored foremost in the first pass of allocation.
Bound states include
\begin{itemize}
 \item R Request 
 \item B Bound
\end{itemize}

\section{Logging}
All of the decision paths for a planning group are incorporated into a single logging entity reflecting
each of the decisions and prioritization.  This data is persisted in XML, generally in a relational database.
