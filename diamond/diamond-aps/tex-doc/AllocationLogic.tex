\documentclass[a4paper,10pt]{book}


\begin{document}
\section{Introduction}
Diamond Advanced Planning is a module that associates demand for product with inventory that satisfies the demand.

Although it was designed explicitly to accomodate the requirements of aerospace subsets of its 
features will readily support virtually any distribution, MRO or discrete manufacturing environment.

\section{Demand Types}
\subsection{Customer Orders}
\subsection{Forecasted Demand}
\subsection{Work Orders}

\section{Supply Types}
\subsection{On hand inventory}
\subsection{Replenishments} 
\subsection{Work Orders}

\section{Aerospace Features}
\section{Glossary}
\paragraph{Allocation}
\paragraph{Approved Manufacturer}
\paragraph{Bound Allocation}
\paragraph{C of C}
See ??? Manufacturer Certificate of Compliance
\paragraph{Dispatch Group}
\paragraph{Eligible Supply}
\paragraph{Manufacturer Lot Code}
\paragraph{Planning Group}
\paragraph{Receiver Number}
\paragraph{Multicerted Item}

\paragraph{Engineered Parts}
An engineered part is an item produced in compliance with an engineering drawing.  Most everyone is familiar
with Stock Keeping Unit or SKU, usually identified by a UPC (Universal Product Code) or an EAN (European ???).

An engineered part may be produced by a wide variety of manufacturers.
\paragraph{Pre-approved Lot}  
A lot which has been pre-approved by the organization ordering.
\paragraph{Manufacturer Certificate of Compliance}
\paragraph{UPC}
\paragraph{Quality Assurance Tests}
\section{Mode}
Allocation has different behavior based on the allocation mode.  The aerospace planning implementation supports 
four phases each with a different mode.
\subsection{Pick Restore}
Existing allocations to onhand inventory from work orders or customer orders that are in pick status are preserved.

This Mode is also used in binding allocations.

An allocation is bound by updating the alloc\_type\_id to 'R' and requesting the engine to bind the allocation.  
The alloc\_type\_id is then set to 'B' for Bound.
\subsection{First Pass}
\subsection{Customer Prioritized}
\subsection{Overship}
The request for Overship is used when variations in the requested ship quantity are supported.  For example a customer orders
1000 washers, after the picker picks the order there are only 10 washers left in the box.  If the customer allows over shipping then 1010 
washers will be picked for his order as it is not worth the time to put 10 washers back in the box and put them away.

The overship phase ensures that the 10 washers are not needed by another demand and changes the 
\section{Restore Onhand Pick Allocations}

\section{Bound Allocations}

\end{document}
