%% Generated by Sphinx.
\def\sphinxdocclass{report}
\documentclass[letterpaper,10pt,english]{sphinxmanual}
\ifdefined\pdfpxdimen
   \let\sphinxpxdimen\pdfpxdimen\else\newdimen\sphinxpxdimen
\fi \sphinxpxdimen=.75bp\relax

\PassOptionsToPackage{warn}{textcomp}
\usepackage[utf8]{inputenc}
\ifdefined\DeclareUnicodeCharacter
% support both utf8 and utf8x syntaxes
  \ifdefined\DeclareUnicodeCharacterAsOptional
    \def\sphinxDUC#1{\DeclareUnicodeCharacter{"#1}}
  \else
    \let\sphinxDUC\DeclareUnicodeCharacter
  \fi
  \sphinxDUC{00A0}{\nobreakspace}
  \sphinxDUC{2500}{\sphinxunichar{2500}}
  \sphinxDUC{2502}{\sphinxunichar{2502}}
  \sphinxDUC{2514}{\sphinxunichar{2514}}
  \sphinxDUC{251C}{\sphinxunichar{251C}}
  \sphinxDUC{2572}{\textbackslash}
\fi
\usepackage{cmap}
\usepackage[T1]{fontenc}
\usepackage{amsmath,amssymb,amstext}
\usepackage{babel}



\usepackage{times}
\expandafter\ifx\csname T@LGR\endcsname\relax
\else
% LGR was declared as font encoding
  \substitutefont{LGR}{\rmdefault}{cmr}
  \substitutefont{LGR}{\sfdefault}{cmss}
  \substitutefont{LGR}{\ttdefault}{cmtt}
\fi
\expandafter\ifx\csname T@X2\endcsname\relax
  \expandafter\ifx\csname T@T2A\endcsname\relax
  \else
  % T2A was declared as font encoding
    \substitutefont{T2A}{\rmdefault}{cmr}
    \substitutefont{T2A}{\sfdefault}{cmss}
    \substitutefont{T2A}{\ttdefault}{cmtt}
  \fi
\else
% X2 was declared as font encoding
  \substitutefont{X2}{\rmdefault}{cmr}
  \substitutefont{X2}{\sfdefault}{cmss}
  \substitutefont{X2}{\ttdefault}{cmtt}
\fi


\usepackage[Bjarne]{fncychap}
\usepackage{sphinx}

\fvset{fontsize=\small}
\usepackage{geometry}

% Include hyperref last.
\usepackage{hyperref}
% Fix anchor placement for figures with captions.
\usepackage{hypcap}% it must be loaded after hyperref.
% Set up styles of URL: it should be placed after hyperref.
\urlstyle{same}
\addto\captionsenglish{\renewcommand{\contentsname}{Contents:}}

\usepackage{sphinxmessages}
\setcounter{tocdepth}{1}



\title{Diamond Advanced Planning}
\date{Nov 28, 2019}
\release{2019-11}
\author{Jim Schmidt}
\newcommand{\sphinxlogo}{\vbox{}}
\renewcommand{\releasename}{Release}
\makeindex
\begin{document}

\pagestyle{empty}
\sphinxmaketitle
\pagestyle{plain}
\sphinxtableofcontents
\pagestyle{normal}
\phantomsection\label{\detokenize{index::doc}}



\chapter{Indices and tables}
\label{\detokenize{index:indices-and-tables}}\begin{itemize}
\item {} 
\DUrole{xref,std,std-ref}{genindex}

\item {} 
\DUrole{xref,std,std-ref}{modindex}

\item {} 
\DUrole{xref,std,std-ref}{search}

\end{itemize}


\chapter{Purpose}
\label{\detokenize{index:purpose}}
This document is intended to introduced Align to Diamond APS and lay out
Align objectives and a pathroad to satisfy those objectives.

This is intended to be evergreen, to evolve as requirements are determined and to be
fleshed out with technical details.

Personally, I would rather a revisioned, indexed, searchable single document than hundreds of
documents.  I will assume the responsibility of maintaining this.


\chapter{Current State}
\label{\detokenize{index:current-state}}
Align has headquarters in Las Angeles and Paris.

The Las Angeles shop is running a highly modified SAP system.

The Paris operation is running Dymax, which is a COBOL/ISAM VMS application.


\chapter{Future State}
\label{\detokenize{index:future-state}}
Create a central repository of inventory from both systems in a relational
database accessible through a browser.

Use all open source software whenever possible
\begin{quote}

AWS
Redhat 8
Postgres 10
Maven 3.2
Java open jdk 8/Spring/Hibernate
Tomcat/Node/Angular
\end{quote}


\chapter{Align Objectives}
\label{\detokenize{index:align-objectives}}

\section{Team}
\label{\detokenize{index:team}}

\section{Migration Plan}
\label{\detokenize{index:migration-plan}}
Export Dymax ISAM files to flat files
Using javautil fixed record mapping utility populate RDBMS tables that map to Dymax records

Write conversion scripts to convert to Diamond Schema.

Supplement with additional data as necessary

Choose the 100 highest value parts as the basis of test suite of data.

Setup Rules

Augment data - setup multiple equivalents, quality assurance valuations, etc.

Run a plan against the parts and review


\chapter{Introduction}
\label{\detokenize{index:introduction}}
The Diamond Advanced Planning System (APS) is the most feature-rich
inventory planning system in the market today. There is no other
planning engine out there that can match the Diamond APS in terms of
speed and versatility.

Diamond APS is a unique combination of DRP/MRP planning and execution that is aware
of engineered parts and Aerospace constraints.

Diamond distribution was designed to address the
supply chain planning and execution requirements of distributors.
Diamond was custom designed to address the rigorous specifications of
the Aerospace industry, which imposes significantly tighter controls and
tracking functionality than a normal distribution environment.

Features Supports Synchronous and Asynchronous planning queues. Parts in
the synchronous planning queue are planned the instant they are put into
the queue. The planning engine plans parts in the asynchronous planning
queue when it reaches the top of the queue. The asynchronous planning
queue is activated every time a part is inserted into the queue.


\section{Demand Types}
\label{\detokenize{index:demand-types}}
Supports multiple types of Demand. Demand types supported are
\begin{itemize}
\item {} 
Firm Customer Order

\item {} 
Work Orders (Components of Work Orders)

\item {} 
Safety Stock

\item {} 
Forecast

\end{itemize}


\section{Supply Types}
\label{\detokenize{index:supply-types}}
Supports multiple types of Supply. Supply Types supported are
\begin{itemize}
\item {} 
On-Hand at Location

\item {} 
In-Transit from another Location

\item {} 
Purchase Orders

\item {} 
Work Orders

\end{itemize}


\section{Supply Partitioning}
\label{\detokenize{index:supply-partitioning}}

\subsection{Physical Segregration}
\label{\detokenize{index:physical-segregration}}
Physical Segregation of Inventory is done using Facilities


\subsection{Logical Segregation}
\label{\detokenize{index:logical-segregation}}
Logical of Inventory within a facility is done using \sphinxstyleemphasis{Supply Pools}

\sphinxstyleemphasis{Sourcing Rules} attached to every demand specify which supply to consume
and the order in which to consume.

Part Number, substitutes and equivalents are all planned together


\section{Qualifying Inventory}
\label{\detokenize{index:qualifying-inventory}}\begin{itemize}
\item {} 
Ability to specify the revision level required for Sales Order and
Work Order, Safety Stock and Forecast. The revision level hierarchy
setup in the Parts Master will automatically allocate a higher
revision level if the requested revision level is not available.

\item {} 
Ability to specify certification requirements for Sales Order and
Work Order and associate it with the demand.

\item {} 
Certification requirements for Safety Stock and Forecasts are picked
up from the Customer Master

\item {} 
Each certification has a weight associated with it. APS uses
qualifying lots with the lowest weight to satisfy any demand.

\item {} 
Ability to specify the manufacturer required at the Demand level or
can specify a list of approved manufacturers at the Customer Level.
Customer List can be set to Include or Exclude approved
manufacturers

\item {} 
Ability to specify Country of Origin on the Sales Orders

\item {} 
Ability to specify “manufactured after” or “must not expire before”
date on Sales Orders for parts with a shelf life.

\item {} 
Automatically excludes expired part numbers

\item {} 
Kits can be setup in the system to not have mixed manufacturer lots
allocated to them for any single kit. APS automatically determines
the lot quantity and how many full kits it can satisfy and then
switches to the next available lot to satisfy any remaining kits.

Tests include
\begin{itemize}
\item {} 
BuybackTest

\item {} 
CertTest

\item {} 
ConsignmentTest

\item {} 
CountryOfOriginTest

\item {} 
CustomerItemManufacturers

\item {} 
CustomerItemManufacturerTest

\item {} 
ExpiryDateTest

\item {} 
ExpiryDateTest

\item {} 
ExpiryDateTest

\item {} 
IsCustomerSubstitute

\item {} 
IsGlobalSubstitute

\item {} 
IsItemOrSubstitute

\item {} 
IsSubstitute

\item {} 
ManufacturerTest

\item {} 
RevisionLevelTest

\item {} 
SourcingRuleTest

\item {} 
StandardAerospaceEligibilityCompositeTest

\item {} 
SupplyEligibilityTest

\item {} 
SupplyIsBuybackFromCustomerTest

\item {} 
SupplyItemSatifiesDemandItem

\end{itemize}

\end{itemize}


\section{Demand Prioritization}
\label{\detokenize{index:demand-prioritization}}
Demand Prioritization module compares every demand for a given set of
part numbers and sorts them in the following order
\begin{itemize}
\item {} 
Sales Orders

\item {} 
Work Orders

\item {} 
Safety Stock

\item {} 
Forecasts

\end{itemize}


\section{Preserve Allocations}
\label{\detokenize{index:preserve-allocations}}
Once prioritized, Demand Prioritization also preserves On-hand allocations to
Customer Orders and Work Orders within lead-time. All allocations to
Safety Stock are preserved. Preserving existing allocations prevents
demands that have been put in later from taking stock away from already
existing orders. Safety Stock demand is allocated to Customer Orders and
Work Orders within the same forecast groups if there is a shortage.
Forecast Demands are reduced by the quantity of open Customer Order
demands for the same month for the same forecast group, which stops over
stating of demand for a given month.


\section{Supply Prioritization}
\label{\detokenize{index:supply-prioritization}}
Supply prioritization uses the sourcing rules to determine which
supplies to use to satisfy a given demand.

Once the qualifying supplies are identified, they are then sorted based on the type of the demand and
type of the supply.

An example for this would be to use the oldest lots
to satisfy open sales orders while using the newest possible lots to
satisfy safety stock demand. Since safety stock demand is never shipped,
it blocks the newer inventory allowing the older lots to ship out before
the newer ones. Supply prioritization changes the FIFO order for parts
based on the settings in the Parts Master. Parts with a shelf life can
be consumed based on the Manufacture Date or on the Expiration Date of
the lots. Supply Prioritization also automatically relaxes all the
constraints on the demand when allocating consignment or buyback supply
to a demand. Buyback and consignment supply is stock received from the
customer that is shipped back to them when they need it. This stock is
always deemed to meet customer requirements.
\begin{itemize}
\item {} 
Allocations against on-hand supply are classified as Firm or Planned
depending on if the on-hand supply is readily available in the
primary facility or if the on-hand supply is a planned facility
transfer or a processed facility transfer in transit to the primary
facility. The primary facility for a demand is identified based on
the sourcing rule used to determine the eligible supply for the
demand.

\item {} 
APS automatically creates work orders for kits. Since APS supports
multi-level Bills of Material, it creates work orders for sub-kits
and re-plans all the items recursively till all the demands for kits
have been allocated either to on-hand inventory or to a work order.

\item {} 
Purchase Orders schedules that are late are automatically padded by
a user-defined factor and pushed forward. This enables the system to
provide realistic availability dates for the demands which are
allocated to those PO schedules.

\item {} 
Automatically allocates demands to a Purchase Order if the demand is
“X” days out in the future and there is a PO Schedule coming
available “X” days before the demand is due. The value of “X” is
read in from a Control table.

\item {} 
APS will suggest a optimum reschedule date for the PO Schedules that
have allocations against demands that need to be expedited or
rescheduled to come in at a date later than the current promise date
provided by the vendor

\item {} 
APS will also suggest cancellation of PO schedules that are not
needed to meet any demand that is present in the system.

\end{itemize}

Allocation logic fully traceable. An XML or JSON log file may be created created
for each item group planned detailing each demand and all supplies,
which ones were allocated and which ones were rejected and the reason
for rejection.

Ability to bind a given supply to a given demand as long as the supply
is qualified for the demand. Allocations once bound are held bound
unless unbound by the user.


\section{APS Output}
\label{\detokenize{index:aps-output}}
The Inventory planning process is the most affect by running Diamond
APS. The APS output is fully web-based and provides the Inventory
planners with all the information required to make sound buying
decisions. Inventory planners have the ability to lookup shortfalls by
specifying a whole range of filter conditions. Listed below are the
details of the outputs provided by Diamond APS.


\section{Projected Inventory Position}
\label{\detokenize{index:projected-inventory-position}}
\sphinxstyleemphasis{PIP} Projected Inventory Position.

Each supply has an \sphinxstyleemphasis{available date}

For on-hand inventory that is the current date or the \sphinxstyleemphasis{effective date} in case of some
simulations.

For purchase orders that is the \sphinxstyleemphasis{current promise date}.

For work orders that is the \sphinxstyleemphasis{need by date}.

A timeline is created by generating \sphinxstyleemphasis{buckets}, typically calendar months.

Each supply has its own PIP, which is strictly decreasing.  Allocations of demand are bucketed based on \sphinxstyleemphasis{current promise date}
or the associated supply \sphinxstyleemphasis{available date}, whichever is later.

In the aggregate positions can increase and decrease as in a traditional DRP system.  Aggregations can
be at the part, supply pool, facility level or any combination thereof.

If the availability date for a demand is greater than the \sphinxstyleemphasis{current promise date}, there is a shortage.  This shortage
may be actionable
\begin{itemize}
\item {} 
Expedite the purchase order

\item {} 
Relieve other supply to satisfy this demand by changing its supply pool

\item {} 
Create a new requisition

\end{itemize}

If the earliest demand date for a supply falls in a later bucket than the supply availability this may be actionable:
\begin{itemize}
\item {} 
Reschedule the Purchase Order or Work Order

\end{itemize}

Consideration must be made to eliminate \sphinxstyleemphasis{nervouos} adjustments, those for a short time period or for dollar amounts
such that de-expediting costs exceed the time value of the early receipt

In traditional DRP the additional replenishment is an Economic Order Quantity \sphinxstyleemphasis{EOQ}

This is not applicable to aerospace due to the unit price sensitivity to order quantity.

You probably know what an \sphinxstyleemphasis{EOQ}, if not a quick refresher is just a \sphinxstyleemphasis{google} or \sphinxstyleemphasis{duckduckgo} search away.

This model also fails to take into consideration a myriad of constraints which may
make a given supply ineligible for a given demand.  This is a model for SKU inventory,
not for engineered parts.

Note that orders and forecast are both demands but are not aggregated, orders \sphinxstyleemphasis{consume
the forecast}.

Optimal Replenishment Quantity


\section{Shortages}
\label{\detokenize{index:shortages}}
Diamond APS classifies shortages into the following categories
\begin{itemize}
\item {} 
Unallocated Customer Orders

\item {} 
Unallocated Work Orders

\item {} 
Unallocated Safety Stock

\item {} 
Unallocated Forecasts

\item {} 
Customers Orders allocated beyond the requested date

\item {} 
Work Orders allocated beyond the requested date

\item {} 
Safety Stock allocated beyond the requested date

\item {} 
Forecasts allocated beyond the requested date

\end{itemize}

Users can choose a combination of any of these shortage conditions and
then apply the following filters to narrow their search

Part Number Mask (A wildcard search for a range of Part Numbers)

The Part Category. Normally buyers are responsible for purchasing a
certain category or categories of parts. This help narrow the results to
only the parts they are responsible for purchasing.

Within Lead Time.  This restricts the output only to shortages that occur within the lead
time for a given

Part Outstanding Vendor Quotes less than “X” days. This
further narrows the search and ignores the parts that have outstanding
vendor quotes that are less than “X” days old. Vendors normally take
some time to respond to quotes and this help buyers from seeing the same
parts on the list even after they have worked on it.

Planning Horizon End Date. This restricts the list of parts being shown to have shortages
only within the Date specified here Buyer. This only shows the parts the
specified buyer is responsible for buying.

Customer Code. This restricts the list of parts only to the shortages for the customer specified.
Maximum Part to display. The default is set to 100. The users can specify any number greater than 0.

Once the search criteria is specified, APS will go through its planning
results and find all the part numbers that match the specified search
criteria. It will then sort them into 4 groups.
\begin{itemize}
\item {} 
Unallocated or Late Customer Orders

\item {} 
Unallocated or Late Work Orders

\item {} 
Unallocated or Late Safety Stock

\item {} 
Unallocated or Late Forecasts

\end{itemize}

A Part will only appear in one of these groups, the group in which the
part has the earliest shortage. Each part then links off into a 12-month
time-phased view of the Demand and Supply outlook. The time-phased
output has columns for past due, current, 12 months starting with the
current month and a column for demands and supplies coming in beyond 12
months. This page also provides links to see the following information
The Allocation Trace Log. This file contains a complete log of the
allocation process for the part and all its substitutes and equivalents.
Provides a listing of all the supplies available to allocation and also
lists each demand followed by which supplies were allocated to it and
which supplies were not and also provides a reason for ineligible
supplies for the Demand.

Work Book. The Work Book create an excel spread sheet with the following
information about the Part.

The On-Hand inventory summary by Lot, Facility and Supply Pool,
\begin{itemize}
\item {} 
Customer Quotes

\item {} 
Vendor Quotes

\item {} 
Open Purchase Orders

\item {} 
Open Sales Orders

\item {} 
Shipments and a forecast by forecast group.

\item {} 
Provides a cross-tab view of the customers and their approved manufacturers. Helps buyers make a choice of buying from the manufacturer that will satisfy the most number of customers.

\end{itemize}

The work book can then be saved of and helps buyers
maintain a log of the demand/supply scenario at the time they made any
purchase. The system takes a snapshot of the full state of planning at the time
of requisition approval.

A listing of all available supply and which demands it is
allocated to A listing of all demands and what supplies are allocated to
it Approved Manufacturers.

Rescheduling information for any PO Schedules in the system Shipment
Details. Provides a listing of all the shipment of this part to any
customer. Forecast History. Provides a time-phased display of the
forecast history by forecast group for the given part. Also lists the
forecasts by forecast group. Shipment Summary. Provides a year-month
cross-tab of shipment of this part. Shipment Summary by Customer.
Provides a year-month cross-tab of shipment by customer of this part.
Ability to lookup shortfalls by the following
\begin{itemize}
\item {} 
Item Certification

\item {} 
Manufacturer and vendor certification

\item {} 
Lists shortfalls based on explicit certifications requested on the
Demand. A review of these might help the user offer

\item {} 
Country of Origin

\item {} 
Approved Manufacturer

\item {} 
Explicit Manufacturer Requested on the Demand

\item {} 
Revision Level

\item {} 
Re-certification opportunities

\end{itemize}

Ability to lookup Rescheduling Requirements in the following groups


\subsection{Purchase Orders to be Expedited}
\label{\detokenize{index:purchase-orders-to-be-expedited}}
Provides a summary by Vendor of the Purchase Orders that need to be
expedited to meet current and forecasted demand. Users can then drill
down into each Vendor and look at each individual PO Schedules need to
be expedited and the system also suggested expedite date taking into
account the time required to process the receipt after it arrives.


\subsection{Purchase Orders to be Rescheduled}
\label{\detokenize{index:purchase-orders-to-be-rescheduled}}
Provides a summary by Vendor of the Carrying Cost and the Cancelable
Cost for PO Schedules that can either be pushed out for cancelled. Users
can then drill down into each Vendor and look at each individual PO
Schedule that needs to be rescheduled. For PO schedules that need to be
pushed out, the system suggests the new date by which they are required.


\subsection{Purchase Orders to be Cancelled}
\label{\detokenize{index:purchase-orders-to-be-cancelled}}
Provides a summary by Vendor of the Cancelable cost of outstanding PO
Schedules. Users can then drill down into each Vendor and see the
individual PO Schedules that need to be cancelled.


\subsubsection{Align Future State}
\label{\detokenize{FutureState:align-future-state}}\label{\detokenize{FutureState::doc}}

\paragraph{Objectives}
\label{\detokenize{FutureState:objectives}}\begin{enumerate}
\sphinxsetlistlabels{\arabic}{enumi}{enumii}{}{.}%
\item {} 
Maximum buyer productivity

\item {} 
Eliminate unnecessary purchases

\item {} 
Develop a standardized methodology for buyers that is deterministic,
with the same input two buyers should come to the same conclusions.

\item {} 
Buyers should be able to test the results of a simulation by, for
example adding a secondary manufacturer CofC to a lot and replan the
part and get the answer back within a second with a new single
screen.

\item {} 
All scenarios should be stored with the results.

\item {} 
Upon acceptance of a scenario the simulation changes should be
reported so that the source systems can be update.

\item {} 
Status of simulation changes \# Requested \# Not possible - stops
further recommendations to take this action \# Active - Once a
download from the source system reflects this change

\item {} 
A report of requested modification not yet completed on source
systems

\end{enumerate}


\paragraph{Purchasing Operational Efficiency}
\label{\detokenize{FutureState:purchasing-operational-efficiency}}

\subparagraph{Purchasing Review Board}
\label{\detokenize{FutureState:purchasing-review-board}}
Requisitions may be reviewed by the purchasing review board
\begin{itemize}
\item {} 
Approval

\item {} 
Disapproval

\item {} 
Record disapproval reason for requisitions Purchasing review board

\end{itemize}

can select (or create and select) a reason such as:
\begin{itemize}
\item {} \begin{description}
\item[{Review equivalent parts}] \leavevmode\begin{itemize}
\item {} 
Is onhand under another part

\item {} 
insufficient quotations (other vendors may have lower costs)

\end{itemize}

\end{description}

\end{itemize}


\subparagraph{Speed up quotations}
\label{\detokenize{FutureState:speed-up-quotations}}\begin{itemize}
\item {} 
Automatically email vendors request for quotations

\item {} 
Automated quote response have the vendors provide a CSV, JSON or XML
file with the quote to be automatically uploaded to the system.

\end{itemize}

For example a vendor could create a spreadsheet with the following
columns
\begin{itemize}
\item {} 
item\_cd

\item {} 
quantity

\item {} 
manufacturer

\item {} 
price

\item {} 
available date

\end{itemize}

by emailing to \sphinxhref{mailto:quotes@yourco.com}{quotes@yourco.com} these quotes can be automatically
loaded into the system without changes to the legacy system,


\subparagraph{Buyer information}
\label{\detokenize{FutureState:buyer-information}}
The buyer should have single screen that shows:
\begin{enumerate}
\sphinxsetlistlabels{\arabic}{enumi}{enumii}{}{.}%
\item {} 
Supplies

\item {} 
On hand

\item {} 
Open Purchase Orders

\item {} 
Open Work Orders

\item {} 
Demand

\item {} 
Forecasted
\begin{enumerate}
\sphinxsetlistlabels{\arabic}{enumii}{enumiii}{}{.}%
\item {} 
Raw

\item {} 
Consumed

\item {} 
Unconsumed

\end{enumerate}

\item {} 
Safety Stock

\item {} 
Reserved Inventory

\item {} 
Quarantined

\item {} 
Restricted access (JIT programs, Committed Service Level Agreement
Plans)

\item {} 
All part numbers in the planning group

\item {} 
Every part and all equivalents, transitively, that is the equivalents
to those equivalents until exhausted.

\item {} 
Customer specific substitutions

\item {} 
Approved manufacturer matrix Customers down the left, manufacturers
across the top

\item {} 
Requisitions

\item {} 
Supplier on-time historical metrics

\item {} 
Supply ineligibility drill-down

\item {} 
Vendor Quotes

\item {} 
Time phased inventory position, Pipeline (Global, by Facility, by
planner)

\item {} 
On hand inventory in aggregate with the ability to open details with
a single click

\item {} 
Sales history for the last three years in multiple dimensions

\item {} 
Time Dimensions include annual, quarterly and monthly

\item {} 
Ability to see by customer

\item {} 
Existing purchase orders

\item {} 
Existing facility transfers in process

\item {} 
Detailed reason why supplies are not eligible for a demand that is
allocated late or short

\item {} 
A matrix of approved manufactures and customers

\item {} 
See the part and all transitive equivalent parts

\item {} 
Late or short demands

\end{enumerate}

In Diamond this is all done locally in the web browser with no network
requests so it is virtually instantaneous.


\subparagraph{Recommendations}
\label{\detokenize{FutureState:recommendations}}\begin{enumerate}
\sphinxsetlistlabels{\arabic}{enumi}{enumii}{}{.}%
\item {} 
Purchase orders that can be cancelled

\item {} 
Get a manufacturer Certificate of Compliance for existing inventory
to satisfy a requisition with existing inventory

\item {} 
Supply prioritization Use buyback inventory before using our
inventory for appropriate customers

\item {} 
Allocation based pricing

\item {} 
Items with a shelf life have oldest allocated first

\item {} 
Less valuable items are allocated first

\item {} 
Based on Certifications (dual certified parts have more value)

\item {} 
Facility Transfer

\item {} 
Supply Pool Transfer

\item {} 
Expedite or de-expedite a purchase order

\end{enumerate}


\subparagraph{Alerts}
\label{\detokenize{FutureState:alerts}}\begin{itemize}
\item {} 
Obsolete Inventory

\item {} 
Expiring Inventory

\item {} 
Purchase Exceeding x\% of previous maximum unit price

\item {} 
Purchase Exceeding x\% of previous minimum unit price

\item {} 
Purchase of specified dollars not yet approved

\end{itemize}


\paragraph{Extensibility}
\label{\detokenize{FutureState:extensibility}}
Any component must be easily plugged in with an alternative
implementation that is compliant with the corresponding interface,
\begin{itemize}
\item {} 
Demand Priority

\item {} 
Eligibility Requirements

\item {} 
Supply Prioritization
\begin{itemize}
\item {} 
Lot value determination

\end{itemize}

\item {} 
Recommendation Handlers for propagating accepted recommendations to
source system

\end{itemize}


\subsubsection{Implementation}
\label{\detokenize{FutureState:implementation}}\begin{itemize}
\item {} 
Extract necessary data from legacy systems

\item {} 
Load into Advanced Planning

\item {} 
Augment with necessary but unavailable information

\item {} 
Run a full plan

\item {} 
Review recommendations
\begin{itemize}
\item {} 
Accept recommendation (must define Action Handlers) Reject
recommendation (select reason to be persisted across full reloads)

\end{itemize}

\end{itemize}


\paragraph{Questions}
\label{\detokenize{FutureState:questions}}\begin{enumerate}
\sphinxsetlistlabels{\arabic}{enumi}{enumii}{}{.}%
\item {} 
Inventory Restriction

\item {} 
How do you restrict availability of inventory for special purposes
such as
\begin{itemize}
\item {} 
JIT contracts

\item {} 
Committed Service Level Agreements

\item {} 
Kitting and Assembly

\end{itemize}

\item {} 
Do you have automated approved manufacturer eligibility?

\item {} 
Do you have prioritization for lots with expiry dates?

\item {} 
How do you calculate the residual cost of goods for broker buys for
the

\item {} 
Are you exclusively FIFO or do you consider lots that have lower cost
that satisfies the demand (taking into consideration multiple
certifications, incremental cost of Quality Assurance testing and
destructive tests?), etc.? Quantity that exceeds the customer demand?

\item {} 
Quality Assurance Do you have a quality assurance program that
supports skip lot testing and pre-approved lots ( lots that have
already passed the QA requirements for a customer should receive
higher priority for that customer and lower priority for others)

\item {} 
What supply eligibility rules do you have?

\item {} 
How do you pin an allocation to a demand ?

\item {} 
Does your system recommend when alternate availability is preferable
to a pinned allocation?

\end{enumerate}


\paragraph{Simple Example}
\label{\detokenize{FutureState:simple-example}}
During one of my calls with Peter he told me that he was reviewing
purchase orders a simple line such as “Buyers don’t buy the correct
quantities to get a good price” was extended to:


\subparagraph{Compute Optimal Purchase Quantity}
\label{\detokenize{FutureState:compute-optimal-purchase-quantity}}
Compute a projected per unit cost by solving the equation

unit\_cost = (setup\_cost / qty) + incremental cost

For two different known qty and prices (vendor quotes) using linear
algebra


\subparagraph{Graph this relationsihip}
\label{\detokenize{FutureState:graph-this-relationsihip}}
Find the “price knee” the first derivative of the function, the slope of
the tangent starts to level off (it asymptotically approaches 0, meaning
the limit is the unit cost doesn’t decrease at all. Depending on setup
cost, incremental cost and annual consumption a three year supply may be
ten percent more than a one year supply, it may also be three times the
acquisition cost and additional carrying costs must be considered.

Vendor quotes should include this range of quantities, purchasing
quantities should be in this range, buys can be made and even scheduled
so that lower per unit costs can be realized.


\paragraph{Purchasing Procedures}
\label{\detokenize{FutureState:purchasing-procedures}}
When a part needs to be replenished
\begin{enumerate}
\sphinxsetlistlabels{\arabic}{enumi}{enumii}{}{.}%
\item {} 
Vendor quotes for the price range should be required.

\item {} 
Purchase amounts over a defined limit should be reviewed and
approved.

\item {} 
Requisitions should be created in the new purchase decision
application and once approved, be created as purchase orders in the
execution system (Dymax and SAP).

\item {} 
Checks for any constraints including approved manufacturers should be
simulated

\item {} 
Existing inventory carried under equivalent part numbers should be
considered.

\end{enumerate}

The opportunities for process improvement are best addressed by
evaluating your current processes and the issues your experts realize
and developing a system to address those issues.


\paragraph{Constraints}
\label{\detokenize{FutureState:constraints}}
Your new process should:
\begin{enumerate}
\sphinxsetlistlabels{\arabic}{enumi}{enumii}{}{.}%
\item {} 
Be external to SAP and Dymax, requiring no modifications to either
system. This eliminates risk and complexity.

\item {} 
Should include data from both operations for inventory, purchases and
demands

\item {} 
Incorporate new procedures and policies to reflect best practices

\item {} 
Reduce the effort of sales staff and purchasing staff to perform
their functions

\item {} 
Define metrics to evaluate performance and progress

\item {} 
Have an alert system of reports of issues that need to be addressed.

\item {} 
Require no hardware or other infrastructure or the installation of
any software on any Align computer.

\end{enumerate}


\subsubsection{Questions}
\label{\detokenize{FutureState:id1}}
What is the current cost of
\begin{enumerate}
\sphinxsetlistlabels{\arabic}{enumi}{enumii}{}{.}%
\item {} 
Not buying the correct quantities

\item {} 
Not taking into consideration multiple certifications

\item {} 
Buying inventory in one operation that is excess inventory in the
other operation

\item {} 
Time wasted gathering information to create a purchase order

\end{enumerate}


\subsubsection{Conclusion}
\label{\detokenize{FutureState:conclusion}}
Align has the expertise in house to participate in the design of a
business process and software to optimize the purchasing and sales
operations, there is no need to wait for an IT person who has much less
experience than your director of purchasing and other operations
personnel.

A one hour phone call every two weeks is not going to ever get you a
design.

I have no doubt that several times every day a sub-optimal purchase
results in a expense greater than the cost of developing a design.

A design is best done by whiteboard meetings, starting with a blank
whiteboard. Even with 25 years experience in Aerospace, it would be
presumption, and flat out wrong for me to give a Powerpoint
presentation and say “This is the universal answer to all problems, it
will fit your situation”; this is the approach of someone hawking
software.
\begin{itemize}
\item {} 
Automatically email vendors request for quotations

\item {} 
Automated quote response have the vendors provide a CSV, JSON or XML
file with the quote to be

\item {} 
automatically uploaded to the system.

\end{itemize}

For example a vendor could create a spreadsheet with the following
columns
\begin{itemize}
\item {} 
item\_cd

\item {} 
quantity

\item {} 
manufacturer

\item {} 
price

\item {} 
available date

\end{itemize}

by emailing to \sphinxhref{mailto:quotes@yourco.com}{quotes@yourco.com} these quotes can be automatically
loaded into the system without changes to the legacy system,

The buyer should have single screen that shows:
\begin{enumerate}
\sphinxsetlistlabels{\arabic}{enumi}{enumii}{}{.}%
\item {} 
Supplies

\item {} 
On hand

\item {} 
Open Purchase Orders

\item {} 
Open Work Orders

\item {} 
Demand

\item {} 
Forecasted
\begin{enumerate}
\sphinxsetlistlabels{\arabic}{enumii}{enumiii}{}{.}%
\item {} 
Raw

\item {} 
Consumed

\item {} 
Unconsumed

\end{enumerate}

\item {} 
Safety Stock

\item {} 
Work Orders

\item {} 
Reserved Inventory

\item {} 
Quarantined

\item {} 
Restricted access (JIT programs, Committed Service Level Agreement
Plans)

\item {} 
All part numbers in the planning group

\item {} 
Every part and all equivalents, transitively, that is the equivalents
to those equivalents until exhausted.

\item {} 
Customer specific substitutions

\item {} 
Approved manufacturer matrix Customers down the left, manufacturers
across the top

\item {} 
Requisitions

\item {} 
Supplier on-time historical metrics

\item {} 
Supply ineligibility drill-down

\item {} 
Vendor Quotes

\item {} 
Time phased inventory position, Pipeline (Global, by Facility, by
planner)

\item {} 
On hand inventory in aggregate with the ability to open details with
a single click

\item {} 
Sales history for the last three years in multiple dimensions

\item {} 
Time Dimensions include annual, quarterly and monthly

\item {} 
Ability to see by customer

\item {} 
Existing purchase orders

\item {} 
Existing facility transfers in process

\item {} 
Detailed reason why supplies are not eligible for a demand that is
allocated late or short

\item {} 
A matrix of approved manufactures and customers

\item {} 
See the part and all transitive equivalent parts

\item {} 
Late or short demands

\end{enumerate}

In Diamond this is all done locally in the web browser with no network
requests so it is virtually instantaneous.


\subsubsection{Modifications}
\label{\detokenize{FutureState:modifications}}\begin{itemize}
\item {} 
All code is in Java supported by Spring with Hibernate for Object
Relation Management, these are widely adoped open source solutions

\item {} 
Presentation uses the Model, View Controller approach and the model
may be exposed as a JavaBean or XML if one prefers to use XSL.

\item {} 
Diamond dependencies are all vastly popular open source but it
extremely unlikely anyone will have any need to modify anyy of the
open source code

\item {} 
I have trained non-programmers to modify Diamond in less than a
month.

\end{itemize}


\paragraph{Questions}
\label{\detokenize{FutureState:id2}}\begin{enumerate}
\sphinxsetlistlabels{\arabic}{enumi}{enumii}{}{.}%
\item {} 
Inventory Restriction

\item {} 
How do you restrict availability of inventory for special purposes
such as
\begin{itemize}
\item {} 
JIT contracts

\item {} 
Committed Service Level Agreements

\item {} 
Kitting and Assembly

\end{itemize}

\item {} 
Do you have automated approved manufacturer eligibility?

\item {} 
Do you have prioritization for lots with expiry dates?

\item {} 
How do you calculate the residual cost of goods for broker buys for
the

\item {} 
Are you exclusively FIFO or do you consider lots that have lower cost
that satisfies the demand (taking into consideration multiple
certifications, incremental cost of Quality Assurance testing and
destructive tests?), etc.? quantity that exceeds the customer demand?

\item {} 
Quality Assurance Do you have a quality assurance program that
supports skip lot testing and pre-approved lots ( lots that have
already passed the QA requirements for a customer should receive
higher priority for that customer and lower priority for others)

\item {} 
What supply eligibility rules do you have?

\item {} 
How do you pin an allocation to a demand ?

\item {} 
Does your system recommend when alternate availability is preferable
to a pinned allocation?

\end{enumerate}



\renewcommand{\indexname}{Index}
\printindex
\end{document}