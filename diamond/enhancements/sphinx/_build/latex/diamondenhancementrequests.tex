%% Generated by Sphinx.
\def\sphinxdocclass{report}
\documentclass[letterpaper,10pt,english]{sphinxmanual}
\ifdefined\pdfpxdimen
   \let\sphinxpxdimen\pdfpxdimen\else\newdimen\sphinxpxdimen
\fi \sphinxpxdimen=.75bp\relax

\PassOptionsToPackage{warn}{textcomp}
\usepackage[utf8]{inputenc}
\ifdefined\DeclareUnicodeCharacter
% support both utf8 and utf8x syntaxes
  \ifdefined\DeclareUnicodeCharacterAsOptional
    \def\sphinxDUC#1{\DeclareUnicodeCharacter{"#1}}
  \else
    \let\sphinxDUC\DeclareUnicodeCharacter
  \fi
  \sphinxDUC{00A0}{\nobreakspace}
  \sphinxDUC{2500}{\sphinxunichar{2500}}
  \sphinxDUC{2502}{\sphinxunichar{2502}}
  \sphinxDUC{2514}{\sphinxunichar{2514}}
  \sphinxDUC{251C}{\sphinxunichar{251C}}
  \sphinxDUC{2572}{\textbackslash}
\fi
\usepackage{cmap}
\usepackage[T1]{fontenc}
\usepackage{amsmath,amssymb,amstext}
\usepackage{babel}



\usepackage{times}
\expandafter\ifx\csname T@LGR\endcsname\relax
\else
% LGR was declared as font encoding
  \substitutefont{LGR}{\rmdefault}{cmr}
  \substitutefont{LGR}{\sfdefault}{cmss}
  \substitutefont{LGR}{\ttdefault}{cmtt}
\fi
\expandafter\ifx\csname T@X2\endcsname\relax
  \expandafter\ifx\csname T@T2A\endcsname\relax
  \else
  % T2A was declared as font encoding
    \substitutefont{T2A}{\rmdefault}{cmr}
    \substitutefont{T2A}{\sfdefault}{cmss}
    \substitutefont{T2A}{\ttdefault}{cmtt}
  \fi
\else
% X2 was declared as font encoding
  \substitutefont{X2}{\rmdefault}{cmr}
  \substitutefont{X2}{\sfdefault}{cmss}
  \substitutefont{X2}{\ttdefault}{cmtt}
\fi


\usepackage[Bjarne]{fncychap}
\usepackage{sphinx}

\fvset{fontsize=\small}
\usepackage{geometry}

% Include hyperref last.
\usepackage{hyperref}
% Fix anchor placement for figures with captions.
\usepackage{hypcap}% it must be loaded after hyperref.
% Set up styles of URL: it should be placed after hyperref.
\urlstyle{same}
\addto\captionsenglish{\renewcommand{\contentsname}{Contents:}}

\usepackage{sphinxmessages}
\setcounter{tocdepth}{1}



\title{Diamond Enhancement Requests}
\date{Dec 13, 2019}
\release{19.11}
\author{Jim Schmidt}
\newcommand{\sphinxlogo}{\vbox{}}
\renewcommand{\releasename}{Release}
\makeindex
\begin{document}

\pagestyle{empty}
\sphinxmaketitle
\pagestyle{plain}
\sphinxtableofcontents
\pagestyle{normal}
\phantomsection\label{\detokenize{index::doc}}



\chapter{Item Statistics}
\label{\detokenize{IcItemStat:item-statistics}}\label{\detokenize{IcItemStat::doc}}

\section{Overview}
\label{\detokenize{IcItemStat:overview}}
During a team phone call on December 15 ABC code requirement was
identified.

This demonstrates computing a number of potentially useful statistics


\section{Approach}
\label{\detokenize{IcItemStat:approach}}\begin{itemize}
\item {} 
Create a table to hold statistics

\item {} 
Create a script to populate statistics by item

\item {} 
Create a service to obtain the data model for the web pages

\item {} 
Modify the filter screen to allow query filters on the statistics

\item {} 
Modify the web pages to show the statistics information

\end{itemize}


\section{Code}
\label{\detokenize{IcItemStat:code}}

\subsection{Create the table}
\label{\detokenize{IcItemStat:create-the-table}}
drop table ic\_item\_stat;
\begin{description}
\item[{create table ic\_item\_stat (}] \leavevmode
item\_nbr integer primary key references ic\_item\_mast,
abc\_cd  varchar(1),
distinct\_open\_ord\_cust\_count integer,
distinct\_org\_cust\_qte          integer,
distinct\_cust\_open\_order\_count integer

\end{description}

);
\begin{description}
\item[{alter table ic\_item\_stat add(}] \leavevmode
check (abc\_cd in (‘A’,’B’,’C’)

\end{description}

);


\subsection{SQlrunner file}
\label{\detokenize{IcItemStat:sqlrunner-file}}
This is a yaml file
\sphinxurl{https://en.wikipedia.org/wiki/YAML}
.. code:: sql
\begin{description}
\item[{truncate\_ic\_item\_stats:}] \leavevmode
sql: \textgreater{} truncate table ic\_item\_stat
description: delete all the rows from ic\_item\_stat

\item[{distinct\_cust\_open\_order\_count:}] \leavevmode\begin{description}
\item[{sql: \textgreater{} update ic\_item\_stat}] \leavevmode
set distinct\_cust\_open\_order\_count =
( select count(distinct org\_nbr\_cst
\begin{quote}

from oe\_ord\_hdr\_dtl\_vw
\end{quote}

)

\end{description}

description: distinct cust open order count

\item[{number\_of\_open\_orders}] \leavevmode
sql: \textgreater{}
description: number of open orders

\item[{number\_of\_distinct\_customer\_quotes\_last\_three\_months:}] \leavevmode\begin{description}
\item[{sql: \textgreater{} update  ic\_item\_stats (}] \leavevmode
select count(distinct(org\_nbr\_cst)
from sq\_qte\_vw

\end{description}

distinct\_customer\_quotes\_prev\_3\_months
description: Count distinct customer quotes last three months

\item[{max\_sales\_price\_last\_three\_months}] \leavevmode
sql: \textgreater{}
description: Max sales price last three months

\item[{quote\_conversion\_ratio:}] \leavevmode
sql: \textgreater{}
description: quote conversion ratio

\item[{dollars\_open\_orders:}] \leavevmode
sql: \textgreater{}
description: Dollar open orders

\item[{number\_of\_customers\_open\_orders:}] \leavevmode\begin{description}
\item[{sql: \textgreater{} update  ic\_item\_stats (}] \leavevmode
select count(distinct(org\_nbr\_cst)
from oe\_ord\_hdr\_dtl\_vw
where ord\_stat\_id = ‘O’

\end{description}

description: number of distinct open order customers

\item[{abc\_code:}] \leavevmode
sql\textgreater{}
description: Pareto code
narrative: \textgreater{}
\begin{quote}

A,B,C classification
compute aggregate sum of dollars
Basing on shipments does not reflect new items
Basing on open orders the dollar totals will be low for new items
\end{quote}

\item[{unallocated\_order\_qty:}] \leavevmode
description: Unalloc ord qty
sql: \textgreater{}

\item[{percent\_of\_open\_orders\_base\_curr}] \leavevmode
sql: \textgreater{}
description: Percent of open orders for base currency

\end{description}


\subsection{Create a service}
\label{\detokenize{IcItemStat:create-a-service}}
package com.pacificdataservices.diamond.apsweb;

import java.io.IOException;
import java.sql.Connection;
import java.sql.SQLException;

import javax.sql.DataSource;

import org.javautil.core.json.JsonSerializer;
import org.javautil.core.json.JsonSerializerGson;
import org.javautil.core.sql.Binds;
import org.javautil.core.sql.SqlStatement;
import org.javautil.util.NameValue;
import org.slf4j.Logger;
import org.slf4j.LoggerFactory;
import org.springframework.beans.factory.annotation.Autowired;
import org.springframework.web.bind.annotation.RequestMapping;
import org.springframework.web.bind.annotation.RequestParam;
import org.springframework.web.bind.annotation.RestController;

@RestController
public class IcItemStatController \{
\begin{quote}

private Logger logger = LoggerFactory.getLogger(getClass());
@Autowired
private DataSource datasource;

@RequestMapping(“/icItemStat”)
public String  planData(
\begin{quote}
\begin{quote}
\begin{description}
\item[{@RequestParam(value=”itemNbr”) String itemNbr)}] \leavevmode
throws SQLException, IOException \{

\end{description}
\end{quote}

logger.info(“invoked with itemNumber \{\}”,itemNbr);
Connection conn = datasource.getConnection();
SqlStatement ss = new SqlStatement(conn, “select * from ic\_item\_stat where item\_nbr = :item\_nbr”);
Binds binds = new Binds();
binds.put(“item\_nbr”, itemNbr);
NameValue nameValue = ss.getNameValue(binds,true);
JsonSerializer serializer = new JsonSerializerGson();
String json = serializer.toJsonPretty(nameValue);
return json;
\end{quote}

\}
\end{quote}

\}


\subsection{Create a node service}
\label{\detokenize{IcItemStat:create-a-node-service}}
\begin{sphinxVerbatim}[commandchars=\\\{\}]
\PYG{k+kn}{import} \PYG{p}{\PYGZob{}} \PYG{n}{HttpClient} \PYG{p}{\PYGZcb{}} \PYG{k+kn}{from} \PYG{l+s+s1}{\PYGZsq{}}\PYG{l+s+s1}{@angular/common/http}\PYG{l+s+s1}{\PYGZsq{}}\PYG{p}{;}
\PYG{k+kn}{import} \PYG{p}{\PYGZob{}} \PYG{n}{Observable} \PYG{p}{\PYGZcb{}} \PYG{k+kn}{from} \PYG{l+s+s1}{\PYGZsq{}}\PYG{l+s+s1}{rxjs}\PYG{l+s+s1}{\PYGZsq{}}\PYG{p}{;}
\PYG{o}{.}\PYG{o}{.}\PYG{o}{.}
\PYG{n}{getIcItemStat}\PYG{p}{(}\PYG{p}{)}\PYG{p}{:} \PYG{n}{Observable}\PYG{o}{\PYGZlt{}}\PYG{n}{IcItemStat}\PYG{p}{[}\PYG{p}{]}\PYG{o}{\PYGZgt{}} \PYG{p}{\PYGZob{}}
     \PYG{k}{return} \PYG{n}{this}\PYG{o}{.}\PYG{n}{http}\PYG{o}{.}\PYG{n}{get}\PYG{o}{\PYGZlt{}}\PYG{n}{IcItemStat}\PYG{p}{[}\PYG{p}{]}\PYG{o}{\PYGZgt{}}\PYG{p}{(}\PYG{n}{HOST} \PYG{o}{+} \PYG{l+s+s1}{\PYGZsq{}}\PYG{l+s+s1}{/api/v1/icitemstat/}\PYG{l+s+s1}{\PYGZsq{}} \PYG{o}{+} \PYG{n}{itemnbr}\PYG{p}{)}\PYG{p}{;}
 \PYG{p}{\PYGZcb{}}
\end{sphinxVerbatim}


\section{Modify web page template}
\label{\detokenize{IcItemStat:modify-web-page-template}}

\chapter{Optimal Replenishment Quantity}
\label{\detokenize{OptimalReplenishmentQty:optimal-replenishment-quantity}}\label{\detokenize{OptimalReplenishmentQty::doc}}

\section{Overview}
\label{\detokenize{OptimalReplenishmentQty:overview}}
Optimal Replenishment Quantity has two contradictory goals
\begin{itemize}
\item {} 
Larger replenishment quantities may reduce cost per unit if the item is manufactured on

\end{itemize}

demand due to the setup cost being apportioned over a smaller number of units.  However,
popular items may be continually produced by a manufacturer or produced in sufficient
quantities to TODO ameliorate or minimize this.

Larger replenishment quantities reduce safety stock requirements as variations in demand
may be satisfied by working inventory.

Larger replenishment quantities reduce inventory turns.  Inventory turns are defined as
average months supply on hand   / (average monthly consumption * 12).

Higher replenishment quantities may negatively effect cash flow.  If inventory can
be acquired on consignment or larger purchase quantities can be scheduled over several
months with minimal additional per unit cost this effect can be mitigated.  Vendor
negotiations are required to achieve either objective, vendor incentive may be a commitment
to buy a larger number of parts.

Not to be dismissed lightly is the strange pricing as a part is quoted as a different
part number and a vendor willingness to issue a certificate of compliance for the
equivalent part or customer willingness to accept an alternate part number.

Ignorance of equivalent part numbers is common in the industry.

Although part numbers are specified in engineering drawings vendors and customers may
give different names, including varying use of dashes.  This is addressed in Diamond and
is called \sphinxstyleemphasis{nomenclature}.  Prior deployments have had in excess of one million alternate
names for various parts.  Acquiring this data may be labor intensive with benefits for
Pareto part class (“ABC\_CODE”) of class “A”, the top 20 percent of items based on annual
sales currency.

Let us assume in the interest of simplification a scenario that ignores Approved Manufacturer
constraints, lead times and cash-flow mitigation strategies, such as


\section{Optimal Cash Flow}
\label{\detokenize{OptimalReplenishmentQty:optimal-cash-flow}}
Approaches

Consignment inventory
Scheduled Delivery


\section{Stock outs}
\label{\detokenize{OptimalReplenishmentQty:stock-outs}}

\section{Lowest cost}
\label{\detokenize{OptimalReplenishmentQty:lowest-cost}}

\section{Compute Optimal Replenishment}
\label{\detokenize{OptimalReplenishmentQty:compute-optimal-replenishment}}
UC = Unit Cost  : Setup Cost / unit qty + incremental cost

Replenishment Cost = Unit Cost * Unit Qty
AMC Average Monthly Consumption
AMOH : Average Monthly Onhand  Replen Qty - sum AMC for i (1 : n) \textless{} Replen Qt
Replen Qty
Carrying cost = AMOH * COF
COF : Cost of Funds
Receiving Cost:  Cost to place a purchase order line, receive, putaway and pay an AP line
Risk of Obsolescence :  Fewer customers increases risk amonth other factors, varies over time

Compusting optimal replenishent quantity is straight forward math at this point.

Throw in a maximum outstanding value of new inventory accross ABC Codes and there are judgment calls required.


\chapter{Delivery Location}
\label{\detokenize{DeliveryLocation:delivery-location}}\label{\detokenize{DeliveryLocation::doc}}
U.S.A. vs France


\chapter{Lead Time By Vendor}
\label{\detokenize{LeadTimeByVendor:lead-time-by-vendor}}\label{\detokenize{LeadTimeByVendor::doc}}

\section{Overview}
\label{\detokenize{LeadTimeByVendor:overview}}
Vendors may have drastically different lead times.


\section{Acquiring}
\label{\detokenize{LeadTimeByVendor:acquiring}}
Data for lead times should be derived from the latest vendor quotes that is:

Note that item\_nbr, the item surrogate key is not used, allowing you to
get vendor quotes for items that have not been set up;

create view ic\_item\_vnd\_lead\_time as
select  item\_cd\_qte,
\begin{quote}

vq\_qte\_dt,
vq\_qte\_eff\_dt,
max(vq\_qte\_exp\_dt) max\_qte\_exp\_dt,
(max(vq\_qte\_exp\_dt) - vq\_qte\_eff\_dt) / 7 lead\_tm\_wks
\end{quote}

from vq\_qte\_vw
group by org\_nbr\_vnd,
\begin{quote}

item\_cd\_qte,
vq\_qte\_eff\_dt,
vq\_qte\_dt;
\end{quote}


\section{YAML}
\label{\detokenize{LeadTimeByVendor:yaml}}\begin{description}
\item[{ic\_item\_vnd\_lead\_tm:}] \leavevmode\begin{description}
\item[{sql: \textgreater{}}] \leavevmode\begin{quote}

create view ic\_item\_vnd\_lead\_time as
select item\_cd\_qte,
\begin{quote}
\begin{quote}

vq\_qte\_dt,
vq\_qte\_eff\_dt,
max(vq\_qte\_exp\_dt) max\_qte\_exp\_dt,
(max(vq\_qte\_exp\_dt) - vq\_qte\_eff\_dt) / 7 lead\_tm\_wks
\end{quote}

from vq\_qte\_vw
group by org\_nbr\_vnd,
\begin{quote}

item\_cd\_qte,
vq\_qte\_eff\_dt,
vq\_qte\_dt;
\end{quote}
\end{quote}
\end{quote}

description: Create ic\_item\_vnd\_lead\_tm
narrative: \textgreater{}
\begin{quote}

TODO describe consequence of using the quote expiration date, vq\_qte\_exp\_dt
rather than the quote of quotation or the effectivedate of quotation.
\end{quote}

\end{description}

\end{description}


\section{Assumptions}
\label{\detokenize{LeadTimeByVendor:assumptions}}
The most recent vendor quote for lead times will be used.

If multiple vendor quotes exist for the same quote date, the maximum
lead time will be used.

Full historical vendor quotes are useful to see trends in cost and lead
time.


\section{Issues}
\label{\detokenize{LeadTimeByVendor:issues}}
Vendor on-time performance.

TODO flesh out


\chapter{Simulation}
\label{\detokenize{Simulation:simulation}}\label{\detokenize{Simulation::doc}}
Planning simulation


\chapter{Questions}
\label{\detokenize{Questions:questions}}\label{\detokenize{Questions::doc}}
Vendor Quotes

Customer Quotes

Base Currency

Supply pools implementation in US and French operations.

How many distinct part numbers?

How many Customer and vendor quotes?

Sales volumes for US and France, number of sales and dollars per operation per year.

Describe your current kitting for both operations.

Describe supply eligibility features and deficiencies for both operations.

Describe project approval functional, technical, financial.

Post implementaton functional specifications consulting is time and materials, coding desing,coding
and documetation is fixed price.


\chapter{Data Loading}
\label{\detokenize{Dataload:data-loading}}\label{\detokenize{Dataload::doc}}
Base currency

Acceptance criteria

Funding

Accept base functionality

Accept willingness to extend existing functionality

Accept ability to contribute to functional requirements for extensions

Define minimal  viable product.

Accept that there is no commercial “off the shelf” product that will satisfy
Align needs and desires.

Accept that in-house development cost and risk is far higher than licensing Diamond APS, which
has near 25 years of incorporation of distributor requirements.

Accept that there is low risk associated with application support.

Address current promise date for open orders is before the planning date, that is,
the purchase order is past due.

Acknowledge that we provide valuable assistance in defining user requirements, that
this experience vastly exceeds consulting experience from other consulting companies.

Reject other consultants that state \sphinxstyleemphasis{SMOP} Simply a matter of Coding without demonstrated
ability to actually deliver a high performance, \sphinxstyleemphasis{non-kludgy}, supportable solution.

Accept that an optimal solution involves a general issue and that we provide very valuable
service in defining detailed functional requirements and a clear technical implementation plan.


\chapter{Integration}
\label{\detokenize{Issues:integration}}\label{\detokenize{Issues::doc}}
US operations and Frenc operations should continue with no system modifications.

The ability to used Diamond facility transfers will be supported by intra-company purchases.


\chapter{Technology}
\label{\detokenize{Technology:technology}}\label{\detokenize{Technology::doc}}

\section{Java vs PHP}
\label{\detokenize{Technology:java-vs-php}}

\section{Postgres vs mysql}
\label{\detokenize{Technology:postgres-vs-mysql}}

\section{Spring MVC vs Angular}
\label{\detokenize{Technology:spring-mvc-vs-angular}}

\section{Stack}
\label{\detokenize{Technology:stack}}
java open\_jdk1.8
postgres 10 order later
maven
git
AWSEC2
Linux Redhat 8
Tomcat
Node.js and Angular 8


\chapter{Indices and tables}
\label{\detokenize{index:indices-and-tables}}\begin{itemize}
\item {} 
\DUrole{xref,std,std-ref}{genindex}

\item {} 
\DUrole{xref,std,std-ref}{modindex}

\item {} 
\DUrole{xref,std,std-ref}{search}

\end{itemize}



\renewcommand{\indexname}{Index}
\printindex
\end{document}