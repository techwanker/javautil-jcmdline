%% Generated by Sphinx.
\def\sphinxdocclass{report}
\documentclass[letterpaper,10pt,english]{sphinxmanual}
\ifdefined\pdfpxdimen
   \let\sphinxpxdimen\pdfpxdimen\else\newdimen\sphinxpxdimen
\fi \sphinxpxdimen=.75bp\relax

\PassOptionsToPackage{warn}{textcomp}
\usepackage[utf8]{inputenc}
\ifdefined\DeclareUnicodeCharacter
% support both utf8 and utf8x syntaxes
  \ifdefined\DeclareUnicodeCharacterAsOptional
    \def\sphinxDUC#1{\DeclareUnicodeCharacter{"#1}}
  \else
    \let\sphinxDUC\DeclareUnicodeCharacter
  \fi
  \sphinxDUC{00A0}{\nobreakspace}
  \sphinxDUC{2500}{\sphinxunichar{2500}}
  \sphinxDUC{2502}{\sphinxunichar{2502}}
  \sphinxDUC{2514}{\sphinxunichar{2514}}
  \sphinxDUC{251C}{\sphinxunichar{251C}}
  \sphinxDUC{2572}{\textbackslash}
\fi
\usepackage{cmap}
\usepackage[T1]{fontenc}
\usepackage{amsmath,amssymb,amstext}
\usepackage{babel}



\usepackage{times}
\expandafter\ifx\csname T@LGR\endcsname\relax
\else
% LGR was declared as font encoding
  \substitutefont{LGR}{\rmdefault}{cmr}
  \substitutefont{LGR}{\sfdefault}{cmss}
  \substitutefont{LGR}{\ttdefault}{cmtt}
\fi
\expandafter\ifx\csname T@X2\endcsname\relax
  \expandafter\ifx\csname T@T2A\endcsname\relax
  \else
  % T2A was declared as font encoding
    \substitutefont{T2A}{\rmdefault}{cmr}
    \substitutefont{T2A}{\sfdefault}{cmss}
    \substitutefont{T2A}{\ttdefault}{cmtt}
  \fi
\else
% X2 was declared as font encoding
  \substitutefont{X2}{\rmdefault}{cmr}
  \substitutefont{X2}{\sfdefault}{cmss}
  \substitutefont{X2}{\ttdefault}{cmtt}
\fi


\usepackage[Bjarne]{fncychap}
\usepackage{sphinx}

\fvset{fontsize=\small}
\usepackage{geometry}

% Include hyperref last.
\usepackage{hyperref}
% Fix anchor placement for figures with captions.
\usepackage{hypcap}% it must be loaded after hyperref.
% Set up styles of URL: it should be placed after hyperref.
\urlstyle{same}
\addto\captionsenglish{\renewcommand{\contentsname}{Contents:}}

\usepackage{sphinxmessages}
\setcounter{tocdepth}{2}



\title{APS Portal}
\date{Dec 26, 2019}
\release{2019-12-22-22-02}
\author{Jim Schmidt}
\newcommand{\sphinxlogo}{\vbox{}}
\renewcommand{\releasename}{Release}
\makeindex
\begin{document}

\pagestyle{empty}
\sphinxmaketitle
\pagestyle{plain}
\sphinxtableofcontents
\pagestyle{normal}
\phantomsection\label{\detokenize{index::doc}}



\chapter{Introduction}
\label{\detokenize{index:introduction}}
This is the portal.


\chapter{Overview}
\label{\detokenize{index:overview}}

\chapter{Objective}
\label{\detokenize{index:objective}}\begin{itemize}
\item {} 
Purpose

\item {} 
Functionality

\item {} 
Description

\end{itemize}
\begin{quote}

This document reflects the primary source of interaction with the system
\end{quote}


\section{ABC - Pareto}
\label{\detokenize{100-ABC:abc-pareto}}\label{\detokenize{100-ABC::doc}}

\subsection{Purpose}
\label{\detokenize{100-ABC:purpose}}
Concentrate inventory and dollars on
\begin{itemize}
\item {} 
high volume,

\item {} 
high profit item

\item {} 
highest service level

\item {} 
core products

\item {} 
JIT / SLA / KITS

\end{itemize}

Metric


\subsection{Assumptions}
\label{\detokenize{100-ABC:assumptions}}
Users wish to quickly summarize the characteristics of an item


\subsection{Item Statistics}
\label{\detokenize{100-ABC:item-statistics}}

\subsection{Overview}
\label{\detokenize{100-ABC:overview}}
This demonstrates computing a number of potentially useful statistics


\subsection{Definition}
\label{\detokenize{100-ABC:definition}}
This is also known as Pareto or 80/20 rule

\sphinxurl{https://en.wikipedia.org/wiki/Pareto\_principle}
\begin{itemize}
\item {} 
Top 20 percent of sales items are rated A

\item {} 
Next 60 percent rated B

\item {} 
Bottom 20 Rated C

\end{itemize}

This has the following problems, an item may be at 21\% but is indistinguishable from a 79


\subsection{Approach}
\label{\detokenize{100-ABC:approach}}\begin{itemize}
\item {} 
Create a table to hold statistics

\item {} 
Create a script to populate statistics by item

\item {} 
Create a service to obtain the data model for the web page

\item {} 
create an angular 8 controller \DUrole{xref,std,std-ref}{label-name}

\item {} 
Modify the filter screen to allow query filters on the statistics

\item {} 
Modify the web pages to show the statistics information

\end{itemize}


\subsection{Purpose}
\label{\detokenize{100-ABC:id1}}
Concentrate inventory and dollars on
\begin{itemize}
\item {} 
high volume,

\item {} 
high profit item

\item {} 
highest service level for \sphinxstylestrong{important} products, those with high margins, not just hhigh

\end{itemize}

markups , complementery sales, Service Level Agreements, Just in Time Contracts and Kits.
\begin{itemize}
\item {} 
core products

\item {} 
{\color{red}\bfseries{}**}What should the service level be?  An item with a 5\% markup that turns 12 times a year may be

\end{itemize}

more profitable than an item with a 30\% markup and 1 annual turn**
\begin{quote}

\sphinxstylestrong{Certainly a B item at 21\% contriibution to profit margin is more valuable than a B item at 79\%.}
\end{quote}

package com.pacificdataservices.diamond.apsweb;

import java.io.IOException; import java.sql.Connection; import
java.sql.SQLException;

import javax.sql.DataSource;

import org.javautil.core.json.JsonSerializer; import
org.javautil.core.json.JsonSerializerGson; import
org.javautil.core.sql.Binds; import org.javautil.core.sql.SqlStatement;
import org.javautil.util.NameValue; import org.slf4j.Logger; import
org.slf4j.LoggerFactory; import
org.springframework.beans.factory.annotation.Autowired; import
org.springframework.web.bind.annotation.RequestMapping; import
org.springframework.web.bind.annotation.RequestParam; import
org.springframework.web.bind.annotation.RestController;

@RestController public class IcItemStatController \{
\begin{quote}

private Logger logger = LoggerFactory.getLogger(getClass());
@Autowired private DataSource datasource;
\end{quote}
\begin{description}
\item[{@RequestMapping(“/icItemStat”) public String planData(}] \leavevmode
@RequestParam(value=”itemNbr”) String itemNbr) throws SQLException,
IOException \{ \sphinxhref{http://logger.info}{logger.info}(“invoked with
itemNumber \{\}”,itemNbr); Connection conn =
datasource.getConnection(); SqlStatement ss = new SqlStatement(conn,
“select * from ic\_item\_stat where item\_nbr = :item\_nbr”); Binds
binds = new Binds(); binds.put(“item\_nbr”, itemNbr); NameValue
nameValue = ss.getNameValue(binds,true); JsonSerializer serializer =
new JsonSerializerGson(); String json =
serializer.toJsonPretty(nameValue); return json; \}

\end{description}

{\color{red}\bfseries{}{}`}\sphinxurl{https://www.briantracy.com/blog/personal-success/how-to-use-the-80-20-rule-pareto-principle/} \textless{}\sphinxurl{https://www.briantracy.com/blog/personal-success/how-to-use-the-80-20-rule-pareto-principle/}\textgreater{}
\}


\subsubsection{Create a node service}
\label{\detokenize{100-ABC:create-a-node-service}}
\begin{sphinxVerbatim}[commandchars=\\\{\}]
import \PYGZob{} HttpClient \PYGZcb{} from \PYGZsq{}@angular/common/http\PYGZsq{};
import \PYGZob{} Observable \PYGZcb{} from \PYGZsq{}rxjs\PYGZsq{};
...
getIcItemStat(): Observable\PYGZlt{}IcItemStat[]\PYGZgt{} \PYGZob{}
     return this.http.get\PYGZlt{}IcItemStat[]\PYGZgt{}(HOST + \PYGZsq{}/api/v1/icitemstat/\PYGZsq{} + itemnbr);
 \PYGZcb{}
\end{sphinxVerbatim}


\subsection{Modify web page template}
\label{\detokenize{100-ABC:modify-web-page-template}}
\sphinxurl{https://en.wikipedia.org/wiki/Pareto\_principle}

\sphinxurl{https://www.briantracy.com/blog/personal-success/how-to-use-the-80-20-rule-pareto-principle/}


\subsection{Item Statistics}
\label{\detokenize{100-ABC:id6}}

\subsection{Item Statistics Fields}
\label{\detokenize{100-ABC:item-statistics-fields}}
\sphinxstylestrong{Name}

\sphinxstylestrong{Code}

\sphinxstylestrong{Description}

\sphinxstylestrong{Benefit}

\sphinxstylestrong{Disadvantage}

\sphinxstylestrong{Compare to}

While we are at it we may as well get
\begin{itemize}
\item {} 
Number of customers

\item {} 
Number of approved manufacturers

\item {} 
Annual Turns

\item {} 
Sales Conversion Percentile from quotes

\end{itemize}

ABC

ABC\_SLS

Top 20 of previous12 month contribution to sales dollars

Does not adequately support new product introduction

ABCUSTQUOTE

Top 20 percent of CUST OPEN Quotes

Similar to ABC


\subsection{Approach}
\label{\detokenize{100-ABC:id7}}\begin{itemize}
\item {} 
Create a table to hold statistics

\item {} 
Create a script to populate statistics by item

\item {} 
Create a service to obtain the data model for the web pages

\item {} 
Modify the filter screen to allow query filters on the statistics

\item {} 
Modify the web pages to show the statistics information

\end{itemize}


\section{Optimal Replenishment Quantity}
\label{\detokenize{200-OptimalReplenishmentQuantity:optimal-replenishment-quantity}}\label{\detokenize{200-OptimalReplenishmentQuantity::doc}}

\subsection{Objectives}
\label{\detokenize{200-OptimalReplenishmentQuantity:objectives}}\begin{itemize}
\item {} 
Highest Profit

\item {} 
Highest Customer Satisfaction

\item {} 
JITS / KITS / SLA

\end{itemize}


\section{Unit Cost}
\label{\detokenize{300-UnitCost:unit-cost}}\label{\detokenize{300-UnitCost::doc}}
During one of my calls with Peter he told me that he was reviewing
purchase orders a simple line such as “Buyers don’’t buy the correct
quantities to get a good price” was extended to:


\subsection{Compute Optimal Purchase Quantity}
\label{\detokenize{300-UnitCost:compute-optimal-purchase-quantity}}

\subsection{Cost Types}
\label{\detokenize{300-UnitCost:cost-types}}
Compute a projected per unit cost by solving the equation

unit\_cost = (setup\_cost / qty) + incremental cost

For two different known qty and prices (vendor quotes) using linear
algebra


\subsection{Graph this relationship}
\label{\detokenize{300-UnitCost:graph-this-relationship}}
Find the “price knee” the first derivative of the function, the slope of
the tangent starts to level off (it asymptotically approaches 0, meaning
the limit is the unit cost doesn’’t decrease at all. Depending on setup
cost, incremental cost and annual consumption a three year supply may be
ten percent more than a one year supply, it may also be three times the
acquisition cost and additional carrying costs must be considered.

Vendor quotes should include this range of quantities, purchasing
quantities should be in this range, buys can be made and even scheduled
so that lower per unit costs can be realized.

During one of my calls with Peter he told me that he was reviewing
purchase orders a simple line such as “Buyers don’’t buy the correct
quantities to get a good price” was extended to:


\subsection{Compute Optimal Purchase Quantity}
\label{\detokenize{300-UnitCost:id1}}
Compute a projected per unit cost by solving the equation

unit\_cost = (setup\_cost / qty) + incremental cost

For two different known qty and prices (vendor quotes) using linear
algebra


\subsection{Graph this relationsihip}
\label{\detokenize{300-UnitCost:graph-this-relationsihip}}
Find the “price knee” the first derivative of the function, the slope of
the tangent starts to level off (it asymptotically approaches 0, meaning
the limit is the unit cost doesn’’t decrease at all. Depending on setup
cost, incremental cost and annual consumption a three year supply may be
ten percent more than a one year supply, it may also be three times the
acquisition cost and additional carrying costs must be considered.
by  vendor
Vendor quotes should include this range of quantities, purchasing
quantities should be in this range, buys can be made and even scheduled
so that lower per unit costs can be realized.
by  vendor


\section{Multiple Lead Times}
\label{\detokenize{400-MultipleLeadTimes:multiple-lead-times}}\label{\detokenize{400-MultipleLeadTimes::doc}}

\subsection{Overview}
\label{\detokenize{400-MultipleLeadTimes:overview}}
Vendors may have drastically different lead times.


\subsection{Acquiring}
\label{\detokenize{400-MultipleLeadTimes:acquiring}}
Data for lead times should be derived from the latest vendor quotes that is the longest lead tie from
the maximum effective date for a vendor quote.

Note that item\_nbr, the item surrogate key is not used, allowing you to
get vendor quotes for items that have not been set up;

\begin{sphinxVerbatim}[commandchars=\\\{\}]
\PYG{k}{create} \PYG{k}{view} \PYG{n}{ic\PYGZus{}item\PYGZus{}vnd\PYGZus{}lead\PYGZus{}time} \PYG{k}{as}
\PYG{k}{select}  \PYG{n}{item\PYGZus{}cd\PYGZus{}qte}\PYG{p}{,}
    \PYG{n}{vq\PYGZus{}qte\PYGZus{}dt}\PYG{p}{,}
    \PYG{n}{vq\PYGZus{}qte\PYGZus{}eff\PYGZus{}dt}\PYG{p}{,}
    \PYG{k}{max}\PYG{p}{(}\PYG{n}{vq\PYGZus{}qte\PYGZus{}exp\PYGZus{}dt}\PYG{p}{)} \PYG{n}{max\PYGZus{}qte\PYGZus{}exp\PYGZus{}dt}\PYG{p}{,}
    \PYG{p}{(}\PYG{k}{max}\PYG{p}{(}\PYG{n}{vq\PYGZus{}qte\PYGZus{}exp\PYGZus{}dt}\PYG{p}{)} \PYG{o}{\PYGZhy{}} \PYG{n}{vq\PYGZus{}qte\PYGZus{}eff\PYGZus{}dt}\PYG{p}{)} \PYG{o}{/} \PYG{l+m+mi}{7} \PYG{n}{lead\PYGZus{}tm\PYGZus{}wks}
\PYG{k}{from} \PYG{n}{vq\PYGZus{}qte\PYGZus{}vw}
\PYG{k}{group} \PYG{k}{by} \PYG{n}{org\PYGZus{}nbr\PYGZus{}vnd}\PYG{p}{,}
    \PYG{n}{item\PYGZus{}cd\PYGZus{}qte}\PYG{p}{,}
    \PYG{n}{vq\PYGZus{}qte\PYGZus{}eff\PYGZus{}dt}\PYG{p}{,}
    \PYG{n}{vq\PYGZus{}qte\PYGZus{}dt}\PYG{p}{;}
\end{sphinxVerbatim}


\subsection{YAML}
\label{\detokenize{400-MultipleLeadTimes:yaml}}
\begin{sphinxVerbatim}[commandchars=\\\{\}]
\PYG{n}{ic\PYGZus{}item\PYGZus{}vnd\PYGZus{}lead\PYGZus{}tm}\PYG{p}{:}
    \PYG{n}{sql}\PYG{p}{:} \PYG{o}{\PYGZgt{}}
          \PYG{n}{create} \PYG{n}{view} \PYG{n}{ic\PYGZus{}item\PYGZus{}vnd\PYGZus{}lead\PYGZus{}time} \PYG{k}{as}
          \PYG{n}{select} \PYG{n}{item\PYGZus{}cd\PYGZus{}qte}\PYG{p}{,}
               \PYG{n}{vq\PYGZus{}qte\PYGZus{}dt}\PYG{p}{,}
               \PYG{n}{vq\PYGZus{}qte\PYGZus{}eff\PYGZus{}dt}\PYG{p}{,}
               \PYG{n+nb}{max}\PYG{p}{(}\PYG{n}{vq\PYGZus{}qte\PYGZus{}exp\PYGZus{}dt}\PYG{p}{)} \PYG{n}{max\PYGZus{}qte\PYGZus{}exp\PYGZus{}dt}\PYG{p}{,}
               \PYG{p}{(}\PYG{n+nb}{max}\PYG{p}{(}\PYG{n}{vq\PYGZus{}qte\PYGZus{}exp\PYGZus{}dt}\PYG{p}{)} \PYG{o}{\PYGZhy{}} \PYG{n}{vq\PYGZus{}qte\PYGZus{}eff\PYGZus{}dt}\PYG{p}{)} \PYG{o}{/} \PYG{l+m+mi}{7} \PYG{n}{lead\PYGZus{}tm\PYGZus{}wks}
           \PYG{k+kn}{from} \PYG{n+nn}{vq\PYGZus{}qte\PYGZus{}vw}
           \PYG{n}{group} \PYG{n}{by} \PYG{n}{org\PYGZus{}nbr\PYGZus{}vnd}\PYG{p}{,}
              \PYG{n}{item\PYGZus{}cd\PYGZus{}qte}\PYG{p}{,}
              \PYG{n}{vq\PYGZus{}qte\PYGZus{}eff\PYGZus{}dt}\PYG{p}{,}
              \PYG{n}{vq\PYGZus{}qte\PYGZus{}dt}\PYG{p}{;}
     \PYG{n}{description}\PYG{p}{:} \PYG{n}{Create} \PYG{n}{ic\PYGZus{}item\PYGZus{}vnd\PYGZus{}lead\PYGZus{}tm}
     \PYG{n}{narrative}\PYG{p}{:} \PYG{o}{\PYGZgt{}}
        \PYG{n}{TODO} \PYG{n}{describe} \PYG{n}{consequence} \PYG{n}{of} \PYG{n}{using} \PYG{n}{the} \PYG{n}{quote} \PYG{n}{expiration} \PYG{n}{date}\PYG{p}{,} \PYG{n}{vq\PYGZus{}qte\PYGZus{}exp\PYGZus{}dt}
        \PYG{n}{rather} \PYG{n}{than} \PYG{n}{the} \PYG{n}{quote} \PYG{n}{of} \PYG{n}{quotation} \PYG{o+ow}{or} \PYG{n}{the} \PYG{n}{effective} \PYG{n}{date} \PYG{n}{of} \PYG{n}{quotation}\PYG{o}{.}
\end{sphinxVerbatim}


\subsection{Assumptions}
\label{\detokenize{400-MultipleLeadTimes:assumptions}}
The most recent vendor quote for lead times will be used.

If multiple vendor quotes exist for the same quote date, the maximum
lead time will be used.

Full historical vendor quotes are useful to see trends in cost and lead
time.


\subsection{Issues}
\label{\detokenize{400-MultipleLeadTimes:issues}}
Vendor on-time performance.


\section{Lead Time By Vendor}
\label{\detokenize{450-MultipleLeadTimes2:lead-time-by-vendor}}\label{\detokenize{450-MultipleLeadTimes2::doc}}

\subsection{Overview}
\label{\detokenize{450-MultipleLeadTimes2:overview}}
Vendors may have drastically different lead times.


\subsection{Acquiring}
\label{\detokenize{450-MultipleLeadTimes2:acquiring}}
Data for lead times should be derived from the latest vendor quotes that is:

Note that item\_nbr, the item surrogate key is not used, allowing you to
get vendor quotes for items that have not been set up;

\begin{sphinxVerbatim}[commandchars=\\\{\}]
\PYG{k}{create} \PYG{k}{view} \PYG{n}{ic\PYGZus{}item\PYGZus{}vnd\PYGZus{}lead\PYGZus{}time} \PYG{k}{as}
\PYG{k}{select}  \PYG{n}{item\PYGZus{}cd\PYGZus{}qte}\PYG{p}{,}
    \PYG{n}{vq\PYGZus{}qte\PYGZus{}dt}\PYG{p}{,}
    \PYG{n}{vq\PYGZus{}qte\PYGZus{}eff\PYGZus{}dt}\PYG{p}{,}
    \PYG{k}{max}\PYG{p}{(}\PYG{n}{vq\PYGZus{}qte\PYGZus{}exp\PYGZus{}dt}\PYG{p}{)} \PYG{n}{max\PYGZus{}qte\PYGZus{}exp\PYGZus{}dt}\PYG{p}{,}
    \PYG{p}{(}\PYG{k}{max}\PYG{p}{(}\PYG{n}{vq\PYGZus{}qte\PYGZus{}exp\PYGZus{}dt}\PYG{p}{)} \PYG{o}{\PYGZhy{}} \PYG{n}{vq\PYGZus{}qte\PYGZus{}eff\PYGZus{}dt}\PYG{p}{)} \PYG{o}{/} \PYG{l+m+mi}{7} \PYG{n}{lead\PYGZus{}tm\PYGZus{}wks}
\PYG{k}{from} \PYG{n}{vq\PYGZus{}qte\PYGZus{}vw}
\PYG{k}{group} \PYG{k}{by} \PYG{n}{org\PYGZus{}nbr\PYGZus{}vnd}\PYG{p}{,}
    \PYG{n}{item\PYGZus{}cd\PYGZus{}qte}\PYG{p}{,}
    \PYG{n}{vq\PYGZus{}qte\PYGZus{}eff\PYGZus{}dt}\PYG{p}{,}
    \PYG{n}{vq\PYGZus{}qte\PYGZus{}dt}\PYG{p}{;}
\end{sphinxVerbatim}


\subsection{YAML}
\label{\detokenize{450-MultipleLeadTimes2:yaml}}\begin{description}
\item[{ic\_item\_vnd\_lead\_tm:}] \leavevmode\begin{description}
\item[{sql: \textgreater{}}] \leavevmode\begin{quote}

create view ic\_item\_vnd\_lead\_time as
select item\_cd\_qte,
\begin{quote}
\begin{quote}

vq\_qte\_dt,
vq\_qte\_eff\_dt,
max(vq\_qte\_exp\_dt) max\_qte\_exp\_dt,
(max(vq\_qte\_exp\_dt) - vq\_qte\_eff\_dt) / 7 lead\_tm\_wks
\end{quote}

from vq\_qte\_vw
group by org\_nbr\_vnd,
\begin{quote}

item\_cd\_qte,
vq\_qte\_eff\_dt,
vq\_qte\_dt;
\end{quote}
\end{quote}
\end{quote}

description: Create ic\_item\_vnd\_lead\_tm
narrative: \textgreater{}
\begin{quote}

TODO describe consequence of using the quote expiration date, vq\_qte\_exp\_dt
rather than the quote of quotation or the effectivedate of quotation.
\end{quote}

\end{description}

\end{description}


\subsection{Assumptions}
\label{\detokenize{450-MultipleLeadTimes2:assumptions}}
The most recent vendor quote for lead times will be used.

If multiple vendor quotes exist for the same quote date, the maximum
lead time will be used.

Full historical vendor quotes are useful to see trends in cost and lead
time.


\subsection{Issues}
\label{\detokenize{450-MultipleLeadTimes2:issues}}
Vendor on-time performance.

TODO flesh out


\section{Requisition}
\label{\detokenize{750-Requisitions:requisition}}\label{\detokenize{750-Requisitions::doc}}
{]}How and why to create a requisition


\subsection{Instructions}
\label{\detokenize{750-Requisitions:instructions}}\begin{enumerate}
\sphinxsetlistlabels{\arabic}{enumi}{enumii}{}{.}%
\item {} 
Look at the report section of the portal

\item {} 
Review the information

\item {} 
Fill out the checkist

\item {} 
Simulate

\item {} 
Create the requisititioni

\item {} 
Proper quantity for ABC?

\end{enumerate}


\subsection{Gather Information}
\label{\detokenize{750-Requisitions:gather-information}}\begin{itemize}
\item {} 
Lead Times

\item {} 
Unit Costs

\item {} 
Optimal Replenishment Quantities

\item {} 
Approved Manufacturers

\end{itemize}


\subsubsection{Requisition Checklist}
\label{\detokenize{750-Requisitions:requisition-checklist}}

\subsubsection{Intrinsic Information}
\label{\detokenize{750-Requisitions:intrinsic-information}}

\subsubsection{Extrinsic Information}
\label{\detokenize{750-Requisitions:extrinsic-information}}

\subsubsection{Simulate}
\label{\detokenize{750-Requisitions:simulate}}
DOCTHIS


\subsubsection{Sequence Diagram}
\label{\detokenize{750-Requisitions:sequence-diagram}}
\noindent\sphinxincludegraphics{{Requisitions}.png}
\begin{enumerate}
\sphinxsetlistlabels{\arabic}{enumi}{enumii}{}{.}%
\item {} 
What are the requisition review requirements?
\begin{itemize}
\item {} 
When is a requisition subject to review?

\item {} 
What are the needs additional work conditions?
\begin{itemize}
\item {} 
Need more vendor quotes

\item {} 
Need work on equivalent parts

\item {} 
Incorrect buy quantity

\item {} 
Get existing inventory certified.

\end{itemize}

\end{itemize}

\end{enumerate}


\subsection{Out of Scope}
\label{\detokenize{750-Requisitions:out-of-scope}}
Lead times can vary drastically based on
\begin{itemize}
\item {} 
vendor

\item {} 
material shortages

\item {} 
new product introduction

\item {} 
product recall MRO (Maintenance, Repair and Overhaul)Definition

\end{itemize}

The elapsed time, usually measured in weeks between when a product
is ordered and receivied.

Usually not considered
\begin{itemize}
\item {} 
Recieving time

\item {} 
Inspection Time

\item {} 
Intra-facility transfer time

\item {} 
These are consider \sphinxstyleemphasis{availability times}

\end{itemize}

Typically in DRP Planning
\begin{itemize}
\item {} 
There is only one lead time

\item {} 
This lead time is fairly consistent

\item {} 
Costs due not vary based on lead time.

\end{itemize}


\subsection{In aerospace}
\label{\detokenize{750-Requisitions:in-aerospace}}

\subsection{Source of Lead Time}
\label{\detokenize{750-Requisitions:source-of-lead-time}}

\subsubsection{vendor quotes}
\label{\detokenize{750-Requisitions:vendor-quotes}}
Take the lead time from the maximum quote expiration date


\subsubsection{Summarized Lead Times should include}
\label{\detokenize{750-Requisitions:summarized-lead-times-should-include}}\begin{itemize}
\item {} 
Vendor Code

\item {} 
Vendor type (Manufacturer, Distibutor, intra-company

\item {} 
Vendor quote beginning and ending effective date

\item {} 
Date of request for quote

\end{itemize}


\subsubsection{Lead time details should include:}
\label{\detokenize{750-Requisitions:lead-time-details-should-include}}\begin{itemize}
\item {} 
Historical lead times

\end{itemize}


\subsubsection{Lead time projections}
\label{\detokenize{750-Requisitions:lead-time-projections}}
Factors that can effect lead time, Lead times can vary drastically based
on
\begin{itemize}
\item {} 
vendor

\item {} 
Some vendors will stock

\item {} 
Some manufacturers will build to stock

\item {} 
Some manufactureres will build only on demand, see Cost per Unit

\item {} 
material shortages

\item {} 
new product introduction

\item {} 
product recall MRO (Maintenance, Repair and Overhaul)

\end{itemize}

\begin{DUlineblock}{0em}
\item[] 
\end{DUlineblock}


\subsection{Instructions}
\label{\detokenize{750-Requisitions:id1}}\begin{enumerate}
\sphinxsetlistlabels{\arabic}{enumi}{enumii}{}{.}%
\item {} 
Look at the report section of the portal

\item {} 
Review the information

\item {} 
Fill out the checklist

\item {} 
Create the requisition

\end{enumerate}


\subsubsection{Requisition Checklist}
\label{\detokenize{750-Requisitions:id2}}

\subsection{Sequence Diagram}
\label{\detokenize{750-Requisitions:id3}}
\noindent\sphinxincludegraphics{{Requisitions}.png}


\subsubsection{Intrinsic Information}
\label{\detokenize{750-Requisitions:id4}}

\subsubsection{Extrinsic Information}
\label{\detokenize{750-Requisitions:id5}}
Gather all information, it should all be available in the portal


\subsection{Related articles}
\label{\detokenize{750-Requisitions:related-articles}}


\renewcommand{\indexname}{Index}
\printindex
\end{document}