%% Generated by Sphinx.
\def\sphinxdocclass{report}
\documentclass[letterpaper,10pt,english]{sphinxmanual}
\ifdefined\pdfpxdimen
   \let\sphinxpxdimen\pdfpxdimen\else\newdimen\sphinxpxdimen
\fi \sphinxpxdimen=.75bp\relax

\PassOptionsToPackage{warn}{textcomp}
\usepackage[utf8]{inputenc}
\ifdefined\DeclareUnicodeCharacter
% support both utf8 and utf8x syntaxes
  \ifdefined\DeclareUnicodeCharacterAsOptional
    \def\sphinxDUC#1{\DeclareUnicodeCharacter{"#1}}
  \else
    \let\sphinxDUC\DeclareUnicodeCharacter
  \fi
  \sphinxDUC{00A0}{\nobreakspace}
  \sphinxDUC{2500}{\sphinxunichar{2500}}
  \sphinxDUC{2502}{\sphinxunichar{2502}}
  \sphinxDUC{2514}{\sphinxunichar{2514}}
  \sphinxDUC{251C}{\sphinxunichar{251C}}
  \sphinxDUC{2572}{\textbackslash}
\fi
\usepackage{cmap}
\usepackage[T1]{fontenc}
\usepackage{amsmath,amssymb,amstext}
\usepackage{babel}



\usepackage{times}
\expandafter\ifx\csname T@LGR\endcsname\relax
\else
% LGR was declared as font encoding
  \substitutefont{LGR}{\rmdefault}{cmr}
  \substitutefont{LGR}{\sfdefault}{cmss}
  \substitutefont{LGR}{\ttdefault}{cmtt}
\fi
\expandafter\ifx\csname T@X2\endcsname\relax
  \expandafter\ifx\csname T@T2A\endcsname\relax
  \else
  % T2A was declared as font encoding
    \substitutefont{T2A}{\rmdefault}{cmr}
    \substitutefont{T2A}{\sfdefault}{cmss}
    \substitutefont{T2A}{\ttdefault}{cmtt}
  \fi
\else
% X2 was declared as font encoding
  \substitutefont{X2}{\rmdefault}{cmr}
  \substitutefont{X2}{\sfdefault}{cmss}
  \substitutefont{X2}{\ttdefault}{cmtt}
\fi


\usepackage[Bjarne]{fncychap}
\usepackage{sphinx}

\fvset{fontsize=\small}
\usepackage{geometry}

% Include hyperref last.
\usepackage{hyperref}
% Fix anchor placement for figures with captions.
\usepackage{hypcap}% it must be loaded after hyperref.
% Set up styles of URL: it should be placed after hyperref.
\urlstyle{same}
\addto\captionsenglish{\renewcommand{\contentsname}{Contents:}}

\usepackage{sphinxmessages}
\setcounter{tocdepth}{2}



\title{Diamond APS}
\date{Jan 05, 2020}
\release{2019-26 13:02}
\author{Jim Schmidt}
\newcommand{\sphinxlogo}{\vbox{}}
\renewcommand{\releasename}{Release}
\makeindex
\begin{document}

\pagestyle{empty}
\sphinxmaketitle
\pagestyle{plain}
\sphinxtableofcontents
\pagestyle{normal}
\phantomsection\label{\detokenize{index::doc}}

\bigskip\hrule\bigskip





\chapter{Objectives}
\label{\detokenize{BusinessProcessReengineering:objectives}}\begin{enumerate}
\sphinxsetlistlabels{\arabic}{enumi}{enumii}{}{.}%
\item {} 
Maximum buyer productivity

\item {} 
Eliminate unnecessary purchases

\item {} 
Develop a standardized methodology for buyers that is deterministic,
with the same input two buyers should come to the same conclusions.

\item {} 
Buyers should be able to test the results of a simulation by, for
example adding a secondary manufacturer CofC to a lot and replan the
part and get the answer back within a second with a new single
screen.

\item {} 
All scenarios should be stored with the results.

\item {} 
Upon acceptance of a scenario the simulation changes should be
reported so that the source systems can be update.

\item {} 
Status of simulation changes \# Requested \# Not possible - stops
further recommendations to take this action \# Active - Once a
download from the source system reflects this change

\item {} 
A report of requested modification not yet completed on source
systems

\end{enumerate}


\chapter{Purchasing Operational Efficiency}
\label{\detokenize{BusinessProcessReengineering:purchasing-operational-efficiency}}

\section{Purchasing Review Board}
\label{\detokenize{BusinessProcessReengineering:purchasing-review-board}}
\begin{DUlineblock}{0em}
\item[] Requisitions may be reviewd by the purchasing review board * Approval
\end{DUlineblock}

* Disapproval
\textbar{}  * Record disapproval reason for requisitions Purchasing review board
can select (or create and select) a reason such as
\textbar{}  * Review equivalent parts * Is onhand under another part *
insufficient quotations (other vendors may have lower costs)


\section{Speed up quotations}
\label{\detokenize{BusinessProcessReengineering:speed-up-quotations}}\begin{itemize}
\item {} 
Automatically email vendors request for quotations

\item {} 
Automated quote response have the vendors provide a CSV, JSON or XML
file with the quote to be

\item {} 
automatically uploaded to the system.

\end{itemize}

For example a vendor could create a spreadsheet with the following
columns
\begin{itemize}
\item {} 
item\_cd

\item {} 
quantity

\item {} 
manufacturer

\item {} 
price

\item {} 
available date

\end{itemize}

by emailing to \sphinxhref{mailto:quotes@yourco.com}{quotes@yourco.com} these quotes can be automatically
loaded into the system without changes to the legacy system,


\section{Buyer information}
\label{\detokenize{BusinessProcessReengineering:buyer-information}}
The buyer should have single screen that shows:
\begin{enumerate}
\sphinxsetlistlabels{\arabic}{enumi}{enumii}{}{.}%
\item {} 
Supplies

\item {} 
On hand

\item {} 
Open Purchase Orders

\item {} 
Open Work Orders

\item {} 
Demand

\item {} 
Forecasted
\begin{enumerate}
\sphinxsetlistlabels{\arabic}{enumii}{enumiii}{}{.}%
\item {} 
Raw

\item {} 
Consumed

\item {} 
Unconsumed

\end{enumerate}

\item {} 
Safety Stock

\item {} 
Reserved Inventory

\item {} 
Quarantined

\item {} 
Restricted access (JIT programs, Committed Service Level Agreement
Plans)

\item {} 
All part numbers in the planning group

\item {} 
Every part and all equivalents, transitively, that is the equivalents
to those equivalents until exhausted.

\item {} 
Customer specific substitutions

\item {} 
Approved manufacturer matrix Customers down the left, manufacturers
across the top

\item {} 
Requisitions

\item {} 
Supplier on-time historical metrics

\item {} 
Supply ineligibility drill-down

\item {} 
Vendor Quotes

\item {} 
Time phased inventory position, Pipeline (Global, by Facility, by
planner)

\item {} 
On hand inventory in aggregate with the ability to open details with
a single click

\item {} 
Sales history for the last three years in multiple dimensions

\item {} 
Time Dimensions include annual, quarterly and monthly

\item {} 
Ability to see by customer

\item {} 
Existing purchase orders

\item {} 
Existing facility transfers in process

\item {} 
Detailed reason why supplies are not eligible for a demand that is
allocated late or short

\item {} 
A matrix of approved manufactures and customers

\item {} 
See the part and all transitive equivalent parts

\item {} 
Late or short demands

\end{enumerate}

In Diamond this is all done locally in the web browser with no network
requests so it is virtually instaneous.


\section{Recommendations}
\label{\detokenize{BusinessProcessReengineering:recommendations}}\begin{enumerate}
\sphinxsetlistlabels{\arabic}{enumi}{enumii}{}{.}%
\item {} 
Purchase orders that can be cancelled

\item {} 
Get a manufacturer Certificate of Compliance for existing inventory
to satisfy a requisition with existing inventory

\item {} 
Supply prioritization Use buyback inventory before using our
inventory for appropriate customers

\item {} 
Allocation based pricing

\item {} 
Items with a shelf life have oldest allocated first

\item {} 
Less valuable items are allocated first

\item {} 
Based on Certifications (dual certed parts have more value)

\item {} 
Facility Transfer

\item {} 
Supply Pool Transfer

\item {} 
Expedite or de-expedite a purchase order

\end{enumerate}


\section{Alerts}
\label{\detokenize{BusinessProcessReengineering:alerts}}\begin{itemize}
\item {} 
Obsolete Inventory

\item {} 
Expiring Inventory

\item {} 
Purchase Exceeding x\% of previous maximum unit price

\item {} 
Purchase Exceeding x\% of previous minimum unit price

\item {} 
Purchase of specified dollars not yet approved

\end{itemize}


\chapter{Computing based on Setup}
\label{\detokenize{BusinessProcessReengineering:computing-based-on-setup}}
“Aerospace distribution can be very profitable if the buying is
correct”.
\begin{itemize}
\item {} 
Buy the right parts

\item {} 
Buy the optimal quantity

\end{itemize}


\chapter{Introduction}
\label{\detokenize{BusinessProcessReengineering:introduction}}
Define a process to
\begin{enumerate}
\sphinxsetlistlabels{\arabic}{enumi}{enumii}{}{.}%
\item {} 
Improve purchasing efficiency

\item {} 
Provide decision support for better purchasing decisions

\item {} 
Monitor performance

\item {} 
Create and enforce policies

\item {} 
Improve detailed and summary information

\item {} 
Leverage inter company inventory

\end{enumerate}


\chapter{Objectives}
\label{\detokenize{BusinessProcessReengineering:id1}}
Lower unit cost

We will introduce:
\begin{itemize}
\item {} 
Multiple Certifications

\item {} 
Supply Prioritization

\item {} 
Eligible Supply

\end{itemize}


\chapter{Traditional DRP}
\label{\detokenize{BusinessProcessReengineering:traditional-drp}}

\section{Forecast}
\label{\detokenize{BusinessProcessReengineering:forecast}}\begin{itemize}
\item {} 
History is aggregated by month

\item {} 
Various forecast models are applied to the history simulating
forecasts into historical periods

\item {} 
Forecast performance is evaluated over lead time

\item {} 
Best performing model is chosen

\item {} 
Forecast is made

\item {} 
Safety stock is calculated based on service level and forecast
statistics

\item {} 
Economic Order Quantity is computed

\end{itemize}


\section{Projections}
\label{\detokenize{BusinessProcessReengineering:projections}}
\sphinxstyleemphasis{Buckets} are created, usually monthly

Starting with onhand inventory

Forecast demand is decrease by firm demand during the corresponinding
bucket

For each sucessive bucket the preceeeding position is incremented by
replensishments and decremented by unconsumed forecast and actual
demand.

Additional Replenishment Orders are created based on the EOQ and
incremental purchase quantities.


\section{SKUs}
\label{\detokenize{BusinessProcessReengineering:skus}}
In traditional DRP a Stock Keeping Unit is the basis of planning.

Say there are four different \sphinxstyleemphasis{D Cell batteries}, 888888-333333 F-R-Eddy
Alkaline and 666666-22222 Deer-a-Bull and 33333-55555 BRAND-X
77777-22222 BRAND-Z.

These batteries would be planned independently.

Now consider an engineer for a major flashlight compant specifies that
specifies a battery based on dimensions, chemical composition and
electrical properties, voltage and ampere hours and a voltage discharge
curve and calls this \sphinxstyleemphasis{BATT-D-ALK}.

F-R-Eddy, Deer-a-Bull and Brand-X all issue a \sphinxstyleemphasis{Certificate of
Compliance} stating that their batteries meet all of the engineering
requirement of \sphinxstyleemphasis{BATT-D-ALK}

Let’’s assume

1,000 888888-333333

2,000 666666-222222

500 333333-555555

300 777777-222222

How many \sphinxstyleemphasis{BATT-D-ALKS} do you have?

Answer: You have 3,500


\chapter{Scenario}
\label{\detokenize{BusinessProcessReengineering:scenario}}
Now assume you get a customer order for 1,000 \sphinxstyleemphasis{BATT-D-ALK} for next
month.

What is your projected position for each of the parts at the end of next
month?

That depends on whether the 888888-333333, the 66666-222222 or the
333333-555555 parts are used.

Perhaps:


\section{Supply Prioritization}
\label{\detokenize{BusinessProcessReengineering:supply-prioritization}}
This section introduces the concept of supply prioritization,
determinining which eligible supply to to use to satisfy demand.
\begin{itemize}
\item {} 
88888-33333 Parts cost more so you fill with the cheaper part

\item {} 
You use some of each because they have a shelf life and you want to
get rid of those near the expiry date.

\item {} 
Another customer will only accept from Deer-A-Bull so you fulfill
with F-R-Ready to reserve your Deer-A-Bull inventory as it is more
widely accepted.

\end{itemize}

Perhaps: The customer only approves


\chapter{Buy Quantity}
\label{\detokenize{BusinessProcessReengineering:buy-quantity}}
Simple part, no equivalent parts.

Average monthly consumption is 200 units

Vendor quote unit price 1,000 3.02 2,500 1.647

So you buy a years supply and move onto the next part.

STOP!

Let’’s assume unit\_cost = setup\_cost/nbr\_units + incremental cost

substituting in \sphinxstyleemphasis{unit\_cost} and \sphinxstyleemphasis{nbr\_units} in the equation twice
leaves us two equations with two unknowns. Using linear algebra we solve
for the setup cost and incremental\_cost and plot this.

We see …

Now we can quote lower prices and hopefully get a higher quote
conversion while simultaneously getting a higher profit margin.

???


\section{Prioritization}
\label{\detokenize{BusinessProcessReengineering:prioritization}}

\subsection{Sourcing Rule}
\label{\detokenize{BusinessProcessReengineering:sourcing-rule}}

\subsection{Purchase Orders}
\label{\detokenize{BusinessProcessReengineering:purchase-orders}}

\subsubsection{Availability Date}
\label{\detokenize{BusinessProcessReengineering:availability-date}}

\subsubsection{Late}
\label{\detokenize{BusinessProcessReengineering:late}}

\subsubsection{Vendor On Time Performance}
\label{\detokenize{BusinessProcessReengineering:vendor-on-time-performance}}

\subsection{Attribute Weight values}
\label{\detokenize{BusinessProcessReengineering:attribute-weight-values}}

\subsection{Other Demand for}
\label{\detokenize{BusinessProcessReengineering:other-demand-for}}

\subsection{FIFO}
\label{\detokenize{BusinessProcessReengineering:fifo}}

\chapter{Data Requirements}
\label{\detokenize{BusinessProcessReengineering:data-requirements}}
Onhand Inventory Purchase Orders Work Orders

Forecast Customer Orders Work Orders

Item Master

History Vendor Quotes Customer Quotes Approved Manufacturers


\chapter{Legacy System}
\label{\detokenize{BusinessProcessReengineering:legacy-system}}
No changes to the legacy system will be required

Diamond APS can support

Vendor Quotes Customer Quotes Requisitions


\section{Legacy System}
\label{\detokenize{BusinessProcessReengineering:id2}}
Purchasing interaction with Dymax and SAP can be reduced to a data entry
function, all decisions can be made in Diamond.


\section{Risk}
\label{\detokenize{BusinessProcessReengineering:risk}}
There is zero risk, no existing systems are modified. No execution
processes are affected.


\chapter{Concepts}
\label{\detokenize{BusinessProcessReengineering:concepts}}

\section{Supply Pools}
\label{\detokenize{BusinessProcessReengineering:supply-pools}}
Supply pools are logical collections of inventory and may be used to
\begin{enumerate}
\sphinxsetlistlabels{\arabic}{enumi}{enumii}{}{.}%
\item {} 
Ensure that the pool is only available to the appropriate customers

\item {} 
Distinguish between regular, buyback and consigment inventory

\item {} 
Used in prioritization of supply for demand

\end{enumerate}


\section{Forecast Groups}
\label{\detokenize{BusinessProcessReengineering:forecast-groups}}
Demand for a customer or a collection of customers may be aggregated in
history and an aggregate forecast created. Forecast groups may have
their own eligibility constraints.


\section{Eligible Inventory}
\label{\detokenize{BusinessProcessReengineering:eligible-inventory}}
Aerospace parts are not fungible, more than a specified part number is
necessary to satify customer demand.


\section{Demand Prioritization}
\label{\detokenize{BusinessProcessReengineering:demand-prioritization}}
Demand prioritization is the set of rules that determines which demands
get fulfilled and in what order.


\section{Supply Prioritization}
\label{\detokenize{BusinessProcessReengineering:id3}}
Supply prioritization is the set of rules that determines which
inventory is consumed for a given demand.


\section{Forecast Consumption}
\label{\detokenize{BusinessProcessReengineering:forecast-consumption}}

\chapter{Procedures}
\label{\detokenize{BusinessProcessReengineering:procedures}}

\section{Requisitions}
\label{\detokenize{BusinessProcessReengineering:requisitions}}

\section{Review}
\label{\detokenize{BusinessProcessReengineering:review}}

\section{Approval}
\label{\detokenize{BusinessProcessReengineering:approval}}

\section{Purchasing Approval}
\label{\detokenize{BusinessProcessReengineering:purchasing-approval}}

\section{Portal}
\label{\detokenize{BusinessProcessReengineering:portal}}

\chapter{Benefits}
\label{\detokenize{BusinessProcessReengineering:benefits}}

\chapter{Questions}
\label{\detokenize{BusinessProcessReengineering:questions}}\begin{enumerate}
\sphinxsetlistlabels{\arabic}{enumi}{enumii}{}{.}%
\item {} 
How long does it take to get

\end{enumerate}

Sales History Vendor Quotes Customer Quotes Approved Manufacturers Item
Equivalencies Forecasted Demand Purchase Orders Requisitions

For a part and all of its equivalents in both the United States and
France?
\begin{enumerate}
\sphinxsetlistlabels{\arabic}{enumi}{enumii}{}{.}%
\item {} 
How do you compute a buy quantity?

\end{enumerate}


\section{Extensibility}
\label{\detokenize{BusinessProcessReengineering:extensibility}}
Any component must be easily plugged in with an alternative
implementation that is compliant with the corresponding interface,
\begin{itemize}
\item {} 
Demand Priority

\item {} 
Eligibility Requirements

\item {} 
Supply Prioritization
\begin{itemize}
\item {} 
Lot value determination

\end{itemize}

\item {} 
Recommendation Handlers for propagating accepted recommendations to
source system

\end{itemize}


\chapter{Implementation}
\label{\detokenize{BusinessProcessReengineering:implementation}}\begin{itemize}
\item {} 
Extract necessary data from legacy systems

\item {} 
Load into Advanced Planning

\item {} 
Augment with necessary but unavailable information

\item {} 
Run a full plan

\item {} 
Review recommendations
\begin{itemize}
\item {} 
Accept recommendation (must define Action Handlers) Reject
recommendation (select reason to be persisted across full reloads)

\end{itemize}

\end{itemize}


\section{Questions}
\label{\detokenize{BusinessProcessReengineering:id4}}\begin{enumerate}
\sphinxsetlistlabels{\arabic}{enumi}{enumii}{}{.}%
\item {} 
Inventory Restriction

\item {} 
How do you restrict availability of inventory for special purposes
such as
\begin{itemize}
\item {} 
JIT contracts

\item {} 
Committed Service Level Agreements

\item {} 
Kitting and Assembly

\end{itemize}

\item {} 
Do you have automated approved manufacturer eligibility?

\item {} 
Do you have prioritization for lots with expiry dates?

\item {} 
How do you calculate the residual cost of goods for broker buys for
the

\item {} 
Are you exclusively FIFO or do you consider lots that have lower cost
that satisfies the demand (taking into consideration multiple
certifications, incremental cost of Quality Assurance testing and
destructive tests?), etc.? quantity that exceeds the customer demand?

\item {} 
Quality Assurance Do you have a quality assurance program that
supports skip lot testing and pre-approved lots ( lots that have
already passed the QA requirements for a customer should receive
higher priority for that customer and lower priority for others)

\item {} 
What supply eligibility rules do you have?

\item {} 
How do you pin an allocation to a demand ?

\item {} 
Does your system recommend when alternate availability is preferable
to a pinned allocation?

\end{enumerate}


\chapter{SAP on the web}
\label{\detokenize{BusinessProcessReengineering:sap-on-the-web}}\begin{itemize}
\item {} 
\sphinxurl{https://www.brightworkresearch.com/sap/2017/11/best-understand-saps-negative-innovation/}

\item {} 
\sphinxurl{https://boards.straightdope.com/sdmb/archive/index.php/t-509111.html}

\item {} 
\sphinxurl{https://www.linkedin.com/pulse/who-knew-sap-could-so-complicated-heather-peyton/}

\item {} 
“We are nowhere near best-in-class, but we are making progress,” says
Steve Rogers, UK managing director of SAP to an audience of his
customers at the German applications firm’s annual user gathering at
the end of last year. It’s not the kind of comment that you expect
from a senior executive at a leading software firm.

\item {} 
\sphinxurl{https://www.thirdstage-consulting.com/lessons-from-an-sap-failure-at-lidl/}

\item {} 
\sphinxurl{https://www.360cloudsolutions.com/top-six-erp-implementation-failures/}

\item {} 
\sphinxurl{https://www.brightworkresearch.com/saphana/2017/06/22/hana-big-data-equals-big-failure/}

\end{itemize}

There is a huge amount that can be done but a specification has not been
articulated.

Concepts must be articulated.


\section{Simple Example}
\label{\detokenize{BusinessProcessReengineering:simple-example}}
During one of my calls with Peter he told me that he was reviewing
purchase orders a simple line such as “Buyers don’’t buy the correct
quantities to get a good price” was extended to:


\subsection{Compute Optimal Purchase Quantity}
\label{\detokenize{BusinessProcessReengineering:compute-optimal-purchase-quantity}}
Compute a projected per unit cost by solving the equation

unit\_cost = (setup\_cost / qty) + incremental cost

For two different known qty and prices (vendor quotes) using linear
algebra


\subsection{Graph this relationsihip}
\label{\detokenize{BusinessProcessReengineering:graph-this-relationsihip}}
Find the “price knee” the first derivative of the function, the slope of
the tangent starts to level off (it asymptotically approaches 0, meaning
the limit is the unit cost doesn’’t decrease at all. Depending on setup
cost, incremental cost and annual consumption a three year supply may be
ten percent more than a one year supply, it may also be three times the
acquisition cost and additional carrying costs must be considered.

Vendor quotes should include this range of quantities, purchasing
quantities should be in this range, buys can be made and even scheduled
so that lower per unit costs can be realized.


\section{Purchasing Procedures}
\label{\detokenize{BusinessProcessReengineering:purchasing-procedures}}
When a part needs to be replenished
\begin{enumerate}
\sphinxsetlistlabels{\arabic}{enumi}{enumii}{}{.}%
\item {} 
Vendor quotes for the price range should be required.

\item {} 
Purchase amounts over a defined limit should be reviewed and
approved.

\item {} 
Requisitions should be created in the new purchase decision
application and once approved, be created as purchase orders in the
execution system (Dymax and SAP).

\item {} 
Checks for any constraints including approved manufacturers should be
simulated

\item {} 
Existing inventory carried under equivalent part numbers should be
considered.

\end{enumerate}

The opportunities for process improvement are best addressed by
evaluating your current processes and the issues your experts realize
and developing a system to address those issues.


\section{Constraints}
\label{\detokenize{BusinessProcessReengineering:constraints}}
Your new process should:
\begin{enumerate}
\sphinxsetlistlabels{\arabic}{enumi}{enumii}{}{.}%
\item {} 
Be external to SAP and Dymax, requiring no modifications to either
system. This eliminates risk and complexity.

\item {} 
Should include data from both operations for inventory, purchases and
demands

\item {} 
Incorporate new procedures and policies to reflect best practices

\item {} 
Reduce the effort of sales staff and purchasing staff to perform
their functions

\item {} 
Define metrics to evaluate performance and progress

\item {} 
Have an alert system of reports of issues that need to be addressed.

\item {} 
Require no hardware or other infrastructure or the installation of
any software on any Align computer.

\end{enumerate}


\chapter{Questions}
\label{\detokenize{BusinessProcessReengineering:id5}}
What is the current cost of
\begin{enumerate}
\sphinxsetlistlabels{\arabic}{enumi}{enumii}{}{.}%
\item {} 
Not buying the correct quantities

\item {} 
Not taking into consideration multiple certifications

\item {} 
Buying inventory in one operation that is excess inventory in the
other operation

\item {} 
Time wasted gathering information to create a purchase order

\end{enumerate}


\chapter{Conclusion}
\label{\detokenize{BusinessProcessReengineering:conclusion}}
Align has the expertise in house to participate in the design of a
business process and software to optimize the purchasing and sales
operations, there is no need to wait for an IT person who has much less
experience than your director of purchasing and other operations
personnel.

A one hour phone call every two weeks is not going to ever get you a
design.

I have no doubt that several times every day a sub-optimal purchase
results in a expense greater than the cost of developing a design.

A design is best done by whiteboard meetings, starting with a blank
whiteboard. Even with 25 years experience in Aerospace, it would be
presumptious, and flat out wrong for me to give a powerpoint
presentation and say “This is the universal answer to all problems, it
will fit your situation”; this is the approach of someone hawking
software.


\chapter{Planning Requirements}
\label{\detokenize{BusinessProcessReengineering:planning-requirements}}

\section{Objectives}
\label{\detokenize{BusinessProcessReengineering:id6}}\begin{enumerate}
\sphinxsetlistlabels{\arabic}{enumi}{enumii}{}{.}%
\item {} 
Maximum buyer productivity

\item {} 
Eliminate unnecessary purchases

\item {} 
Develop a standardized methodology for buyers that is deterministic,
with the same input two buyers should come to the same conclusions.

\item {} 
Buyers should be able to test the results of a simulation by, for
example adding a secondary manufacturer CofC to a lot and replan the
part and get the answer back within a second with a new single
screen.

\item {} 
All scenarios should be stored with the results.

\item {} 
Upon acceptance of a scenario the simulation changes should be
reported so that the source systems can be update.

\item {} 
Status of simulation changes \# Requested \# Not possible - stops
further recommendations to take this action \# Active - Once a
download from the source system reflects this change

\item {} 
A report of requested modification not yet completed on source
systems

\end{enumerate}


\section{Purchasing Operational Efficiency}
\label{\detokenize{BusinessProcessReengineering:id7}}

\subsection{Purchasing Review Board}
\label{\detokenize{BusinessProcessReengineering:id8}}
\begin{DUlineblock}{0em}
\item[] Requisitions may be reviewd by the purchasing review board * Approval
\end{DUlineblock}

* Disapproval
\textbar{}  * Record disapproval reason for requisitions Purchasing review board
can select (or create and select) a reason such as
\textbar{}  * Review equivalent parts * Is onhand under another part *
insufficient quotations (other vendors may have lower costs)


\subsection{Speed up quotations}
\label{\detokenize{BusinessProcessReengineering:id9}}\begin{itemize}
\item {} 
Automatically email vendors request for quotations

\item {} 
Automated quote response have the vendors provide a CSV, JSON or XML
file with the quote to be

\item {} 
automatically uploaded to the system.

\end{itemize}

For example a vendor could create a spreadsheet with the following
columns
\begin{itemize}
\item {} 
item\_cd

\item {} 
quantity

\item {} 
manufacturer

\item {} 
price

\item {} 
available date

\end{itemize}

by emailing to \sphinxhref{mailto:quotes@yourco.com}{quotes@yourco.com} these quotes can be automatically
loaded into the system without changes to the legacy system,


\subsection{Buyer information}
\label{\detokenize{BusinessProcessReengineering:id10}}
The buyer should have single screen that shows:
\begin{enumerate}
\sphinxsetlistlabels{\arabic}{enumi}{enumii}{}{.}%
\item {} 
Supplies

\item {} 
On hand

\item {} 
Open Purchase Orders

\item {} 
Open Work Orders

\item {} 
Demand

\item {} 
Forecasted
\begin{enumerate}
\sphinxsetlistlabels{\arabic}{enumii}{enumiii}{}{.}%
\item {} 
Raw

\item {} 
Consumed

\item {} 
Unconsumed

\end{enumerate}

\item {} 
Safety Stock

\item {} 
Work Orders

\item {} 
Reserved Inventory

\item {} 
Quarantined

\item {} 
Restricted access (JIT programs, Committed Service Level Agreement
Plans)

\item {} 
All part numbers in the planning group

\item {} 
Every part and all equivalents, transitively, that is the equivalents
to those equivalents until exhausted.

\item {} 
Customer specific substitutions

\item {} 
Approved manufacturer matrix Customers down the left, manufacturers
across the top

\item {} 
Requisitions

\item {} 
Supplier on-time historical metrics

\item {} 
Supply ineligibility drill-down

\item {} 
Vendor Quotes

\item {} 
Time phased inventory position, Pipeline (Global, by Facility, by
planner)

\item {} 
On hand inventory in aggregate with the ability to open details with
a single click

\item {} 
Sales history for the last three years in multiple dimensions

\item {} 
Time Dimensions include annual, quarterly and monthly

\item {} 
Ability to see by customer

\item {} 
Existing purchase orders

\item {} 
Existing facility transfers in process

\item {} 
Detailed reason why supplies are not eligible for a demand that is
allocated late or short

\item {} 
A matrix of approved manufactures and customers

\item {} 
See the part and all transitive equivalent parts

\item {} 
Late or short demands

\end{enumerate}

In Diamond this is all done locally in the web browser with no network
requests so it is virtually instaneous.


\subsection{Recommendations}
\label{\detokenize{BusinessProcessReengineering:id11}}\begin{enumerate}
\sphinxsetlistlabels{\arabic}{enumi}{enumii}{}{.}%
\item {} 
Purchase orders that can be cancelled

\item {} 
Get a manufacturer Certificate of Compliance for existing inventory
to satisfy a requisition with existing inventory

\item {} 
Supply prioritization Use buyback inventory before using our
inventory for appropriate customers

\item {} 
Allocation based pricing

\item {} 
Items with a shelf life have oldest allocated first

\item {} 
Less valuable items are allocated first

\item {} 
Based on Certifications (dual certed parts have more value)

\item {} 
Facility Transfer

\item {} 
Supply Pool Transfer

\item {} 
Expedite or de-expedite a purchase order

\end{enumerate}


\subsection{Alerts}
\label{\detokenize{BusinessProcessReengineering:id12}}\begin{itemize}
\item {} 
Obsolete Inventory

\item {} 
Expiring Inventory

\item {} 
Purchase Exceeding x\% of previous maximum unit price

\item {} 
Purchase Exceeding x\% of previous minimum unit price

\item {} 
Purchase of specified dollars not yet approved

\end{itemize}


\section{Extensibility}
\label{\detokenize{BusinessProcessReengineering:id13}}
Any component must be easily plugged in with an alternative
implementation that is compliant with the corresponding interface,
\begin{itemize}
\item {} 
Demand Priority

\item {} 
Eligibility Requirements

\item {} 
Supply Prioritization
\begin{itemize}
\item {} 
Lot value determination

\end{itemize}

\item {} 
Recommendation Handlers for propagating accepted recommendations to
source system

\end{itemize}


\chapter{Implementation}
\label{\detokenize{BusinessProcessReengineering:id14}}\begin{itemize}
\item {} 
Extract necessary data from legacy systems

\item {} 
Load into Advanced Planning

\item {} 
Augment with necessary but unavailable information

\item {} 
Run a full plan

\item {} 
Review recommendations
\begin{itemize}
\item {} 
Accept recommendation (must define Action Handlers) Reject
recommendation (select reason to be persisted across full reloads)

\end{itemize}

\end{itemize}


\chapter{Modifications}
\label{\detokenize{BusinessProcessReengineering:modifications}}\begin{itemize}
\item {} 
All code is in Java supported by Spring with Hibernate for Object
Relation Management, these are widely adoped open source solutions

\item {} 
Presentation uses the Model, View Controller approach and the model
may be exposed as a JavaBean or XML if one prefers to use XSL.

\item {} 
Diamond dependencies are all vastly popular open source but it
extremely unlikely anyone will have any need to modify anyy of the
open source code

\item {} 
I have trained non-programmers to modify Diamond in less than a
month.

\end{itemize}


\section{Questions}
\label{\detokenize{BusinessProcessReengineering:id15}}\begin{enumerate}
\sphinxsetlistlabels{\arabic}{enumi}{enumii}{}{.}%
\item {} 
Inventory Restriction

\item {} 
How do you restrict availability of inventory for special purposes
such as
\begin{itemize}
\item {} 
JIT contracts

\item {} 
Committed Service Level Agreements

\item {} 
Kitting and Assembly

\end{itemize}

\item {} 
Do you have automated approved manufacturer eligibility?

\item {} 
Do you have prioritization for lots with expiry dates?

\item {} 
How do you calculate the residual cost of goods for broker buys for
the

\item {} 
Are you exclusively FIFO or do you consider lots that have lower cost
that satisfies the demand (taking into consideration multiple
certifications, incremental cost of Quality Assurance testing and
destructive tests?), etc.? quantity that exceeds the customer demand?

\item {} 
Quality Assurance Do you have a quality assurance program that
supports skip lot testing and pre-approved lots ( lots that have
already passed the QA requirements for a customer should receive
higher priority for that customer and lower priority for others)

\item {} 
What supply eligibility rules do you have?

\item {} 
How do you pin an allocation to a demand ?

\item {} 
Does your system recommend when alternate availability is preferable
to a pinned allocation?

\end{enumerate}


\chapter{Difference from SAP}
\label{\detokenize{BusinessProcessReengineering:difference-from-sap}}\begin{enumerate}
\sphinxsetlistlabels{\arabic}{enumi}{enumii}{}{.}%
\item {} 
Diamond client interface is natively HTML5 and all functionality is
available in a simple browser, whereas SAP has a cludgy interface,
the native interface is client-server, a technology that is
embarassingly old fashioned and the implementation is poor, at best
\sphinxurl{https://help.sap.com/doc/saphelp\_nw70ehp1/7.01.16/en-US/4d/aeae42cd7fb611e10000000a155106/content.htm?no\_cache=true}

\item {} 
There are no exposed transaction codes in Diamond, everything is menu
driven, reducing training time and simplifying operation efficiency.

\item {} 
SAP has a proprietary database, HANA

\item {} 
Diamond is not complicated it uses International Standards for
everything, and the most widely adopted technologies, there is no
dependency on anything proprietary unless the oracle database is
used.

\end{enumerate}


\chapter{SAP on the web}
\label{\detokenize{BusinessProcessReengineering:id16}}\begin{itemize}
\item {} 
\sphinxurl{https://www.brightworkresearch.com/sap/2017/11/best-understand-saps-negative-innovation/}

\item {} 
\sphinxurl{https://boards.straightdope.com/sdmb/archive/index.php/t-509111.html}

\item {} 
\sphinxurl{https://www.linkedin.com/pulse/who-knew-sap-could-so-complicated-heather-peyton/}

\item {} 
“We are nowhere near best-in-class, but we are making progress,” says
Steve Rogers, UK managing director of SAP to an audience of his
customers at the German applications firm’s annual user gathering at
the end of last year. It’s not the kind of comment that you expect
from a senior executive at a leading software firm.

\item {} 
\sphinxurl{https://www.thirdstage-consulting.com/lessons-from-an-sap-failure-at-lidl/}

\item {} 
\sphinxurl{https://www.360cloudsolutions.com/top-six-erp-implementation-failures/}

\item {} 
\sphinxurl{https://www.brightworkresearch.com/saphana/2017/06/22/hana-big-data-equals-big-failure/}

\end{itemize}


\chapter{Questions}
\label{\detokenize{Questions:questions}}\label{\detokenize{Questions::doc}}\begin{enumerate}
\sphinxsetlistlabels{\arabic}{enumi}{enumii}{}{.}%
\item {} 
How many buyers are there?

\item {} 
How many purchase orders are created per year?

\item {} 
How much time does it take to create a purchase order?

\item {} 
Gather the information
\begin{itemize}
\item {} 
Onhand Inventory

\item {} 
Sales History

\item {} 
Sales Forecast

\item {} 
Forecast Consumption

\item {} 
Approved Manufacturers

\item {} 
Vendor Quotes

\item {} 
Existing Purchase Orders

\end{itemize}
\begin{enumerate}
\sphinxsetlistlabels{\arabic}{enumii}{enumiii}{}{.}%
\item {} 
Vendor Quotes

\end{enumerate}
\begin{itemize}
\item {} 
Get quotes for the parts

\item {} 
Compute Optimal Replenishment Quantity (Not Economic Order
Quantity)

\item {} 
Create a requisition

\end{itemize}

\item {} 
What are the requisition review requirements?
\begin{itemize}
\item {} 
When is a requisition subject to review?

\item {} 
What are the needs additional work conditions?
\begin{itemize}
\item {} 
Need more vendor quotes

\item {} 
Need work on equivalent parts

\item {} 
Incorrect buy quantity

\item {} 
Get existing inventory certified.

\end{itemize}

\end{itemize}

\end{enumerate}


\section{Out of Scope}
\label{\detokenize{Questions:out-of-scope}}
Lead times can vary drastically based on
\begin{itemize}
\item {} 
vendor

\item {} 
material shortages

\item {} 
new product introduction

\item {} 
product recall MRO (Maintenance, Repair and Overhaul)Definition

\end{itemize}

The elapsed time, usually measured in weeks between when a product
is ordered and receivied.

Usually not considered
\begin{itemize}
\item {} 
Recieving time

\item {} 
Inspection Time

\item {} 
Intra-facility transfer time

\item {} 
These are consider \sphinxstyleemphasis{availability times}

\end{itemize}

Typically in DRP Planning
\begin{itemize}
\item {} 
There is only one lead time

\item {} 
This lead time is fairly consistent

\item {} 
Costs due not vary based on lead time.

\end{itemize}


\section{In aerospace}
\label{\detokenize{Questions:in-aerospace}}

\section{Source of Lead Time}
\label{\detokenize{Questions:source-of-lead-time}}

\subsection{vendor quotes}
\label{\detokenize{Questions:vendor-quotes}}
Take the lead time from the maximum quote expiration date


\subsection{Summarized Lead Times should include}
\label{\detokenize{Questions:summarized-lead-times-should-include}}\begin{itemize}
\item {} 
Vendor Code

\item {} 
Vendor type (Manufacturer, Distibutor, intra-company

\item {} 
Vendor quote beginning and ending effective date

\item {} 
Date of request for quote

\end{itemize}


\subsection{Lead time details should include:}
\label{\detokenize{Questions:lead-time-details-should-include}}\begin{itemize}
\item {} 
Historical lead times

\end{itemize}


\subsection{Lead time projections}
\label{\detokenize{Questions:lead-time-projections}}
Factors that can effect lead time, Lead times can vary drastically based
on
\begin{itemize}
\item {} 
vendor

\item {} 
Some vendors will stock

\item {} 
Some manufacturers will build to stock

\item {} 
Some manufactureres will build only on demand, see Cost per Unit

\item {} 
material shortages

\item {} 
new product introduction

\item {} 
product recall MRO (Maintenance, Repair and Overhaul)

\end{itemize}

\begin{DUlineblock}{0em}
\item[] 
\end{DUlineblock}


\section{Instructions}
\label{\detokenize{Questions:instructions}}\begin{enumerate}
\sphinxsetlistlabels{\arabic}{enumi}{enumii}{}{.}%
\item {} 
Look at the report section of the portal

\item {} 
Review the information

\item {} 
Fill out the checklist

\item {} 
Create the requisition

\end{enumerate}


\subsection{Requisition Checklist}
\label{\detokenize{Questions:requisition-checklist}}

\section{Sequence Diagram}
\label{\detokenize{Questions:sequence-diagram}}
\noindent\sphinxincludegraphics{{Requisitions1}.png}


\subsection{Intrinsic Information}
\label{\detokenize{Questions:intrinsic-information}}

\subsection{Extrinsic Information}
\label{\detokenize{Questions:extrinsic-information}}
Gather all information, it should all be available in the portal


\section{Related articles}
\label{\detokenize{Questions:related-articles}}

\chapter{Service Level Agreements}
\label{\detokenize{Questions:service-level-agreements}}

\chapter{Definitions}
\label{\detokenize{Questions:definitions}}
Definitions:

\begin{sphinxVerbatim}[commandchars=\\\{\}]
\PYG{n}{ABC\PYGZus{}SLS\PYGZus{}PCT\PYGZus{}DLR}

    \PYG{n}{Top} \PYG{l+m+mi}{20} \PYG{n}{of} \PYG{n}{previous} \PYG{l+m+mi}{12} \PYG{n}{month} \PYG{n}{contribution} \PYG{n}{to} \PYG{n}{sales} \PYG{n}{dollars}
    \PYG{n}{Does} \PYG{o+ow}{not} \PYG{n}{adequately} \PYG{n}{support} \PYG{n}{new} \PYG{n}{product} \PYG{n}{introduction}

\PYG{n}{ABC\PYGZus{}CUST\PYGZus{}QTE}

\PYG{n}{SVC\PYGZus{}LVL}

    \PYG{n}{Contractual} \PYG{n}{service} \PYG{n}{levels}

\PYG{n}{ABC\PYGZus{}SLS\PYGZus{}PCT} \PYG{n}{ABC} \PYG{n}{Sales} \PYG{n}{Percent}
\end{sphinxVerbatim}

..image:: Portal/images/ActionReport.png


\chapter{Indices and tables}
\label{\detokenize{index:indices-and-tables}}\begin{itemize}
\item {} 
\DUrole{xref,std,std-ref}{genindex}

\item {} 
\DUrole{xref,std,std-ref}{modindex}

\item {} 
\DUrole{xref,std,std-ref}{search}

\end{itemize}



\renewcommand{\indexname}{Index}
\printindex
\end{document}