\documentclass[a4paper,10pt]{book}
\usepackage[latin1]{inputenc}
\usepackage{amsfonts}
\usepackage[T1]{fontenc}
\usepackage[dvips]{graphicx}
\usepackage{alltt} 
\usepackage{alltt} 
 \usepackage[top=2cm, bottom=2cm, left=1.5cm, right=1.5cm]{geometry} 
\usepackage[bookmarks=true]{hyperref}
\usepackage{underscore}
%opening
\title{Exception Generation with Javautil}
\author{jim.schmidt@dbexperts.com}
\hypersetup{
pdfauthor = {jim.schmidt@dbexperts.com},
pdftitle = {Exception Generation with Javautil},
pdfsubject = {Exception Generation with Javautil},
pdfkeywords = {Exception, Installation, Configuration, Dexterous},
pdfcreator = {LaTeX with hyperref package},
pdfproducer = {dvips + ps2pdf}}
\begin{document}
\title{Exception Generation with Javautil}
\author{Jim Schmidt\\
  \texttt{jjs-javautil@dbexperts.com}}
\date{\today}
\maketitle
\tableofcontents
\chapter{Introduction}
Relational databases have a great number of features to enforce data integrity such as

\section{todo}
\begin{itemize} 
 \item fix schema tables with no surrogate key, no foreign keys etc. See exception processing.mer
 \item document the package
 \item document logging
 \item document metrics
 \item document ut_table_msg
 \item document ut_table_report_sum
 \item document exclusion rules
 \item need functional area
 \item document calling as a procedure
 \item what does ut_query do with anything
 \item no real support for ut_table_msg
  \item todo what is ds_table and why does ut_query reference it?
 \item document ut_table_rule and hist figure out what all of these columns are used for
 \begin{itemize}
  \item is ut_query used
 \end{itemize}

\end{itemize}

\begin{itemize}
 \item Primary Keys
 \item Foreign Keys
 \item Not Null
 \item Check Constraints
\end{itemize}
However, there are many logical conditions which are beyond the scope of available functionality.

The Javautil Exception Generator allows you to set up simple rules to identify records or tables that fail
to meet business requirements.

\section{Benefits}

\chapter{Pre-requisites}
\begin{itemize}
 \item Obtain javautil code
 \item configure machine
 \item configure datasources
\end{itemize}
\chapter{How it works}
\section{Overview}
\begin{itemize}
 \item getParms();
 \item getRun();
 \item getRules();
 \item getBinds();
 \item createProcessLog();
 \item processRules();
 \item updateRunStatus();
 \item acknowledge()
\end{itemize}
\section{Creating the User}
grant create sequence to &&user;
\section{Parameters}
\subsection{Run Number}
\paragraph{UT_RULE_GRP_RUN_NBR}
\section{Get UtRuleGrpRun}
\section{Get UT_TABLE_RULE}
Get the rules for the run.
\section{Get UT_RULE_GRP_RUN_PARMS}
\section{Process Rules}
Connect to source - todo describe data
Connect to destination
binds
Run the query
insert into gtt_ut_table_row_msg
merge into ut_table_row
delete where they don't exist
\chapter {Database Objects}
Also depends on the logging tables in Dbexperts3/ddl/oracle/logging
%\includegraphics*[width=\textwidth,viewport=0 0 3046 1632, bb=0 0  0 0]{ExceptionProcessing.jpeg}

\includegraphics*[width=\textwidth,viewport=0 0 3046 1632, bb=0 0 3046 1632]{ExceptionProcessing.jpeg}

\begin{itemize}
	\item huh
\end{itemize}

To generate the tables not only are the mapping files required, the associated beans are even though they are 
never used. 

The dto's must be in the classpath.

http://docs.jboss.org/hibernate/core/3.3/reference/en/html/toolsetguide.html#toolsetguide-s1-2




source and destination databases may be different 

declarative rules

TODO need to support initialization procedure



\end{document}
