%% Generated by Sphinx.
\def\sphinxdocclass{report}
\documentclass[letterpaper,10pt,english]{sphinxmanual}
\ifdefined\pdfpxdimen
   \let\sphinxpxdimen\pdfpxdimen\else\newdimen\sphinxpxdimen
\fi \sphinxpxdimen=.75bp\relax

\PassOptionsToPackage{warn}{textcomp}
\usepackage[utf8]{inputenc}
\ifdefined\DeclareUnicodeCharacter
% support both utf8 and utf8x syntaxes
  \ifdefined\DeclareUnicodeCharacterAsOptional
    \def\sphinxDUC#1{\DeclareUnicodeCharacter{"#1}}
  \else
    \let\sphinxDUC\DeclareUnicodeCharacter
  \fi
  \sphinxDUC{00A0}{\nobreakspace}
  \sphinxDUC{2500}{\sphinxunichar{2500}}
  \sphinxDUC{2502}{\sphinxunichar{2502}}
  \sphinxDUC{2514}{\sphinxunichar{2514}}
  \sphinxDUC{251C}{\sphinxunichar{251C}}
  \sphinxDUC{2572}{\textbackslash}
\fi
\usepackage{cmap}
\usepackage[T1]{fontenc}
\usepackage{amsmath,amssymb,amstext}
\usepackage{babel}



\usepackage{times}
\expandafter\ifx\csname T@LGR\endcsname\relax
\else
% LGR was declared as font encoding
  \substitutefont{LGR}{\rmdefault}{cmr}
  \substitutefont{LGR}{\sfdefault}{cmss}
  \substitutefont{LGR}{\ttdefault}{cmtt}
\fi
\expandafter\ifx\csname T@X2\endcsname\relax
  \expandafter\ifx\csname T@T2A\endcsname\relax
  \else
  % T2A was declared as font encoding
    \substitutefont{T2A}{\rmdefault}{cmr}
    \substitutefont{T2A}{\sfdefault}{cmss}
    \substitutefont{T2A}{\ttdefault}{cmtt}
  \fi
\else
% X2 was declared as font encoding
  \substitutefont{X2}{\rmdefault}{cmr}
  \substitutefont{X2}{\sfdefault}{cmss}
  \substitutefont{X2}{\ttdefault}{cmtt}
\fi


\usepackage[Bjarne]{fncychap}
\usepackage{sphinx}

\fvset{fontsize=\small}
\usepackage{geometry}

% Include hyperref last.
\usepackage{hyperref}
% Fix anchor placement for figures with captions.
\usepackage{hypcap}% it must be loaded after hyperref.
% Set up styles of URL: it should be placed after hyperref.
\urlstyle{same}
\addto\captionsenglish{\renewcommand{\contentsname}{Contents:}}

\usepackage{sphinxmessages}
\setcounter{tocdepth}{1}



\title{pdssr Documentation}
\date{Jan 10, 2020}
\release{0.0.3}
\author{Pacific Data Services}
\newcommand{\sphinxlogo}{\vbox{}}
\renewcommand{\releasename}{Release}
\makeindex
\begin{document}

\pagestyle{empty}
\sphinxmaketitle
\pagestyle{plain}
\sphinxtableofcontents
\pagestyle{normal}
\phantomsection\label{\detokenize{index::doc}}



\chapter{Introduction}
\label{\detokenize{index:introduction}}
Pepsi-Frito uses Custom Data solutions to record and report vending sales.  I am extremely
familiar with this operation as I taught Custom Data Solutions the technology for 25 years
and functioned as their first Chief Information Officer

I redesigned the entire process in the two years I was an employee.

More in {\hyperref[\detokenize{Contributions::doc}]{\sphinxcrossref{\DUrole{doc}{Contributions}}}}


\chapter{Benefits}
\label{\detokenize{index:benefits}}\begin{itemize}
\item {} 
Audit the data and rebate calculations. {\hyperref[\detokenize{Audit::doc}]{\sphinxcrossref{\DUrole{doc}{Audit}}}}

\item {} 
Get answers without asking questions {\hyperref[\detokenize{Answers::doc}]{\sphinxcrossref{\DUrole{doc}{Answers to Questions you should have asked}}}}

\item {} 
Custom Reporting

\item {} 
Online analytical processing

\item {} 
Startup is negligible
\begin{itemize}
\item {} 
Just need copies of the files the distributors already report {\hyperref[\detokenize{cds_record_layout::doc}]{\sphinxcrossref{\DUrole{doc}{Record Layouts}}}}

\item {} 
A product master file

\item {} 
No computing resources, no long term contracts

\end{itemize}

\end{itemize}

Once installed:
\begin{enumerate}
\sphinxsetlistlabels{\arabic}{enumi}{enumii}{}{.}%
\item {} 
Reporting data is loaded into staging tables

\item {} 
Verification of compliance with requirements and self consistency of data is ensured

\item {} 
Data is posted to history tables

\item {} 
Customer addresses are validated

\item {} 
Exceptional conditions are identified and reported

\item {} 
Rebate Calculations are made and reported

\item {} 
Rebate amounts are computed and an ACH file for transmission to a bank for payment is created

\item {} 
Spreadsheets for sales analysis are created

\end{enumerate}


\chapter{Functioning Real World Usage}
\label{\detokenize{index:functioning-real-world-usage}}
Any distributor currently writing files in the CDS reporting format can easily load files
into a local database in order to analyze the data.

We have provided many analyses that report data worthy of further inspection.

This approach is our “answers to questions you should have asked”.


\chapter{Audit}
\label{\detokenize{index:audit}}
Pacific Data Services will audit the data and rebate calculations for your distributors.


\section{Findings}
\label{\detokenize{index:findings}}

\subsection{Item Accuracy}
\label{\detokenize{index:item-accuracy}}
We will compare the items assigned by Custdata to the actual items sold by the distributors.

This affect sales history and rebate calculations


\subsection{Addresses}
\label{\detokenize{index:addresses}}
Report incorrect addresses and undeliverable addresses.


\subsection{Rebate}
\label{\detokenize{index:rebate}}
Calculate the actual rebate for the top 50 rebate amounts issued versus what Custdata
computed.


\section{Report augmentation}
\label{\detokenize{index:report-augmentation}}

\subsection{Answers to Questions you should have asked}
\label{\detokenize{index:answers-to-questions-you-should-have-asked}}
Rather than spend time on the Custdata website creating spreadsheets and trying to spot
interesting and unusual data we will create \sphinxstyleemphasis{Alert Reports} and notify you that
there is research warranted.


\subsubsection{Examples}
\label{\detokenize{index:examples}}\begin{itemize}
\item {} 
Newly introduced items not being bought by distributor

\item {} 
Top selling items by category not being bought by distributor

\item {} 
Top selling items by category not being bought by a vending operator

\item {} 
Product declining sales for a vending operator

\item {} 
Dropped products for a vending operator

\item {} 
Parasitic promotion effect (Items not on promotion having declining sales during a promotion

\end{itemize}

as slot contention has had a parasitic effect.


\chapter{Sales}
\label{\detokenize{index:sales}}

\section{Required Data}
\label{\detokenize{index:required-data}}
Distributors already report their data to custdata.com.

They have to do nothing but send us the same data files, no additional work.


\section{Benefits}
\label{\detokenize{index:id1}}

\subsection{Manufacturers}
\label{\detokenize{index:manufacturers}}

\subsubsection{Benefits}
\label{\detokenize{index:id2}}\begin{itemize}
\item {} 
Far lower costs

\item {} 
Faster and Cheaper Custom Reports Generation

\item {} 
Improved Data Quality

\item {} 
Far Lower Costs

\item {} 
Data Identified for them

Currently the manufactures have to click through the Custdata website and create a spreadsheet and see if there is
interesting data.

We can analyze the data and inform them of interesting conditions, for example

\end{itemize}
\begin{quote}
\begin{itemize}
\item {} 
New Vending Operators

\item {} 
Vendors with Declining Sales

\item {} 
Vendors not carrying products

\end{itemize}
\end{quote}


\subsubsection{New Capabilities}
\label{\detokenize{index:new-capabilities}}\begin{itemize}
\item {} 
Address Correction

\item {} 
\end{itemize}


\subsection{Distributors}
\label{\detokenize{index:distributors}}\begin{itemize}
\item {} 
Ability to create an operator facing website

\end{itemize}


\subsection{Vending Operators}
\label{\detokenize{index:vending-operators}}\begin{itemize}
\item {} 
Find new  products to be offered

\end{itemize}


\chapter{Background}
\label{\detokenize{index:background}}

\section{Sales Analysis}
\label{\detokenize{Analyze:sales-analysis}}\label{\detokenize{Analyze::doc}}

\subsection{Gather Product Statistics}
\label{\detokenize{Analyze:gather-product-statistics}}\begin{itemize}
\item {} 
Total sales prior 12 months

\item {} 
Percent of total sales by product

\item {} 
Operator maximum percentage

\item {} 
First sale date

\item {} 
Trend line slope

\item {} 
Seasonality

\end{itemize}


\subsection{Operator Product Statistics}
\label{\detokenize{Analyze:operator-product-statistics}}\begin{itemize}
\item {} 
Pct of sales for product by operator

\item {} 
Ratio of percent of sales for this operator to median percentage

\item {} 
Total dollars sales by product previous twelve months

\end{itemize}


\subsection{Opportunities}
\label{\detokenize{Analyze:opportunities}}

\subsection{Operator Call Report}
\label{\detokenize{Analyze:operator-call-report}}\begin{itemize}
\item {} 
Product mix to standard mix

\item {} 
Suggested products to carry

\end{itemize}


\section{Answers to Questions you should have asked}
\label{\detokenize{Answers:answers-to-questions-you-should-have-asked}}\label{\detokenize{Answers::doc}}
We do the work of finding data that is interesting and provide email notices
that these reports exist.  Anyone subscribing to that event can get an email summary
notification and click directly to the report, only having to login.  We can support
gmail or facebook authentication so the user doesn’t need another username and password.

It couldn’t be easier.

Answers phone friendy format or downloadable in a variety of formats.


\subsection{Examples}
\label{\detokenize{Answers:examples}}\begin{itemize}
\item {} 
Which operators are not selling recently introduced products

\item {} 
Which operators have a sub-optimal mix of the top selling items by category

\item {} 
Product declining sales for a vending operator

\item {} 
Dropped products for a vending operator

\item {} 
Parasitic promotion effect (Items not on promotion having declining sales during a promotion

\end{itemize}

as slot contention has had a parasitic effect.


\section{Sales Analysis}
\label{\detokenize{Answers:sales-analysis}}
List customers and the products they are should be selling but are not.
\begin{itemize}
\item {} 
Rank Products

\item {} 
\end{itemize}

As we are dealing with CDS reporting format and the scope of creation of customers in posting
is beyond the scope of this section I will create a customer with a simple database view.

code:

\begin{sphinxVerbatim}[commandchars=\\\{\}]
\PYG{n}{create} \PYG{o+ow}{or} \PYG{n}{replace} \PYG{n}{view}
\PYG{n}{customer\PYGZus{}vw} \PYG{k}{as}
\PYG{n}{select} \PYG{n}{distinct} \PYG{n}{ship\PYGZus{}to\PYGZus{}cust\PYGZus{}id}
\PYG{k+kn}{from} \PYG{n+nn}{etl\PYGZus{}sale}\PYG{p}{;}
\end{sphinxVerbatim}

In order for these queries to work without recreating the database we assume
an \sphinxstyleemphasis{effective date} which is the date of the last reported sale.
\begin{description}
\item[{code::}] \leavevmode
create or replace view effective\_date as
select max(invoice\_dt) report\_date from etl\_sale;

\end{description}

Total \$ amount of sales for the preceeding 12 months based on last invoice date.

code:

\begin{sphinxVerbatim}[commandchars=\\\{\}]
\PYG{n}{create} \PYG{o+ow}{or} \PYG{n}{replace} \PYG{n}{view} \PYG{n}{tot\PYGZus{}sales\PYGZus{}12} \PYG{k}{as}
\PYG{n}{select} \PYG{n+nb}{sum}\PYG{p}{(}\PYG{n}{extended\PYGZus{}net\PYGZus{}amt}\PYG{p}{)} \PYG{n}{tot\PYGZus{}extended\PYGZus{}net\PYGZus{}amt}
\PYG{k+kn}{from} \PYG{n+nn}{etl\PYGZus{}sale}\PYG{p}{,}
     \PYG{n}{effective\PYGZus{}date}
\PYG{n}{where} \PYG{n}{invoice\PYGZus{}dt}  \PYG{o}{\PYGZgt{}} \PYG{n}{effective\PYGZus{}date}\PYG{o}{.}\PYG{n}{report\PYGZus{}date} \PYG{o}{\PYGZhy{}} \PYG{n}{interval} \PYG{l+s+s1}{\PYGZsq{}}\PYG{l+s+s1}{1 year}\PYG{l+s+s1}{\PYGZsq{}}\PYG{p}{;}
\end{sphinxVerbatim}

code:

\begin{sphinxVerbatim}[commandchars=\\\{\}]
\PYG{n}{create} \PYG{o+ow}{or} \PYG{n}{replace} \PYG{n}{view} \PYG{n}{customer\PYGZus{}product\PYGZus{}last\PYGZus{}12\PYGZus{}months} \PYG{k}{as}
\PYG{n}{select}  \PYG{n}{etl\PYGZus{}sale}\PYG{o}{.}\PYG{n}{ship\PYGZus{}to\PYGZus{}cust\PYGZus{}id}\PYG{p}{,}
        \PYG{n}{case\PYGZus{}gtin}\PYG{p}{,}
        \PYG{n+nb}{sum}\PYG{p}{(}\PYG{n}{extended\PYGZus{}net\PYGZus{}amt}\PYG{p}{)} \PYG{n}{sum\PYGZus{}extended\PYGZus{}net\PYGZus{}amt}
\PYG{k+kn}{from}    \PYG{n+nn}{etl\PYGZus{}sale}\PYG{p}{,}
        \PYG{n}{effective\PYGZus{}date}
\PYG{n}{where}   \PYG{n}{invoice\PYGZus{}dt}  \PYG{o}{\PYGZgt{}} \PYG{n}{effective\PYGZus{}date}\PYG{o}{.}\PYG{n}{report\PYGZus{}date} \PYG{o}{\PYGZhy{}} \PYG{n}{interval} \PYG{l+s+s1}{\PYGZsq{}}\PYG{l+s+s1}{1 year}\PYG{l+s+s1}{\PYGZsq{}}
\PYG{n}{group} \PYG{n}{by} \PYG{n}{ship\PYGZus{}to\PYGZus{}cust\PYGZus{}id}\PYG{p}{,}
        \PYG{n}{case\PYGZus{}gtin}
\PYG{n}{order} \PYG{n}{by} \PYG{n+nb}{sum}\PYG{p}{(}\PYG{n}{extended\PYGZus{}net\PYGZus{}amt}\PYG{p}{)} \PYG{n}{desc}\PYG{p}{;}
\end{sphinxVerbatim}

There are many ways to rank product.

Units, gross revenue, profit, turns…

This is based simply on gross revenue

code:

\begin{sphinxVerbatim}[commandchars=\\\{\}]
\PYG{n}{create} \PYG{o+ow}{or} \PYG{n}{replace} \PYG{n}{view} \PYG{n}{product\PYGZus{}rank\PYGZus{}12\PYGZus{}vw} \PYG{k}{as}
\PYG{n}{select} \PYG{n}{etl\PYGZus{}sale}\PYG{o}{.}\PYG{n}{case\PYGZus{}gtin}\PYG{p}{,} \PYG{n+nb}{sum}\PYG{p}{(}\PYG{n}{etl\PYGZus{}sale}\PYG{o}{.}\PYG{n}{extended\PYGZus{}net\PYGZus{}amt}\PYG{p}{)} \PYG{n}{sum\PYGZus{}ext}\PYG{p}{,}
       \PYG{n+nb}{sum}\PYG{p}{(}\PYG{n}{etl\PYGZus{}sale}\PYG{o}{.}\PYG{n}{extended\PYGZus{}net\PYGZus{}amt}\PYG{p}{)} \PYG{o}{*} \PYG{l+m+mi}{100} \PYG{o}{/} \PYG{n}{tot\PYGZus{}sales\PYGZus{}12}\PYG{o}{.}\PYG{n}{tot\PYGZus{}extended\PYGZus{}net\PYGZus{}amt}
\PYG{k+kn}{from} \PYG{n+nn}{etl\PYGZus{}sale}\PYG{p}{,}
     \PYG{n}{tot\PYGZus{}sales\PYGZus{}12}
     \PYG{n}{effective\PYGZus{}date}
\PYG{n}{where} \PYG{n}{invoice\PYGZus{}dt}  \PYG{o}{\PYGZgt{}} \PYG{n}{effective\PYGZus{}date}\PYG{o}{.}\PYG{n}{report\PYGZus{}date} \PYG{o}{\PYGZhy{}} \PYG{n}{interval} \PYG{l+s+s1}{\PYGZsq{}}\PYG{l+s+s1}{1 year}\PYG{l+s+s1}{\PYGZsq{}}
                \PYG{n}{group} \PYG{n}{by} \PYG{n}{case\PYGZus{}gtin}
                \PYG{n}{order} \PYG{n}{by} \PYG{n+nb}{sum}\PYG{p}{(}\PYG{n}{extended\PYGZus{}net\PYGZus{}amt}\PYG{p}{)} \PYG{n}{desc}
        \PYG{p}{;}
\end{sphinxVerbatim}


\section{Product Not Sold}
\label{\detokenize{Answers:product-not-sold}}
What products are customers not selling?

code:

\begin{sphinxVerbatim}[commandchars=\\\{\}]
\PYG{n}{create} \PYG{o+ow}{or} \PYG{n}{replace} \PYG{n}{view} \PYG{n}{product\PYGZus{}undersold\PYGZus{}by\PYGZus{}customer\PYGZus{}vw} \PYG{k}{as}
\PYG{n}{select} \PYG{n}{customers}\PYG{o}{.}\PYG{n}{ship\PYGZus{}to\PYGZus{}cust\PYGZus{}id}\PYG{p}{,}
       \PYG{n}{top\PYGZus{}products}\PYG{o}{.}\PYG{n}{case\PYGZus{}gtin}
\PYG{k+kn}{from} \PYG{n+nn}{customers}\PYG{p}{,}
     \PYG{n}{top\PYGZus{}products}
\PYG{n}{where} \PYG{o+ow}{not} \PYG{n}{exists}
  \PYG{p}{(} \PYG{n}{select} \PYG{l+s+s1}{\PYGZsq{}}\PYG{l+s+s1}{x}\PYG{l+s+s1}{\PYGZsq{}} \PYG{k+kn}{from} \PYG{n+nn}{etl\PYGZus{}sale}\PYG{p}{,} \PYG{n}{top\PYGZus{}products}
    \PYG{n}{where}
        \PYG{n}{etl\PYGZus{}sale}\PYG{o}{.}\PYG{n}{ship\PYGZus{}to\PYGZus{}cust\PYGZus{}id} \PYG{o}{=} \PYG{n}{customers}\PYG{o}{.}\PYG{n}{ship\PYGZus{}to\PYGZus{}cust\PYGZus{}id}  \PYG{o+ow}{and}
        \PYG{n}{etl\PYGZus{}sale}\PYG{o}{.}\PYG{n}{case\PYGZus{}gtin} \PYG{o}{=} \PYG{n}{top\PYGZus{}products}\PYG{o}{.}\PYG{n}{case\PYGZus{}gtin}
  \PYG{p}{)}
\PYG{n}{order} \PYG{n}{by} \PYG{n}{customers}\PYG{o}{.}\PYG{n}{ship\PYGZus{}to\PYGZus{}cust\PYGZus{}id}\PYG{p}{,}
         \PYG{n}{case\PYGZus{}gtin}\PYG{p}{;}
\end{sphinxVerbatim}

Now let us consider the amount that could be sold if these customers were to sell
your standard ratio of product.

We do that by multiplying the ratio to the dollar total for the customer.

In the best case scenario, the vending operator will add a slot for the product, replacing
with a product you don’t distribute to him.  There is a distinct possibility that the sale
will be slightly parasitic and a lower volume product you do distribute will be replaced.


\section{Audit}
\label{\detokenize{Audit:audit}}\label{\detokenize{Audit::doc}}
Pacific Data Services will audit the data and rebate calculations for your distributors.


\subsection{Findings}
\label{\detokenize{Audit:findings}}

\subsubsection{Item Accuracy}
\label{\detokenize{Audit:item-accuracy}}
We will compare the items assigned by Custdata to the actual items sold by the distributors.

This affect sales history and rebate calculations


\subsubsection{Addresses}
\label{\detokenize{Audit:addresses}}
Report incorrect addresses and undeliverable addresses.


\subsubsection{Rebate}
\label{\detokenize{Audit:rebate}}
Calculate the actual rebate for the top 50 rebate amounts issued versus what Custdata
computed.


\section{Benefits}
\label{\detokenize{Benefits:benefits}}\label{\detokenize{Benefits::doc}}

\subsection{Frito-Lay Pepsi}
\label{\detokenize{Benefits:frito-lay-pepsi}}

\subsubsection{Audit}
\label{\detokenize{Benefits:audit}}
Identify product miscategorization.  Product GTINS may be “guessed” from the product description,
this is error prone.


\subsubsection{ACH rebate payments}
\label{\detokenize{Benefits:ach-rebate-payments}}
Rather than mail out checks, ACH transactions can be made, Pepsico-Frito can review and forward to
its bank and rebate payments can be sent out electronically for about \$0.01 per transaction.

This is faster, less expensive and less error prone than mailing out checks and more convenient for
the distributors.


\subsubsection{Anomolies}
\label{\detokenize{Benefits:anomolies}}
Report outliers on data reporting.

This is performed by the \sphinxstyleemphasis{javautil.org} \sphinxstyleemphasis{Condition Identification} package, written by
Jim Schmidt for use by Trinity Technical Services customers to identify potential problems
with data loading into the data warehouse.

Jim Schmidt demonstrated this while building an entire rebate program for a Custom Data Solutions
client and it was used to replace the paper reporting system, saving a huge amount of time and increasing
the identification of suspect data using metrics derived from historical data.  This was completely
designed and written by Jim Schmidt prior to joining CDS.


\subsubsection{Product Assignment}
\label{\detokenize{Benefits:product-assignment}}

\subsection{Distributors}
\label{\detokenize{Benefits:distributors}}

\subsubsection{Address Standardization}
\label{\detokenize{Benefits:address-standardization}}
Correct and standardize all mailing addresses and report undeliverable USPS addresses for vending operators.


\subsubsection{Geo Coding}
\label{\detokenize{Benefits:geo-coding}}
Return latitude and longitude for all valid mailing addresses.


\subsubsection{Product Identification}
\label{\detokenize{Benefits:product-identification}}

\subsubsection{Phone App}
\label{\detokenize{Benefits:phone-app}}
Salesmen can generate reports from their phone and forward to vending operators.


\subsubsection{Customer Facing Reports}
\label{\detokenize{Benefits:customer-facing-reports}}

\subsection{Apps}
\label{\detokenize{Benefits:apps}}
Phone app for


\subsection{Vending Operators}
\label{\detokenize{Benefits:vending-operators}}

\section{Product Identification}
\label{\detokenize{Benefits:id1}}

\section{Contributions}
\label{\detokenize{Contributions:contributions}}\label{\detokenize{Contributions::doc}}
Jim Schmidt and his company Trinity Technical Services introduced and mentored
Custom Data Solutions on every technology they used since 1990.

In 1990 Chuck Schmidt, the owner and president of Custom Data Solutions was
developing software in assembly language on alphamicro computers.


\subsection{Unix/Linux}
\label{\detokenize{Contributions:unix-linux}}
I introduced Chuck to Unix, starting with Interactive Unix, a Kodak supported
version of AT\&T Unix System 5 in 1991.

Thereafter Trinity became a reseller of MIPS computers and then
Digital Equipment.


\subsection{Oracle}
\label{\detokenize{Contributions:oracle}}
Jim introduced Chuck to Oracle and provided training and technical assistance
from 1992 to 2007. Trinity staff trained Custom Data Staff on Oracle Forms,
Oracle Reports and the Database.


\subsection{C}
\label{\detokenize{Contributions:c}}

\subsection{Java}
\label{\detokenize{Contributions:java}}

\subsection{Became Employee}
\label{\detokenize{Contributions:became-employee}}
In 2007 I made an agreement with Custom Data Solutions to become
the CIO with the understanding that he could work remotely.

So I sold everything I owned, two houses, two boats, two jet-skis, two ATVs
and the rest of my possessions.

When my youngest son went off to college, I drove him to Purdue and
went to Michigan to fix a few problems for a couple of months, that
turned out to be over a year.

During that time period
\begin{itemize}
\item {} 
Migrated 10 vending databases to a single database

\item {} 
Upgraded from Oracle 7 to Oracle 11

\item {} 
Installed Oracle Enterprise manager and taught how to use it

\item {} 
Created a development and a test database. Formerly all testing was done in production.

\end{itemize}


\subsection{Technical Standards}
\label{\detokenize{Contributions:technical-standards}}\begin{itemize}
\item {} 
Estabished the use of build tools, Ant and Maven

\item {} 
Instituted source version control, CVS and Subversion

\item {} 
Unit Testing

\item {} 
Integration Testing

\item {} 
Designed and build a document management service

\item {} 
Migrated off Digital Unix to Linux

\item {} 
Introduced virtualization, vmware, openvz and virtualbox

\item {} 
\end{itemize}


\subsection{Products}
\label{\detokenize{Contributions:products}}

\subsubsection{Check 21}
\label{\detokenize{Contributions:check-21}}

\subsubsection{Document Management}
\label{\detokenize{Contributions:document-management}}

\subsubsection{Workbook Parser}
\label{\detokenize{Contributions:workbook-parser}}

\subsubsection{Dexterous}
\label{\detokenize{Contributions:dexterous}}

\subsection{Redesign}
\label{\detokenize{Contributions:redesign}}
CDS had a prospective client that needed rebate process


\subsubsection{Data Load}
\label{\detokenize{Contributions:data-load}}
I replaced the java program that loaded the data into a stored procedure
that loaded into staging tables.


\subsubsection{Exception Processing}
\label{\detokenize{Contributions:exception-processing}}
I employed the javautil Exception Processing system to identify load problems.

Unknown to me at the time, Custom Data Solutions identified problems with reports.
The report modification was cumbersome, execution was slow, printing was slow
and wasteful.

By generating metrics of minimum and maximum and standard deviations from
the mean the identification of outlier input data was greatly improved.


\subsubsection{Maintenance System}
\label{\detokenize{Contributions:maintenance-system}}
The maintenance and control system was all written in Apex, whereas CDS was
using Oracle Forms.  My staff trained CDS on developing in Apex.


\subsubsection{Posting}
\label{\detokenize{Contributions:posting}}

\bigskip\hrule\bigskip



\subsubsection{Workbooks}
\label{\detokenize{Contributions:workbooks}}

\subsection{Others}
\label{\detokenize{Contributions:others}}\begin{itemize}
\item {} 
Servlets

\item {} 
Javascripts

\item {} 
Excel workbooks

\item {} 
Detachable tablespaces

\item {} 
Materialized Views

\item {} 
Virtual Private Databases

\end{itemize}


\subsection{Javautil}
\label{\detokenize{Contributions:javautil}}
javautil.org was Jim Schmidt’s domain
javautil.com still is


\section{History of Jim Schmidt and Custom Data Solutions}
\label{\detokenize{Contributions:history-of-jim-schmidt-and-custom-data-solutions}}

\subsection{Creation}
\label{\detokenize{Contributions:creation}}
Custom Data Solutions was initially called CDS for Chuck and Dee Schmidt.

Chuck is my older brother and Dee is his wife.

Chuck left his job at Michigan National Bank to offer data processing services, initially using a TRS-80 to print mailing labels.

He subsequently bought an Alpha Micro \sphinxurl{http://www.s100computers.com/Hardware\%20Folder/Alpha\%20Micro/History/History.htm} and wrote some business applications.

On a visit to Detroit I explained to him the benefits of Unix and got him started, installing Interactive Unix from Kodak on a PC from 133 floppy disks in 1990.

Between 1983 and 1990 I worked as Vice President of International Banking Systems for RepublicBank Dallas, as a principal at a Distribution Requirements Planning Company and then at Computer Associates where I single handedly designed and wrote the ACH settlement software system in six weeks.

In 1990 I started Trinity Technical Services with two partners and ended up buying them both out.

My first contract was the selection of an ERP system for a distributor in Texas.  The chosen vendor was CIM-JIT and it was my first exposure to Oracle. I soon
became more proficient in the CIM-JIT software than the vendor and flew around
country to modify to delivered software to comply with commitments to customers.

In the process I became quite proficient in fixing performance problems.

Convinced that Oracle was the finest relational database, I convinced Chuck to
come down to my Dallas office for a quick introduction.

He immediately saw the benefits of SQL over procedural file manipulation and
also became an Oracle Partner.

CDS somehow happened to land a contract to handle rebates for Leaf Gum.


\subsection{Technologies}
\label{\detokenize{Contributions:technologies}}

\subsubsection{Unix}
\label{\detokenize{Contributions:unix}}

\subsubsection{Oracle}
\label{\detokenize{Contributions:id1}}

\paragraph{Oracle Forms}
\label{\detokenize{Contributions:oracle-forms}}

\paragraph{Oracle Reports}
\label{\detokenize{Contributions:oracle-reports}}

\paragraph{PL/SQL}
\label{\detokenize{Contributions:pl-sql}}

\subsubsection{C}
\label{\detokenize{Contributions:id2}}

\subsubsection{Java}
\label{\detokenize{Contributions:id3}}

\subsubsection{JSP}
\label{\detokenize{Contributions:jsp}}

\subsubsection{Graphics}
\label{\detokenize{Contributions:graphics}}

\subsubsection{Javascript}
\label{\detokenize{Contributions:javascript}}

\subsubsection{Spreadsheets}
\label{\detokenize{Contributions:spreadsheets}}

\subsubsection{Digital Unix}
\label{\detokenize{Contributions:digital-unix}}

\subsubsection{Linux}
\label{\detokenize{Contributions:linux}}

\subsubsection{Build tools}
\label{\detokenize{Contributions:build-tools}}
Build scripts were shell scripts.

I introduced standardized build tools, first ant and then Maven


\subsubsection{Unit Tests}
\label{\detokenize{Contributions:unit-tests}}
I introduced junit for unit testing code and cobertura for code coverage analysis.


\subsubsection{Integration Tests}
\label{\detokenize{Contributions:integration-tests}}

\subsection{Chuck Retires}
\label{\detokenize{Contributions:chuck-retires}}
Each client had a separate database version 7 of Oracle.
I consolidated 10 databases into one, eliminated database links.

There was a daemon that ran that attempted to match customers on addresses.

I utilized javautils service to call the USPS API to actually standardize the
addresses for exact matches and identify undeliverable addresses


\subsubsection{Help}
\label{\detokenize{Contributions:help}}
Make files
C
Pro*C
Forms


\section{Business Model}
\label{\detokenize{Contributions:business-model}}
Custom Data Services was created by my brother Chuck.  I consulted to them for over 25 years and
taught him everything he knew from linux to relational databases, website creation, c and java.

In 2007 I accepted the position of CIO when my brother sold the company.  I gave up my consulting practice
on the promise that I could work remotely after my son went to college.

After I completely revised all of the infrastructure, which was horrible in more ways than can be credibly described
including the fact that they were seven years out of date on Oracle and had 14 instances running on three different
machines, I upgraded to Oracle 10 and migrated everything to one instance and finally into one schema.

After I sold my house and all my possessions and moved to Costa Rica, they fired me.


\subsection{While CIO}
\label{\detokenize{Contributions:while-cio}}\begin{itemize}
\item {} 
Document Management

\item {} 
Migrated off of Digital Unix

\item {} 
Changed backup procedures

\item {} 
Introduced Software Version Control

\item {} 
Created development and test databases

\item {} 
Trained a cafeteria worker and a pizza delivery person to program

\item {} 
Coded up the most complex spreadsheets for Frito

\item {} 
Introduced virtualization vmware, openvz, virtualbox

\item {} 
Upgraded to Oracle 10

\item {} 
Introduced Oracle Enterprise Manager

\item {} 
Materialized Views

\item {} 
Detachable Tablespaces

\item {} 
Virtual Private Databases

\item {} 
Spring inversion of control, hibernate, aspects

\item {} 
Unit Testing

\item {} 
Check 21 solution

\item {} 
Wikis for documentation procedures

\item {} 
xml

\item {} 
dexterous

\item {} 
workbook parser

\end{itemize}


\subsection{Terminated}
\label{\detokenize{Contributions:terminated}}
They claimed ownership of javautil despite the facts
* javautil.org was owned by Jim Schmidt
* javautil.com was owned by Jim Schmidt

The code base dated back to 1999.

CDS sued Jim Schmidt and claimed ownership of the code, but I would have had to have their current employees testify that
the code they were using was in production at my customer sites before I joined CDS.

A settlement was reached, I allowed them to continue to use the code as it was at them time  while I retained all rights.


\section{History of Jim Schmidt and Custom Data Solutions}
\label{\detokenize{Contributions:id4}}

\subsection{Creation}
\label{\detokenize{Contributions:id5}}
Custom Data Solutions was initially called CDS for Chuck and Dee Schmidt.

Chuck is my older brother and Dee is his wife.

Chuck left his job at Michigan National Bank to offer data processing services, initially using a TRS-80 to print mailing labels.

He subsequently bought an Alpha Micro \sphinxurl{http://www.s100computers.com/Hardware\%20Folder/Alpha\%20Micro/History/History.htm} and wrote some business applications.

On a visit to Detroit I explained to him the benefits of Unix and got him started, installing Interactive Unix from Kodak on a PC from 133 floppy disks in 1990.

Between 1983 and 1990 I worked as Vice President of International Banking Systems for RepublicBank Dallas, as a principal at a Distribution Requirements Planning Company and then at Computer Associates where I single handedly designed and wrote the ACH settlement software system in six weeks.

In 1990 I started Trinity Technical Services with two partners and ended up buying them both out.

My first contract was the selection of an ERP system for a distributor in Texas.  The chosen vendor was CIM-JIT and it was my first exposure to Oracle. I soon
became more proficient in the CIM-JIT software than the vendor and flew around
country to modify to delivered software to comply with commitments to customers.

In the process I became quite proficient in fixing performance problems.

Convinced that Oracle was the finest relational database, I convinced Chuck to
come down to my Dallas office for a quick introduction.

He immediately saw the benefits of SQL over procedural file manipulation and
also became an Oracle Partner.

CDS somehow happened to land a contract to handle rebates for Leaf Gum.


\subsection{Technologies}
\label{\detokenize{Contributions:id6}}

\subsubsection{Unix}
\label{\detokenize{Contributions:id7}}

\subsubsection{Oracle}
\label{\detokenize{Contributions:id8}}

\paragraph{Oracle Forms}
\label{\detokenize{Contributions:id9}}

\paragraph{Oracle Reports}
\label{\detokenize{Contributions:id10}}

\paragraph{PL/SQL}
\label{\detokenize{Contributions:id11}}

\subsubsection{C}
\label{\detokenize{Contributions:id12}}

\subsubsection{Java}
\label{\detokenize{Contributions:id13}}

\subsubsection{JSP}
\label{\detokenize{Contributions:id14}}

\subsubsection{Graphics}
\label{\detokenize{Contributions:id15}}

\subsubsection{Javascript}
\label{\detokenize{Contributions:id16}}

\subsubsection{Spreadsheets}
\label{\detokenize{Contributions:id17}}

\subsubsection{Digital Unix}
\label{\detokenize{Contributions:id18}}

\subsubsection{Linux}
\label{\detokenize{Contributions:id19}}

\subsubsection{Build tools}
\label{\detokenize{Contributions:id20}}

\subsubsection{Unit Tests}
\label{\detokenize{Contributions:id21}}

\subsubsection{Integration Tests}
\label{\detokenize{Contributions:id22}}

\subsection{Chuck Retires}
\label{\detokenize{Contributions:id23}}

\subsubsection{Help}
\label{\detokenize{Contributions:id24}}
Make files
C
Pro*C
Forms


\section{Business Model}
\label{\detokenize{Contributions:id25}}

\subsection{Custom Data Solutions}
\label{\detokenize{Contributions:custom-data-solutions}}
\sphinxurl{http://custdata.com}


\subsubsection{Relationship}
\label{\detokenize{Contributions:relationship}}
Custom Data Services was created by my brother Chuck.  I consulted to them for over 25 years and
taught him everything he knew from linux to relational databases, website creation, c and java.

In 2007 I accepted the position of CIO when my brother sold the company.  I gave up my consulting practice
on the promise that I could work remotely after my son went to college.

After I completely revised all of the infrastructure, which was horrible in more ways than can be credibly described
including the fact that they were seven years out of date on Oracle and had 14 instances running on three different
machines, I upgraded to Oracle 10 and migrated everything to one instance and finally into one schema.

After I sold my house and all my possessions and moved to Costa Rica, they fired me.

They stole my javautil code, there was a lawsuit, but I would have had to have their current employees testify that
the code they were using was in production at my customer sites before I joined CDS.

A settlement was reached, I allowed them to continue to use the code while retaining all rights.


\subsection{Processing Steps}
\label{\detokenize{Contributions:processing-steps}}
Data is uploaded to the data processing service.

The file loads are analyzed at CDS using an old version of my javautil conditionidentification package.

The data is loading into prepost tables.

The data is posted

A snapshot of the data is made once a day for online web reporting in the form of spreadsheets.

Rebate programs are designed and the sales are tracked and the rebates calculated.

Rebate checks are mailed out.


\section{Product Offering}
\label{\detokenize{ProductOffering:product-offering}}\label{\detokenize{ProductOffering::doc}}
So I can offer
\begin{itemize}
\item {} 
Data Reporting

\item {} 
Web Site for Data

\item {} 
Sales Analytics*

\item {} 
Address Validation and Correction

\item {} 
ACH processing for rebates

\item {} 
Availability to Vending operators

\item {} 
A version that runs on Frito/Pepsico laptops so they can work offline while traveling

\item {} 
High level management financial analysis

\end{itemize}


\subsection{Benefits}
\label{\detokenize{ProductOffering:benefits}}\begin{itemize}
\item {} 
Availability of site to operators for tracking rebate progress

\item {} 
Internationalization works outside the U.S.

\item {} 
Lower cost
\begin{itemize}
\item {} 
Using open source database allows for many database servers at minimal incremental cost

\item {} 
Using Asian resources is far less expensive

\end{itemize}

\end{itemize}


\subsection{How to get started}
\label{\detokenize{ProductOffering:how-to-get-started}}
I simply need
\begin{itemize}
\item {} 
the data that is currently reported by the distributors

\item {} 
product master list

\end{itemize}

No cost, no additional work, just send me the data and let me do my magic.


\section{Project Description}
\label{\detokenize{ProjectDescription:project-description}}\label{\detokenize{ProjectDescription::doc}}
This project showcases javautil.org functionality in a simple real world example.

This is a simple application the provides real functionality for sales reporting and rebate
processing.

The objective is to dive into a wide variety of technologies rather than go deep into any
given technology.

With a working app exploring additional features of this technology is easy.


\subsection{Software Description}
\label{\detokenize{ProjectDescription:software-description}}
Distributors of various products report their sales as specified by the data processor, custdata.com.  This
software reads the data currently being reported and
\begin{itemize}
\item {} 
Loads the data into ETL (Extract Transform and Load) tables

\item {} 
Runs a variety of tests to ensure the self consistency of the data and compliance with reporting rules.

\item {} 
Posts the data

\item {} 
Extracts data into various reporting formats

\item {} 
Analyses Data

\item {} 
Creates spreadsheets for users

\item {} 
Validates customer mailing addresses and standardizes them

\item {} 
Computes rebates

\item {} 
Provides Vending operator spreadsheets

\item {} 
Creates ACH files for submitting to a bank to issue rebates to customers

\end{itemize}


\subsection{Tech Steps}
\label{\detokenize{ProjectDescription:tech-steps}}
Install the required software
Configure for build
Build


\subsection{History}
\label{\detokenize{ProjectDescription:history}}
Processing Steps


\section{Technologies}
\label{\detokenize{Technologies:technologies}}\label{\detokenize{Technologies::doc}}

\subsection{Application}
\label{\detokenize{Technologies:application}}\begin{itemize}
\item {} \begin{description}
\item[{Java}] \leavevmode\begin{itemize}
\item {} 
JDBC

\item {} 
Hibernate

\end{itemize}

\end{description}

\item {} \begin{description}
\item[{Build Tools}] \leavevmode\begin{itemize}
\item {} 
Maven

\end{itemize}

\end{description}

\item {} \begin{description}
\item[{Postgres}] \leavevmode\begin{itemize}
\item {} 
SQL

\end{itemize}

\end{description}

\item {} \begin{description}
\item[{Testing}] \leavevmode\begin{itemize}
\item {} 
Unit Testing - SureFire

\item {} 
Integration Testing

\end{itemize}

\end{description}

\item {} 
Spring
* JSON RPC
* Dependency Injection
* Spring Boot

\item {} 
YAML

\item {} 
Sphinx

\item {} 
javautil

\end{itemize}


\subsection{Deployment}
\label{\detokenize{Technologies:deployment}}
Amazon Web Services with high availability and security, unlimited scalability and robust administration.

Service Contracts monthly.


\section{Record Layouts}
\label{\detokenize{cds_record_layout:record-layouts}}\label{\detokenize{cds_record_layout::doc}}
This details the reporting format as last posted by custdata.com
on their website.  This is available at

\begin{sphinxVerbatim}[commandchars=\\\{\}]
\PYG{n}{http}\PYG{p}{:}\PYG{o}{/}\PYG{o}{/}\PYG{n}{web}\PYG{o}{.}\PYG{n}{archive}\PYG{o}{.}\PYG{n}{org}\PYG{o}{/}\PYG{n}{web}\PYG{o}{/}\PYG{o}{*}\PYG{o}{/}\PYG{n}{custdata}\PYG{o}{.}\PYG{n}{com}
\end{sphinxVerbatim}


\subsection{Inventory}
\label{\detokenize{cds_record_layout:inventory}}

\begin{savenotes}\sphinxattablestart
\centering
\begin{tabulary}{\linewidth}[t]{|T|T|T|T|T|}
\hline
\sphinxstyletheadfamily 
field\_name
&\sphinxstyletheadfamily 
type
&\sphinxstyletheadfamily 
offset
&\sphinxstyletheadfamily 
length
&\sphinxstyletheadfamily 
format
\\
\hline
DISTRIBUTOR\_ID
&
DIGITS
&
0
&
10
&
\{:0\textgreater{}10s\}
\\
\hline
MFR\_ID
&
DIGITS
&
10
&
10
&
\{:0\textgreater{}10s\}
\\
\hline
MFR\_PRODUCT\_ID
&
DIGITS
&
20
&
8
&
\{:0\textgreater{}8s\}
\\
\hline
COMMENTS
&
TEXT
&
28
&
96
&
\{:\textless{}96s\}
\\
\hline
CASES
&
INTEGER
&
124
&
6
&
\{0:06d\}
\\
\hline
BOXES
&
INTEGER
&
130
&
6
&
\{0:06d\}
\\
\hline
UNITS
&
INTEGER
&
136
&
6
&
\{0:06d\}
\\
\hline
CASE\_GTIN
&
DIGITS
&
142
&
14
&
\{:0\textless{}14s\}
\\
\hline
FILLER
&
LITERAL
&
156
&
12
&
\{:\textless{}12s\}
\\
\hline
RECORD\_TYPE
&
LITERAL
&
168
&
2
&
\{:\textless{}2s\}
\\
\hline
\end{tabulary}
\par
\sphinxattableend\end{savenotes}


\subsection{Customer}
\label{\detokenize{cds_record_layout:customer}}

\begin{savenotes}\sphinxattablestart
\centering
\begin{tabulary}{\linewidth}[t]{|T|T|T|T|T|}
\hline
\sphinxstyletheadfamily 
field\_name
&\sphinxstyletheadfamily 
type
&\sphinxstyletheadfamily 
offset
&\sphinxstyletheadfamily 
length
&\sphinxstyletheadfamily 
format
\\
\hline
FILLER\_00\_05
&
LITERAL
&
0
&
6
&
\{:\textgreater{}6\}
\\
\hline
CLASS\_OF\_TRADE
&
TEXT
&
6
&
4
&
\{:\textless{}4s\}
\\
\hline
SHIP\_TO\_CUST\_ID
&
TEXT
&
10
&
10
&
\{:0\textgreater{}10s\}
\\
\hline
CUST\_NM
&
TEXT
&
20
&
30
&
\{:\textless{}30s\}
\\
\hline
ADDR\_1
&
TEXT
&
50
&
30
&
\{:\textless{}30s\}
\\
\hline
ADDR\_2
&
TEXT
&
80
&
30
&
\{:\textless{}30s\}
\\
\hline
CITY
&
TEXT
&
110
&
25
&
\{:\textless{}25s\}
\\
\hline
STATE
&
TEXT
&
135
&
2
&
\{:\textless{}2s\}
\\
\hline
POSTAL\_CD
&
DIGITS
&
137
&
9
&
\{:0\textgreater{}9s\}
\\
\hline
TEL\_NBR
&
DIGITS
&
146
&
10
&
\{:0\textgreater{}10s\}
\\
\hline
NATIONAL\_ACCT\_ID
&
TEXT
&
156
&
10
&
\{:0\textgreater{}10s\}
\\
\hline
SPECIAL\_FLG
&
TEXT
&
166
&
1
&
\{:\textgreater{}1s\}
\\
\hline
FILLER\_1
&
LITERAL
&
167
&
1
&
\{:\textgreater{}1s\}
\\
\hline
RECORD\_TYPE
&
LITERAL
&
168
&
2
&
\{:\textgreater{}2s\}
\\
\hline
\end{tabulary}
\par
\sphinxattableend\end{savenotes}


\subsection{Inventory Total}
\label{\detokenize{cds_record_layout:inventory-total}}

\begin{savenotes}\sphinxattablestart
\centering
\begin{tabulary}{\linewidth}[t]{|T|T|T|T|T|}
\hline
\sphinxstyletheadfamily 
field\_name
&\sphinxstyletheadfamily 
type
&\sphinxstyletheadfamily 
offset
&\sphinxstyletheadfamily 
length
&\sphinxstyletheadfamily 
format
\\
\hline
HEADER
&
LITERAL
&
0
&
10
&
\{:\textgreater{}1s\}
\\
\hline
FILLER36
&
LITERAL
&
10
&
36
&
\{:\textgreater{}36\}
\\
\hline
INVENTORY\_DT
&
DATE
&
46
&
8
&
\%Y\%m\%d
\\
\hline
FILE\_CREATION\_DT
&
DATE
&
54
&
8
&
\%Y\%m\%d
\\
\hline
RECORD\_CNT\_REPORTED
&
INTEGER
&
62
&
9
&
\{0:09d\}
\\
\hline
FILLER97
&
LITERAL
&
71
&
97
&
\{:\textgreater{}97\}
\\
\hline
RECORD\_TYPE
&
LITERAL
&
168
&
2
&
\{:\textgreater{}2\}
\\
\hline
\end{tabulary}
\par
\sphinxattableend\end{savenotes}


\subsection{Customer Total}
\label{\detokenize{cds_record_layout:customer-total}}

\begin{savenotes}\sphinxattablestart
\centering
\begin{tabulary}{\linewidth}[t]{|T|T|T|T|T|}
\hline
\sphinxstyletheadfamily 
field\_name
&\sphinxstyletheadfamily 
type
&\sphinxstyletheadfamily 
offset
&\sphinxstyletheadfamily 
length
&\sphinxstyletheadfamily 
format
\\
\hline
HEADER
&
LITERAL
&
0
&
10
&
\{:\textgreater{}10s\}
\\
\hline
FILLER\_127
&
LITERAL
&
10
&
127
&
\{:\textgreater{}127s\}
\\
\hline
CUSTOMER\_COUNT
&
INTEGER
&
137
&
9
&
\{0:09d\}
\\
\hline
FILLER\_22
&
LITERAL
&
146
&
22
&
\{:\textgreater{}22s\}
\\
\hline
RECORD\_TYPE
&
LITERAL
&
168
&
2
&
\{:\textgreater{}2s\}
\\
\hline
\end{tabulary}
\par
\sphinxattableend\end{savenotes}


\subsection{Sales}
\label{\detokenize{cds_record_layout:sales}}

\begin{savenotes}\sphinxattablestart
\centering
\begin{tabulary}{\linewidth}[t]{|T|T|T|T|T|}
\hline
\sphinxstyletheadfamily 
field\_name
&\sphinxstyletheadfamily 
type
&\sphinxstyletheadfamily 
offset
&\sphinxstyletheadfamily 
length
&\sphinxstyletheadfamily 
format
\\
\hline
DISTRIB\_ID
&
TEXT
&
0
&
10
&
\{:0\textgreater{}10s\}
\\
\hline
MFR\_ID
&
TEXT
&
10
&
10
&
\{:0\textgreater{}10s\}
\\
\hline
MFR\_PRODUCT\_ID
&
TEXT
&
20
&
8
&
\{:0\textgreater{}8s\}
\\
\hline
SHIP\_TO\_CUST\_ID
&
TEXT
&
28
&
10
&
\{:0\textgreater{}10s\}
\\
\hline
INVOICE\_CD
&
TEXT
&
38
&
10
&
\{:0\textgreater{}10s\}
\\
\hline
INVOICE\_DT
&
DATE
&
48
&
8
&
\%Y\%m\%d
\\
\hline
SHIP\_DT
&
DATE
&
56
&
8
&
\%Y\%m\%d
\\
\hline
FILLER
&
LITERAL
&
64
&
9
&
\{:\textless{}9s\}
\\
\hline
EXTENDED\_NET\_AMT
&
INTEGER
&
73
&
9
&
\{0:09d\}
\\
\hline
DISTRIB\_PRODUCT\_REF
&
TEXT
&
82
&
12
&
\{:\textless{}12s\}
\\
\hline
PRODUCT\_DESCR
&
TEXT
&
94
&
30
&
\{:\textless{}30s\}
\\
\hline
CASES\_SHIPPED
&
INTEGER
&
124
&
6
&
\{0:06d\}
\\
\hline
BOXES\_SHIPPED
&
INTEGER
&
130
&
6
&
\{0:06d\}
\\
\hline
UNITS\_SHIPPED
&
INTEGER
&
136
&
6
&
\{0:06d\}
\\
\hline
CASE\_GTIN
&
TEXT
&
142
&
14
&
\{:0\textgreater{}14s\}
\\
\hline
FILLER\_12
&
LITERAL
&
156
&
12
&
\{:\textgreater{}12s\}
\\
\hline
RECORD\_TYPE
&
LITERAL
&
168
&
2
&
\{:\textless{}2s\}
\\
\hline
\end{tabulary}
\par
\sphinxattableend\end{savenotes}


\subsection{Sales Total}
\label{\detokenize{cds_record_layout:sales-total}}

\begin{savenotes}\sphinxattablestart
\centering
\begin{tabulary}{\linewidth}[t]{|T|T|T|T|T|}
\hline
\sphinxstyletheadfamily 
field\_name
&\sphinxstyletheadfamily 
type
&\sphinxstyletheadfamily 
offset
&\sphinxstyletheadfamily 
length
&\sphinxstyletheadfamily 
format
\\
\hline
HEADER
&
LITERAL
&
0
&
10
&
\{:\textgreater{}10s\}
\\
\hline
FILLER\_28
&
LITERAL
&
10
&
28
&
\{:\textgreater{}28s\}
\\
\hline
SALES\_START\_DT
&
DATE
&
38
&
8
&
\%Y\%m\%d
\\
\hline
SALES\_END\_DT
&
DATE
&
46
&
8
&
\%Y\%m\%d
\\
\hline
FILE\_CREATE\_DT
&
DATE
&
54
&
8
&
\%Y\%m\%d
\\
\hline
SALES\_REC\_CNT
&
INTEGER
&
62
&
9
&
\{:0\textgreater{}9d\}
\\
\hline
SUM\_EXT\_NET\_AMT
&
INTEGER
&
71
&
11
&
\{:0\textgreater{}11d\}
\\
\hline
FILLER\_86
&
LITERAL
&
82
&
86
&
\{:\textgreater{}86s\}
\\
\hline
RECORD\_TYPE
&
LITERAL
&
168
&
2
&
\{:\textless{}2s\}
\\
\hline
\end{tabulary}
\par
\sphinxattableend\end{savenotes}



\renewcommand{\indexname}{Index}
\printindex
\end{document}